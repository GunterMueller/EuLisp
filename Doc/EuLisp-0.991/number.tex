\sclause{Numbers}
\index{general}{number}
\index{general}{number!module}
\label{number}
\index{general}{level-0 modules!number}
\index{general}{generic arithmetic}
%
\begin{optPrivate}
    Need to discuss issue of ordering and how that fits in with gaussian
    extensions.  Is it simply enough that there is no applicable method in each
    case?  Where is class number in this hierachy?  Perhaps number is ring?

    Ring defines 0, 1, +, -, * and possibly **.  Euclidean domain adds div,
    factor etc.  Field adds /, etc.
\end{optPrivate}
%
\begin{optRationale}
    In keeping with Lisp tradition, many of the arithmetic operators are n-ary.
    However, because \eulisp\ uses its generic function facility to describe the
    arithmetic operations---so that the user can add new arithmetics and
    integrate them with the rest of the system---and because n-argument
    discrimination cannot be satisfactorily defined, the actual generic
    arithmetic functions are all binary.  The meaning of each n-ary function in
    terms of the binary operator is defined by saying the binary operation is
    left-associative.  Thus, {\tt (+ a b c d)} must be evaluated as {\tt
        (binary+ (binary+ (binary+ a b) c) d)}.  The operators affected by this
    requirement are: \functionref{+}, \functionref{-}, \functionref{*},
    \functionref{/}, \functionref{max} and \functionref{min}.  The arithmetic
    comparison functions are also n-ary, so that {\tt (< a b c d)} must be
    evaluated in the same way.  The operators affected are: \functionref{=} and
    \functionref{<}.
\end{optRationale}
%
\begin{optDefinition}
\noindent
The defined name of this module is {\tt number}.

Numbers can take on many forms with unusual properties, specialized for
different tasks, but two classes of number suffice for the majority of needs,
namely integers (\classref{integer}, \classref{fpi}) and floating point numbers
(\classref{float}, \classref{double-float}).  Thus, these only are defined at
level-0.

In figure~\ref{level-0-number-class-hierarchy} shows the initial number class
hierarchy at level-0.  The inheritance relationships by this diagram are part of
this definition, but it is not defined whether they are direct or not.  For
example, \classref{integer}\ and \classref{float}\ are not necessarily direct
subclasses of \classref{number}, while the class of each number class might be a
subclass of \classref{number}.  Since there are only two concrete number
classes at level-0,
coercion\index{general}{coercion}\index{general}{number!coercion} is simple,
namely from \classref{fixed-precision-integer}\ to \classref{double-float}.  Any
level-0 version of a library module, for example, {\em elementary-functions}
\ref{elementary-functions}, need only define methods for these two classes.
Mathematically, the reals are regarded as a superset of the integers and for the
purposes of this definition we regard \classref{float}\ as a superset of
\classref{integer}\ (even though this will cause representation problems when
variable precision integers are introduced).  Hence, \classref{float}\ is
referred to as being {\em higher} that \classref{integer}\ and arithmetic
involving instances of both classes will cause integers to be converted to an
equivalent floating point value, before the calculation proceeds\footnote{This
    behaviour is popularly referred to as {\em floating point contagion}} (see
in particular \genericref{binary/},
\genericreflabel{binary\protect\%}{binarypercent} and \genericref{binary-mod}).

\begin{figure}[h]
\caption{Level-0 number class hierarchy}
\label{level-0-number-class-hierarchy}
\begin{center}
{\tt\begin{tabbing}
<num\=ber> \\
    \><flo\=at> \\
    \>    \><double-float> \\
    \><integer> \\
    \>    \><fixed-precision-integer>
\end{tabbing}}
\end{center}
\end{figure}

\derivedclass{number}{object}
%
The abstract class which is the superclass of all number classes.

\function{numberp}
%
\begin{arguments}
    \item[object] Object to examine.
\end{arguments}
%
\result%
Returns {\em object\/} if it is a number, otherwise \nil.

\condition{arithmetic-condition}{condition}

\begin{initoptions}
    \item[operator, object] The operator which signalled the condition.
    \item[operand-list, list] The operands passed to the operator.
\end{initoptions}
%
\remarks%
This is the general condition class for conditions arising from arithmetic
operations.

\condition{division-by-zero}{arithmetic-condition}
%
Signalled by any of \genericref{binary/},
\genericreflabel{binary\protect\%}{binarypercent} and \genericref{binary-mod} if
their second argument is zero.

\function{+}
%
\begin{arguments}
    \item[{\optional{number$_1$ number$_2$ ...}}] A sequence of numbers.
\end{arguments}
%
\result%
Computes the sum of the arguments using the generic function
\genericref{binary+}.  Given zero arguments, \functionref{+} returns {\tt 0} of
class \classref{integer}.  One argument returns that argument.  The arguments
are combined left-associatively.

\function{-}
%
\begin{arguments}
    \item[{number$_1$ \optional{number$_2$ ...}}] A non-empty sequence of
    numbers.
\end{arguments}
%
\result%
Computes the result of subtracting successive arguments---from the second to the
last---from the first using the generic function \genericref{binary-}.  Zero
arguments is an error.  One argument returns the negation of the argument, using
the generic function \genericref{negate}.  The arguments are combined
left-associatively.

\function{*}
%
\begin{arguments}
    \item[{\optional{number$_1$ number$_2$ ...}}] A sequence of numbers.
\end{arguments}
%
\result%
Computes the product of the arguments using the generic function
\genericref{binary*}.  Given zero arguments, \functionref{*} returns {\tt 1} of
class \classref{integer}.  One argument returns that argument.  The arguments
are combined left-associatively.

\function{/}
%
\begin{arguments}
    \item[{number$_1$ \optional{number$_2$ ...}}] A non-empty sequence of
    numbers.
\end{arguments}
%
\result%
Computes the result of dividing the first argument by its succeeding arguments
using the generic function \genericref{binary/}.  Zero arguments is an
error.  One argument computes the reciprocal of the argument.  It is an error in
the single argument case, if the argument is zero.

\functionlabeled{\protect\%}{percent}
%
\begin{arguments}
    \item[{number$_1$ \optional{number$_2$ ...}}] A non-empty sequence of
    numbers.
\end{arguments}
%
\result%
Computes the result of taking the remainder of dividing the first argument by
its succeeding arguments using the generic function
\genericreflabel{binary\protect\%}{binarypercent}.  Zero arguments is an error.
One argument returns that argument.

\function{gcd}
%
\begin{arguments}
    \item[{number$_1$ \optional{number$_2$ ...}}] A non-empty sequence of
    numbers.
\end{arguments}
%
\result%
Computes the greatest common divisor of {\em number$_1$} up to {\em number$_n$}
using the generic function \genericref{binary-gcd}.  Zero arguments is an error.  One
argument returns {\em number$_1$}.

\function{lcm}
%
\begin{arguments}
    \item[{number$_1$ \optional{number$_2$ ...}}] A non-empty sequence of
    numbers.
\end{arguments}
%
\result%
Computes the least common multiple of {\em number$_1$} up to {\em number$_n$}
using the generic function \genericref{binary-lcm}.  Zero arguments is an error.  One
argument returns {\em number$_1$}.

\function{abs}
%
\begin{arguments}
    \item[number] A number.
\end{arguments}
%
\result%
Computes the absolute value of {\em number}.

\generic{zerop}
%
\begin{genericargs}
    \item[number] A number.
\end{genericargs}
%
\result%
Compares {\em number\/} with the zero element of the class of {\em number\/}
using the generic function \genericref{binary=}.

\generic{negate}
%
\begin{genericargs}
    \item[number, \classref{number}] A number.
\end{genericargs}
%
\result%
Computes the additive inverse of {\em number}.

\function{signum}
%
\begin{arguments}
    \item[number] A number.
\end{arguments}
%
\result%
Returns {\em number\/} if \genericref{zerop} applied to {\em number\/} is true.
Otherwise returns the result of converting $\pm$1 to the class of {\em number\/}
with the sign of {\em number}.

\function{positivep}
%
\begin{arguments}
    \item[number] A number.
\end{arguments}
%
\result%
Compares {\em number\/} against the zero element of the class of {\em number\/}
using the generic function \genericref{binary<}.

\function{negativep}
%
\begin{arguments}
    \item[number] A number.
\end{arguments}
%
\result%
Compares {\em number\/} against the zero element of the class of {\em number\/}
using the generic function \genericref{binary<}.

\generic{binary+}
%
\begin{genericargs}
    \item[number$_1$, \classref{number}] A number.
    \item[number$_2$, \classref{number}] A number.
\end{genericargs}
%
\result%
Computes the sum of {\em number$_1$} and {\em number$_2$}.

\generic{binary-}
\begin{genericargs}
    \item[number$_1$, \classref{number}] A number.
    \item[number$_2$, \classref{number}] A number.
\end{genericargs}
%
\result%
Computes the difference of {\em number$_1$} and {\em number$_2$}.

\generic{binary*}
%
\begin{genericargs}
    \item[number$_1$, \classref{number}] A number.
    \item[number$_2$, \classref{number}] A number.
\end{genericargs}
%
\result%
Computes the product of {\em number$_1$} and {\em number$_2$}.

\generic{binary/}
%
\begin{genericargs}
    \item[number$_1$, \classref{number}] A number.
    \item[number$_2$, \classref{number}] A number.
\end{genericargs}
%
\result%
Computes the division of {\em number$_1$} by {\em number$_2$} expressed as a
number of the class of the higher of the classes of the two arguments.  The sign
of the result is positive if the signs the arguments are the same.  If the signs
are different, the sign of the result is negative.  If the second argument is
zero, the result might be zero or an error might be signalled (condition class:
\conditionref{division-by-zero}\indexcondition{division-by-zero}).

\genericlabeled{binary\protect\%}{binarypercent}
%
\begin{genericargs}
    \item[number$_1$, \classref{number}] A number.
    \item[number$_2$, \classref{number}] A number.
\end{genericargs}
%
\result%
Computes the value of {\em{}number$_1$}$-i*${\em{}number$_2$} expressed as a
number of the class of the higher of the classes of the two arguments, for some
integer $i$ such that, if {\em number$_2$} is non-zero, the result has the same
sign as {\em number$_1$} and magnitude less then the magnitude of {\em
    number$_2$}.  If the second argument is zero, the result might be zero or an
error might be signalled (condition class:
\conditionref{division-by-zero}\indexcondition{division-by-zero}).

\generic{binary-mod}
%
\begin{genericargs}
    \item[number$_1$, \classref{number}] A number.
    \item[number$_2$, \classref{number}] A number.
\end{genericargs}
%
\result%
Computes the largest integral value not greater than {\em
    number$_1$}$\over${\em{}number$_2$} expressed as a number of the class of
the higher of the classes of the two arguments, such that if {\em number$_2$} is
non-zero, the result has the same sign as {\em number$_2$} and magnitude less
than {\em number$_2$}.  If the second argument is zero, the result might be zero
or an error might be signalled (condition class:
\conditionref{division-by-zero}\indexcondition{division-by-zero}).

\generic{binary-gcd}
%
\begin{genericargs}
    \item[number$_1$, \classref{number}] A number.
    \item[number$_2$, \classref{number}] A number.
\end{genericargs}
%
\result%
Computes the greatest common divisor of {\em number$_1$} and {\em number$_2$}.

\generic{binary-lcm}
%
\begin{genericargs}
    \item[number$_1$, \classref{number}] A number.
    \item[number$_2$, \classref{number}] A number.
\end{genericargs}
%
\result%
Computes the lowest common multiple of {\em number$_1$} and {\em number$_2$}.
%
\end{optDefinition}
