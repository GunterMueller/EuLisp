\clause{Language Structure}
\label{subsec:lang-struct}
\index{general}{language structure}
%
\begin{optPrivate}
    Restriction on level-0 generics seems wierd.

    I [jwd] changed the paragraph on level-0 functions and level-0 applications
    to be, I think, clearer.  But we still don't say what a level-n function is
    for n > 0, we don't define level-n application for any n, and the claim that
    a level-n function is a level-(n-1) application may not be the right one.

    Someone queried use of define and proposed describe, but can't see how this
    helps.
\end{optPrivate}
%
\begin{optDefinition}
The \eulisp\ definition comprises the following items:
\begin{description}
    \item[Level-0]\index{general}{EuLisp@\eulisp!level-0} comprises all the
    level-0 classes, functions, macros and special forms\index{general}{special
        form}, which is this text minus Annex \ref{annex:level-1}.  The object
    system can be extended by user-defined structure classes, and generic
    functions.

    \item[Level-1]\index{general}{EuLisp@\eulisp!level-1} extends level-0 with
    the classes, functions, macros and special forms defined in Annex
    \ref{annex:level-1}.  The object system can be extended by user-defined
    classes and metaclasses.  The implementation of level-1 is not
    necessarily written or writable as a conforming level-0 program.
\end{description}
%
A {\em level-0 function\/} is a (generic) function defined in this text to be
part of a conforming processor for level-0.  A function defined in terms of
level-0 operations is an example of a {\em level-0 application}.

Much of the functionality of \eulisp\ is defined in terms of
modules\index{general}{standard
    module}\index{general}{EuLisp@\eulisp!libraries}.  These modules might be
available (and used) at any level, but certain modules are required at a given
level.  Whenever a module depends on the operations available at a given level,
that dependency will be specified.

\eulisp\ level-0\index{general}{level-0 modules} is provided by the module {\tt
    eulisp0}\index{general}{level-0 modules!eulisp0}.  This module imports and
re-exports the modules specified in table~\ref{level-0-modules}.

Modules comprising {\tt eulisp0}:
\begin{center}
    \label{level-0-modules}
    \begin{tabular}{|ll|}\hline
        Module & Section(s)\\\hline
        {\tt character} & \ref{character}\\
        {\tt collection} & \ref{collection}\\
        {\tt compare} & \ref{compare}\\
        {\tt condition} & \ref{condition}\\
        {\tt convert} & \ref{convert}\\
        {\tt copy} & \ref{copy}\\
        {\tt double} & \ref{double-float}\\
        {\tt fpi} & \ref{fpi}\\
        {\tt formatted-io} & \ref{formatted-io}\\
        {\tt function} & \ref{function}\\
        {\tt keyword} & \ref{keyword}\\
        {\tt list} & \ref{list}\\
        {\tt lock} & \ref{lock}\\
        {\tt mathlib} & \ref{elementary-functions}\\
        {\tt number} & \ref{number}\\
        {\tt telos0} & \ref{obj-0}\\
        {\tt stream} & \ref{stream}\\
        {\tt string} & \ref{string}\\
        {\tt symbol} & \ref{symbol}\\
        {\tt table} & \ref{table}\\
        {\tt thread} & \ref{thread}\\
        {\tt vector} & \ref{vector}\\\hline
    \end{tabular}
\end{center}
%
\noindent
This definition is organized in three parts:
\begin{description}
    %
    \item[\S \ref{syntax}--\ref{control-0}] describes the core of level-0
    of \eulisp, covering modules, simple classes, objects and generic functions,
    threads, conditions, control forms and events.  These sections contain the
    information about \eulisp\ that characterizes the language.
    %
    \item[Annex A] describes the additional classes required at level-0 and the
    operations defined on instances of those classes\index{general}{level-0
        classes}.  The annex is organized by module in alphabetical order.
    These sections contain information about the predefined classes in \eulisp\
    that are necessary to make the language usable, but is not central to an
    appreciation of the language.
    %
    \item[Annex B] describes the reflective aspects of the object system and how
    to program the metaobject protocol and some additional control forms.
\end{description}
%
Prior to these, sections \ref{scope}--\ref{sec:definitions} define the scope of
the text, cite normative references, conformance definitions, error definitions,
typographical and layout conventions and terminology definitions used in this
text.
\end{optDefinition}
%
\clause{Scope}\label{scope}
\begin{optDefinition}
This text specifies the syntax and semantics of the computer
programming language \eulisp\ by defining the requirements for a
conforming \eulisp\ processor and a conforming \eulisp\ program (the
textual representation of data and algorithms).
%
% {\em\begin{verse}
%         NOTE---the term ``program'' is avoided in this definition because of
%         the ambiguity in the representation of such an entity.
%     \end{verse}}
%
\noindent
This text does not specify:
\begin{enumerate}
    %
    \item The size or complexity of an \eulisp\ program that will exceed the
    capacity of any specific configuration or processor, nor the actions to be
    taken when those limits are exceeded.
    %
    \item The minimal requirements of a configuration that is capable of
    supporting an implementation of a \eulisp\ processor.
    %
    \item The method of preparation of a \eulisp\ program for execution or the
    method of activation of this \eulisp\ program once prepared.
    %
    \item The method of reporting errors, warnings or exceptions to the client
    of a \eulisp\ processor.
    %
    \item The typographical representation of a \eulisp\ program for human
    reading.
    %
    \item The means to map module names to the filing system or other object
    storage system attached to the processor.
\end{enumerate}
%
To clarify certain instances of the use of English in this text the following
definitions are provided:
\begin{description}
    \item[must] a verbal form used to introduce a {\em required} property.  All
    conforming processors must satisfy the property.
    %
    % \item[shall:] A verbal form used to introduce a {\em required}
    % property.  All conforming processors must satisfy the property.
    %
    \item[should] A verbal form used to introduce a {\em strongly recommended}
    property.  Implementors of processors are urged (but not required) to
    satisfy the property.
    %
\end{description}
\end{optDefinition}
%
\clause{Normative References}\index{general}{normativeReferences}
\begin{optDefinition}
The following standards contain provisions, which through references
in this text constitute provisions of this definition.  At the
time of writing, the editions indicated were valid.  All standards are
subject to revision and parties making use of this definition are
encouraged to apply the most recent edition of the standard listed
below.
\begin{references}
    %
    \bibitem[ISO 646] {iso646} Information processing --- ISO 7-bit coded
    character set for information interchange, 1983.  Note: this standard is
    currently under revision and the interested reader should see the 1990 Draft
    International Standard version of ISO/IEC 646.
    %
    \bibitem[ISO 2382] {iso2382} Data processing --- vocabulary.
    %
    \bibitem[ISO TR 10034 : 1990] {iso10034} Information technology ---
    Guidelines for the preparation of conformity clauses in programming language
    standards.
    %
    \bibitem[ISO TR 10176 : 1991] {iso10176} Information technology ---
    Guidelines for the preparation of programming language standards. Note: this
    is currently a draft technical report.
    %
    \bibitem[ISO/IEC 9899:1999] {iso9899} Programming Languages --- C.
    %
\end{references}
\end{optDefinition}
%
\clause{Conformance Definitions}\index{general}{conformance}
\label{subsubsec:cse}
\begin{optPrivate}
    JWD: What I think's happened is that a number of convenient definitions have
    been put it, but we haven't yet sorted out just which ones we want and (in
    some cases) just what they should mean.  Anyway, here are some questions:

    Why isn't "processor-undefined" just "undefined"?

    "Portable code" refers to "conforming code", but the latter is not defined.

    It seems that, by omission, "conforming programs" can depend on
    "processor-dependent" behaviour.

    If we're going to have both "portable code" and "conforming program", there
    ought to be some straightforward relationship between them.

    We need to relate the later parts of 2.3.1 to the earlier parts.  For
    example, what are the implications for conformance when something "is an
    error"?

    The first two paragraphs of [section 2.5] give two different definitions of
    "variable".  The second is the Scheme definition.

    The definition of "variable" in the description env isn't quite a
    definition.  It says "denotes a binding ...".

    "Extent" can be for things other than bindings.

    "Dynamic scope" is defined in terms of "control region" which is not
    defined.

    The order of definitions in the section 2.5 description env looks reasonable
    of you look just at the items, but I think it would be easier to read of the
    different kinds of "scope" ("extent") were defined together rather than the
    different applications of "indefinite" ("dynamic").

    Updated conformance section as per draft guidelines referenced in the text.

    910527---added Klaus' terminology with slight adaptations.
\end{optPrivate}
%
\begin{optDefinition}
The following terms are general in that they could be applied to the
definition of any programming language.  They are derived from ISO/IEC
TR 10034: 1990.
%
\begin{definitions}
    \definition{configuration}\index{general}{configuration} Host and target
    computers, any operating systems(s) and software (run-time system) used to
    operate a language {\em processor}.
    %
    \definition{conformity clause}\index{general}{conformity clause} Statement
    that is not part of the language definition but that specifies requirements
    for compliance with the language standard.
    %
    \definition{conforming program}\index{general}{conforming program} Program
    which is written in the language defined by the language standard and which
    obeys all the {\em conformity} {\em clauses} for programs in the language
    standard.
    %
    \definition{conforming processor}\index{general}{conforming processor} {\em
        Processor} which processes {\em conforming} {\em programs} and program
    units and which obeys all the {\em conformity} {\em clauses} for {\em
        processors} in the language standard.
    %
    \definition{error}\index{general}{error} Incorrect program construct or
    incorrect functioning of a program as defined by the language standard.
    %
    \definition{extension}\index{general}{extension} Facility in the {\em
        processor} that is not specified in the language standard but that does
    not cause any ambiguity or contradiction when added to the language
    standard.
    %
    \definition{implementation-defined}\index{general}{implementation-defined}
    Specific to the {\em processor}, but required by the language standard to be
    defined and documented by the implementer.
    %
    \definition{processor}\index{general}{processor} Compiler, translator or
    interpreter working in combination with a {\em configuration}.
    %
\end{definitions}
\end{optDefinition}
%
\clause{Error Definitions}\index{general}{errors}
\begin{optDefinition}
Errors in the language described in this definition fall into one of
the following three classes:
%
\begin{definitions}
%
    \definition{static error}\index{general}{static
        error}\index{general}{error!static} An error which is detected during
    the execution of a \eulisp\ program or which is a violation of the defined
    behaviour of \eulisp.  Static errors have two classifications:
    \begin{enumerate}
        \item Where a {\em conforming} {\em processor} is required to detect the
        erroneous situation or behaviour and report it.  This is signified by
        the phrase {\em an error is signalled}\index{general}{error!signalled}.
        The condition class to be signalled is specified.  Signalling an error
        consists of identifying the condition class related to the error and
        allocating an instance of it.  This instance is initialized with
        information dependent on the condition class.  A conforming \eulisp\
        program can rely on the fact that this condition will be signalled.

        \item Where a {\em conforming} {\em processor} might or might not detect
        and report upon the error.  This is signified by the phrase {\em \ldots
            is an error}.  A processor should provide a mode where such errors
        are detected and reported where possible\index{general}{error!can be
            signalled}.
    \end{enumerate}
    If the result of preparation is a runnable program, then that program must
    signal any static error.

    \definition{environmental error}\index{general}{environmental
        error}\index{general}{error!environmental} An error which is detected by
    the configuration supporting the \eulisp\ processor.  The processor must
    signal the corresponding static error which is identified and handled as
    described above.

    \definition{violation}\index{general}{error}\index{general}{violation}
    \index{general}{error!violation} An error which is detected during the
    preparation of a \eulisp\ program for execution, such as a violation of the
    syntax or static semantics of \eulisp\ in the program under preparation.  A
    {\em conforming} {\em processor} is required to issue a diagnostic if a
    violation is detected.
\end{definitions}

All errors specified in this definition are static unless explicitly stated
otherwise.
\end{optDefinition}
%
\clause{Compliance}\index{general}{compliance}
\begin{optDefinition}
An \eulisp\ processor can conform\index{general}{compliance} at either of the
two levels defined under Language Structure in the Introduction.
Thus a level-0\index{general}{conformance!level-0} conforming processor
must support all the basic expressions, classes and class operations
defined at level-0.  A level-1\index{general}{conformance!level-1}
conforming processor must support all the basic expressions, classes,
class operations and modules defined at level-0 and at level-1.

The following two statements govern the conformance of a processor at a given
level.
\begin{enumerate}
    %
    \item A {\em conforming processor\/} must correctly process all programs
    conforming both to the standard at the specified level and the {\em
        implementation-defined} features of the {\em processor}.
    %
    \item A {\em conforming processor\/} should offer a facility to report the
    use of an {\em extension} which is statically determinable solely from
    inspection of a program, without execution.  (It is also considered
    desirable that a facility to report the use of an {\em extension} which is
    only determinable dynamically be offered.)
    %
\end{enumerate}
%
A level-0\index{general}{conformance!level-0} conforming program is one which
observes the syntax and semantics defined for level-0.  A level-0 conforming
program might not conform at level-1.  A {\em strictly-conforming\/} level-0
program is one that also conforms at level-1.  A level-1 conforming program
observes the syntax and semantics defined for level-1.

In addition, a {\em conforming program\/} at any level must not use any {\em
    extensions\/} implemented by a language {\em processor\/}, but it can rely
on {\em implementation-defined\/} features.

\noindent
The documentation of a {\em conforming processor\/} must include:
\begin{enumerate}
    %
    \item A list of all the {\em implementation-defined\/} definitions or
    values.
    %
    \item A list of all the features of the language standard which are
    dependent on the {\em processor\/} and not implemented by this {\em
        processor\/} due to non-support of a particular facility, where such
    non-support is permitted by the standard.
    %
    \item A list of all the features of the language implemented by this {\em
        processor\/} which are {\em extensions\/} to the standard language.
    %
    \item A statement of conformity, giving the complete reference of the
    language standard with which conformity is claimed, and, if appropriate, the
    level of the language supported by this processor.
    %
\end{enumerate}
\end{optDefinition}
%
\clause{Conventions}\index{general}{conventions}
\label{conventions}
\begin{optPrivate}
    Should note error behaviour in reference to constant module binding (RPG).
    Reorganize the naming convention stuff into a list layout to make it easier
    to read---done (JAP).  What is the original value that is restored on exit?
    (RPG).  Various undefined terms are: object, class, metaclass, function,
    control region, closure, continuation, closer lexically, closer dynamically.
    Note: missing sample names for ``c'' and ``q'' suffixes.

    Someone proposed should define notation for rest lists here, but since they
    do not (and cannot?) occur in the definition, I have not done anything about
    it.
\end{optPrivate}
%
\begin{optDefinition}
This section defines the conventions employed in this text, how
definitions will be laid out, the typefaces to be used, the
meta-language used in descriptions and the naming conventions.  A
later section (\ref{sec:definitions}) contains definitions of the
terms used in this text.

A standard function\index{general}{function!standard function} denotes an
immutable top-lexical binding of the defined name.  All the definitions in this
text are bindings in some module except for the special form operators, which
have no definition anywhere.  All bindings and all the special form operators
can be renamed.
%
\begin{note}
    A description making mention of ``an x'' where ``x'' is the name a
    class usually means ``an instance of {\tt <x>}''.
\end{note}
%
Frequently, a class-descriptive name will be used in the argument list of a
function description to indicate a restriction on the domain to which that
argument belongs.  In the case of a function, it is an error to call it with a
value outside the specified domain.  A generic function can be defined with a
particular domain and/or range.  In this case, any new methods must respect the
domain and/or range of the generic function to which they are to be attached.
The use of a class-descriptive name in the context of a generic function
definition defines the intention of the definition, and is not necessarily a
policed restriction.

The {\em result-class} of an operation, except in one case, refers to a
primitive or a defined class described in this definition.  The exception is for
predicates.  Predicates are defined to return either the empty list---written
\nil---representing the boolean value false\index{general}{false}, or any
value other than \nil, representing true\index{general}{true}.  Although the
class containing exactly this set of values is not defined in the language,
notation is abused for convenience and {\em boolean}\index{general}{boolean} is
defined, for the purposes of this report, to mean that set of values.  If the
true value is a useful value, it is specified precisely in the description of
the function and written as \true.
\end{optDefinition}
%
\sclause{Layout and Typography}
\begin{optDefinition}
Both layout and fonts are used to help in the description of \eulisp.  A
language element is defined as an entry with its name as the heading of a
clause, coupled with its classification.  The syntax notation used is based on
that described in \cite{iso9899} with modifications to support the specification
of a return type and to improve clarity.  Syntactic categories (non-terminals)
are indicated by {\it italic} type, and literal words and characters (terminals)
by {\tt constant width} type.  A colon ({\it :}) following a non-terminal
introduces its definition.  Alternative definitions are listed on separate
lines, except when prefaced by the words ``one of''.  An optional symbol is
indicated by the subscript ``{\it opt}'', a list of zero or more occurrences of
a symbol are indicated by the superscript ``{\it *}'', and a list of one or more
occurrences of a symbol are indicated by the superscript ``{\it +}''.  Examples
of several kinds of entry are now given.  Some subsections of entries are
optional and are only given where it is felt necessary.
%
\keyword{a-special-form}
\Syntax
\saveSyntax{a-special-form}{
\begin{syntaxx}
    \scdef{a-special-form-form}: $\rightarrow$ \sc{result-class} \\
    \>( a-special-form \sc{form-1} ... \scopt{form-n} )
\end{syntaxx}}
\syntaxTable{a-special-form}
%
\begin{arguments}
    \item[\sc{form-1}] description of structure and r\^ole of \sc{form-1}.\\
    $\vdots$
    \item[\scopt{form-n}] description of structure and r\^ole of optional
    argument \scopt{form-n}.
\end{arguments}
%
\result%
A description of the result and, possibly, its \sc{result-class}.
%
\remarks%
Any additional information defining the behaviour of {\tt a-special-form} or the
syntax category \scref{a-special-form-form}.
%
\examples
Some examples of use of the special form and the behaviour that should
result.
%
\seealso%
Cross references to related entries.
%
\function{a-function}
%
\Signature
\saveSignature{a-function}{
\begin{signature}
    (a-function \sc{argument-1} ... \scopt{argument-n})\\
    \>$\rightarrow$ \sc{result-class}
\end{signature}}
\signatureTable{a-function}
%
\begin{arguments}
    \item[\sc{argument-1}] information about the class or classes of
    \sc{argument-1}.\\
    $\vdots$
    \item[\scopt{argument-n}] information about the class or classes of
    the optional argument \sc{argument-n}.
\end{arguments}
%
\result%
A description of the result and, possibly, its \sc{result-class}.
%
\remarks%
Any additional information about the actions of {\tt a-function}.
%
\examples
Some examples of calling the function with certain arguments and the
result that should be returned.
%
\seealso%
Cross references to related entries.
%
\generic{a-generic}
%
\begin{genericargs}
    \item[argument-a, <class-a>] means that {\em argument-a} of {\tt a-generic}
    must be an instance of {\tt <class-a>} and that {\em argument-a} is one of
    the arguments on which {\tt a-generic} specializes.  Furthermore, each
    method defined on {\tt a-generic} may
    specialize only on a subclass of {\tt <class-a>} for {\em argument-a}.\\
    $\vdots$
    \item[argument-n] means that (i) {\em argument-n} is an instance of {\tt
        <object>}, {\em i.e.} it is unconstrained, (ii) {\tt a-generic} does not
    specialize on {\em argument-n}, (iii) no method on {\tt a-generic} can
    specialize on {\em argument-n}.
\end{genericargs}
%
\result%
A description of the result and, possibly, its class.
%
\remarks%
Any additional information about the actions of {\tt a-generic}.  This
can take two forms depending on the function.  This will probably be
in general terms, since the actual behaviour will be determined by the
methods.
%
\seealso%
Cross references to related entries.
%
\method{a-generic}
%
(A method on {\tt a-generic} with the following specialized arguments.)
%
\begin{specargs}
    \item[argument-a, <class-a>] means that {\em argument-a} of the method must
    be an instance of {\tt <class-a>}.  Of course, this argument must be one
    which was defined in {\tt a-generic} as being open to
    specialization and {\tt <class-a>} must be a subclass of the class.\\
    $\vdots$
    \item[argument-n] means that (i) {\em argument-n} is an instance of {\tt
        <object>}, {\em i.e.} it is unconstrained, because {\tt a-generic} does
    not specialize on {\em argument-n}.
\end{specargs}
%
\result%
A description of the result and, possibly, its class.
%
\remarks%
Any additional information about the actions of this method attached
to {\tt a-generic}.
%
\seealso%
Cross references to related entries.
%
\condition{a-condition}{condition}
%
\begin{initoptions}
    \item[initarg-a, value-a] means that an instance of {\tt <a-condition>} has
    a slot called {\tt initarg-a} which should be initialized to {\em value-a},
    where {\em value-a} is often the name of a class, indicating that {\em
        value-a} should be an instance of that class and a description of the
    information that {\em value-a} is
    supposed to provide about the exceptional situation that has arisen.\\
    $\vdots$
    \item[initarg-n, value-n] As for {\tt initarg-a}.
\end{initoptions}
%
\remarks%
Any additional information about the circumstances in which the
condition will be signalled.
%
\derivedclass{a-class}{class}
%
\begin{initoptions}
    \item[initarg-a, value-a] means that {\tt <a-class>} has an
    initarg whose name is {\tt initarg-a} and the description will usually
    say of what class (or classes) {\em value-a} should be an instance.
    This initarg is required.\\ $\vdots$
    \item[{\tt[}initarg-n, value-n{\tt]}]
    The enclosing square brackets denote that this initarg is optional.
    Otherwise the interpretation of the definition is as for {\tt
        initarg-a}.
\end{initoptions}
%
\remarks%
A description of the r\^ole of {\tt <a-class>}.

\constant{a-constant}{a-class}
%
\remarks%
A description of the constant of type \classref{a-class}.

\constant{<a-class-alias>}{a-class}
%
\remarks%
A description of the r\^ole of the constant binding whose value is
\classref{a-class}, {\em i.e.} the class alias.

\end{optDefinition}
%
\sclause{Naming}
%
\begin{optPrivate}
    Consider adopting Scheme conventions for destructive and predicate
    functions.
\end{optPrivate}
%
\begin{optDefinition}
Naming conventions are applied in the descriptions of primitive and
defined classes as well as in the choice of other function names.
Here is a list of the conventions and some examples of their use.
%
\subdefinition{``{\tt <name>}'' wrapping} By convention, classes have names
which begin with ``{\tt <}'' and end with ``{\tt >}''.
%
\subdefinition{``{\tt binary}'' prefix} The two argument version of a n-ary
argument function.  For example \genericref{binary+} is the two argument
(generic) function corresponding to the n-ary argument \functionref{+} function.
%
% \subdefinition{``{\tt c}'' suffix}
% The version of the function named {\tt foo} that is
% guaranteed to terminate on self-referential structures is usually
% named {\tt fooc}.
%
\subdefinition{``{\tt -class}'' suffix} The name of a metaclass of a set of
related classes.  For example, \classref{function-class}, which is the class of
\classref{simple-function}, \classref{generic-function} and any of their
subclasses.  The exception is \classref{class} itself which is the default
metaclass.  The prefix should describe the general domain of the classes in
question, but not necessarily any particular class in the set.
%
\subdefinition{``{\tt generic-}'' prefix} The generic version of the function
named by the stem.
% ---usually
% required to allow for optional or variable numbers of arguments to
% {\tt abc}.
%
\subdefinition{hyphenation} Function and class names made up of more than one
word are hyphenated, for example: {\tt
    compute-specialized-slot-class}.
%
% \subdefinition{``{\tt make-}'' prefix}
% For most primitive or defined classes there is constructor function,
% which is usually named {\tt make-}{\em class-name}---except where
% historical precedent is strong, for example, {\tt cons} is used in
% preference to {\tt make-pair}.
%
% \subdefinition{``{\tt n}'' prefix}
% The destructive version of the function named {\tt foo} is usually
% named {\tt nfoo}, for example the destructive version of {\tt reverse}
% is named {\tt nreverse}.
%
\subdefinition{``{\tt p}'' suffix} A predicate function is named by a ``{\tt
    p}'' suffix if the function or class name (after removing the enclosing {\tt
    <} and {\tt >}) is not hyphenated, for instance, \functionref{consp}, and is
named by a ``{\tt -p}'' suffix if it is, for instance
\functionref{double-float-p}.
%
% \subdefinition{``{\tt q}'' suffix}
% The version of the function named {\tt foo} that uses {\tt eq} for
% comparison is usually named {\tt fooq}.
%
% \subdefinition{``{\tt -ref}'' suffix}
% A slot reader is named {\em class-name}{\tt{}-ref}---where
% appropriate---and as is the corresponding slot writer {\tt (setter
%     {\em class-name}-ref)}---also where appropriate, for example {\tt
%     table-ref}.  This convention is broken by historical precedent for the
% accessors to slots of pairs, which are called {\tt car} and {\tt cdr}.
%
% \subdefinition{``{\tt -using-}'' infix}
% This function name convention is used when calls are cascaded using
% objects derived from the original object so that users can write
% methods on the relevant classes.  For example, in the case of {\tt
%     slot-value}, the class and slot description corresponding to the
% original object and slot name are retrieved, and the new generic
% functions are called ({\tt slot-value-using-class} and {\tt
%     slot-value-using-slot}).
%
% When an operation depends on a global value, such as the current input
% stream, the value might also be made accessible via a function, for
% example {\tt (a-global-value)} and can be updated using the
% corresponding setter function, for example {\tt ((setter
%     a-global-value) another-value)}.
%
\end{optDefinition}
