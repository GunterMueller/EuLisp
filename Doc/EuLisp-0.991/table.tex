\sclause{Tables}
\label{table}
\index{general}{table}
\index{general}{table!module}
\index{general}{level-0 modules!table}
%
\begin{optRationale}
    Operationally, tables resemble hashtables, but the actual representation is
    not defined in order to permit alternative solutions, such as various forms
    of balanced trees.
\end{optRationale}
%
\begin{optDefinition}
The defined name of this module is {\tt table}.  See also
section~\ref{collection} (collections) for further operations on tables.

\derivedclass{table}{collection}
\index{general}{level-0 classes!\theclass{table}}
%
The class of all instances of \classref{table}.
%
\begin{initoptions}
    \item[comparator, \classref{function}]%
    The function to be used to compare keys.  The default comparison function is
    \functionref{eql}.

    \item[fill-value, \classref{object}]%
    An object which will be returned as the default value for any key which does
    not have an associated value.  The default fill value is \nil{}.

    \item[hash-function, \classref{function}]%
    The function to be used to compute an unique key for each object stored in
    the table.  This function must return a fixed precision integer.  The hash
    function must also satisfy the constraint that if the comparison function
    returns true for any two objects, then the hash function must return the
    same key when applied to those two objects.  The default is an
    implementation defined function which satisfies these conditions.
\end{initoptions}

\function{tablep}
%
\begin{arguments}
    \item[object] Object to examine.
\end{arguments}
%
\result%
Returns {\em object\/} if it is a table, otherwise \nil{}.

\function{clear-table}
%
\begin{arguments}
    \item[table] A table.
\end{arguments}
%
\result%
An empty table.
%
\remarks%
All entries in {\em table\/} are deleted.  The result is \functionref{eq} to the
argument, which is to say that the argument is modified.
%
\end{optDefinition}
