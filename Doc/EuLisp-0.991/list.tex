\sclause{Lists}
\index{general}{null}
\label{null}
\index{general}{list}
\index{general}{list!module}
\label{list}
\index{general}{pair}
\label{pair}
%
\begin{optDefinition}
The name of this module is {\tt list}.  The class \classref{list}\ is an
abstract class and has two subclasses: \classref{null}\ and \classref{cons}.
The only instance of \classref{null}\ is the empty list.  The combination of
these two classes allows the creation of proper lists, since a proper list is
one whose last pair contains the empty list in its \functionref{cdr} field.  See
also section~\ref{collection} (collections) for further operations on lists.

% The class \syntaxref{pair}\index{general}{pair} (also known as a {\em dotted
%     pair}) is a 2-tuple, whose fields are called, for historical reasons, car
% and cdr.  Pairs are created by the function \functionref{cons} and the fields
% are accessed by the functions \functionref{car} and \functionref{cdr}.  The
% major use of pairs is in the construction of (proper) lists.  A (proper)
% list\index{general}{list} is defined as either the empty list (denoted by {\tt
%     '()}) or a pair whose \functionref{cdr} is a proper list.  An improper
% list is one containing a \functionref{cdr} which is not a list.

\derivedclass{list}{collection}
\index{general}{level-0 classes!\theclass{list}}
%
The class of all lists.

\syntaxform{()}
%
\remarks%
The empty list\index{general}{external representation!null (empty list)},
which is the only instance of the class \classref{null}, has as its
external representation an open parenthesis followed by a close
parenthesis.  The empty list is also used to denote the boolean value
{\em false}.

\derivedclass{null}{list}
\index{general}{level-0 classes!\theclass{null}}

The class whose only instance is the empty list, denoted \nil.

\function{null}
%
\begin{arguments}
    \item[object] Object to examine.
\end{arguments}
%
\result%
Returns \true\/ if {\em object} is the empty list, otherwise \nil.

\method{generic-prin}
%
\begin{specargs}
    \item[null] The empty list.
    \item[stream] The stream on which the representation is to be output.
\end{specargs}
%
\result%
The empty list.
%
\remarks%
Output the external representation of the empty list on {\em stream\/}
as described above.

\method{generic-write}
%
\begin{specargs}
    \item[null] The empty list.
    \item[stream] The stream on which the representation is to be output.
\end{specargs}
%
\result%
The empty list.
%
\remarks%
Output the external representation of the empty list on {\em stream\/}
as described above.

\syntaxform{pair}
%
A pair\index{general}{external representation!pair} is written as \verb+(+{\em
    object}$_1$ \verb+.+ {\em object}$_2$\verb+)+, where {\em object}$_1$ is
called the \functionref{car} and {\em object}$_2$ is called the
\functionref{cdr}.  There are two special cases in the external representation
of pair.  If {\em object}$_2$ is the empty list, then the pair is written as
\verb+(+{\em object}$_1$\verb+)+.  If {\em object$_2$} is an instance of
\syntaxref{pair}, then the pair is written as \verb+(+{\em object}$_1$ {\em
    object}$_3$ . {\em object$_4$}\verb+)+, where {\em object$_3$} is the
\functionref{car} of {\em object$_2$} and {\em object$_4$} is the
\functionref{cdr} with the above rule for the empty list applying.  By
induction, a list of length $n$ is written as\index{general}{external
    representation!list} \verb+(+{\em object}$_1$ \ldots {\em object$_{n-1}$}
. {\em object}$_n$\verb+)+, with the above rule for the empty list applying.
The representations of {\em object$_1$} and {\em object$_2$} are determined by
the external representations defined in other sections of this definition (see
\classref{character}\ (\ref{character}), \classref{double-float}\
(\ref{double-float}),
% \classref{null}\ (\ref{null}),
\classref{fixed-precision-integer}\ (\ref{spint}),
\classref{string}\ (\ref{string}),
\classref{symbol}\ (\ref{symbol}) and
\classref{vector}\ (\ref{vector}), as well as instances of \classref{cons}\
itself.  The syntax for the external representation of pairs and lists
is defined in Table~\ref{pair-syntax}.

\Syntax
\label{pair-syntax}
\savesyntax\listSyntax\vbox{\small\syntax
null
   = '(', ')';
pair
   = '(', object, '.', object, ')';
list
   = empty list | proper list | improper list;
empty list
   = '()';
proper list
   = '(', {object}, ')';
improper list
   = '(', {object}, '.', object, ')';
\endsyntax}
\syntaxtable{list}{\listSyntax}
%
\examples
\begin{tabular}{ll}
    \nil & the empty list\\
    {\tt (1)} & a list whose \functionref{car} is {\tt 1} and \functionref{cdr} is \nil\\
    {\tt (1 . 2)} & a pair whose \functionref{car} is {\tt 1} and \functionref{cdr} is
    {\tt 2}\\
    {\tt (1 2)} & a list whose \functionref{car} is {\tt 1} and \functionref{cdr} is {\tt
        (2)}
\end{tabular}

\derivedclass{cons}{list}
\index{general}{level-0 classes!\theclass{pair}}

The class of all instances of \classref{cons}.  An instance of the class
\classref{cons}\ (also known informally as a {\em dotted pair\/} or a {\em
    pair\/}) is a 2-tuple, whose slots are called, for historical reasons,
\functionref{car} and \functionref{cdr}.  Pairs are created by the function
\functionref{cons} and the slots are accessed by the functions \functionref{car}
and \functionref{cdr}.  The major use of pairs is in the construction of
(proper) lists.  A (proper) list\index{general}{list} is defined as either the
empty list (denoted by \nil) or a pair whose \functionref{cdr} is a proper list.
An improper list is one containing a \functionref{cdr} which is not a list (see
Table~\ref{pair-syntax}).

It is an error to apply \functionref{car} or \functionref{cdr} or their
\functionref{setter} functions to anything other than a pair.  The empty list is
not a pair and {\tt (car ())} or {\tt (cdr ())} is an error.

\function{consp}
%
\begin{arguments}
    \item[object] Object to examine.
\end{arguments}
%
\result%
Returns {\em object\/} if it is a pair, otherwise \nil.

\function{atom}
%
\begin{arguments}
    \item[object] Object to examine.
\end{arguments}
%
\result%
Returns {\em object\/} if it is not a pair, otherwise \nil.

\function{cons}
%
\begin{arguments}
    \item[object$_1$] An object.  pair.
    \item[object$_2$] An object.  pair.
\end{arguments}
%
\result%
Allocates a new pair whose slots are initialized with {\em object$_1$} in the
\functionref{car} and {\em object$_2$} in the \functionref{cdr}.

\function{car}
%
\begin{arguments}
    \item[pair] A pair.
\end{arguments}
%
\result%
Given a pair, such as the result of {\tt (cons {\em object$_1$} {\em
        object$_2$})}, then the function \functionref{car} returns {\em
    object$_1$}.

\function{cdr}
%
\begin{arguments}
    \item[pair] A pair.
\end{arguments}
%
\result%
Given a pair, such as the result of {\tt (cons {\em object$_1$} {\em
        object$_2$})}, then the function \functionref{cdr} returns {\em
    object$_2$}.

\setter{car}
%
\begin{arguments}
    \item[pair] A pair.
    \item[object] An object.
\end{arguments}
%
\result%
Given a pair, such as the result of {\tt (cons {\em object$_1$} {\em
        object$_2$})}, then the function {\tt (setter car)} replaces {\em
    object$_1$} with {\em object}.  The result is {\em object}.

\setter{cdr}
%
\begin{arguments}
    \item[pair] A pair.
    \item[object] An object.
\end{arguments}
%
\result%
Given a pair, such as the result of {\tt (cons {\em object$_1$} {\em
        object$_2$})}, then the function {\tt (setter cdr)} replaces {\em
    object$_2$} with {\em object}.  The result is {\em object}.
%
\remarks%
Note that if {\em object\/} is not a proper list, then the use of {\tt
(setter cdr)} might change {\em pair\/} into an improper list.

\method{equal}
%
\begin{specargs}
    \item[pair$_1$] A pair.
    \item[pair$_2$] A pair.
\end{specargs}
%
\result%
The result is the conjunction of the pairwise application of \genericref{equal}
to the \functionref{car} fields and the \functionref{cdr} fields of the
arguments.

\method{deep-copy}
%
\begin{specargs}
    \item[pair, \classref{cons}] A pair.
\end{specargs}
%
\result%
Constructs and returns a copy of the list starting at {\em pair\/} copying both
the \functionref{car} and the \functionref{cdr} slots of the list.  The list can
be proper or improper.  Treatment of the objects stored in the \functionref{car}
slot (and the \functionref{cdr} slot in the case of the final pair of an
improper list) is determined by the \genericref{deep-copy} method for the class
of the object.

\method{shallow-copy}
%
\begin{specargs}
    \item[pair, \classref{cons}] A pair.
\end{specargs}
%
\result%
Constructs and returns a copy of the list starting at {\em pair\/} but copying
only the \functionref{cdr} slots of the list, terminating when a pair is
encountered whose \functionref{cdr} slot is not a pair.  The list beginning at
{\em pair\/} can be proper or improper.

\function{list}
%
\begin{arguments}
    \item[{\optional{object$_1$ ... object$_n$}}] A sequence of objects.
\end{arguments}
%
\result%
Allocates a set of pairs each of which has been initialized with {\em
object$_i$} in the \functionref{car} field and the pair whose \functionref{car} field
contains {\em object$_{i+1}$} in the \functionref{cdr} field.  Returns the pair
whose \functionref{car} field contains {\em object$_1$}.
%
\examples
\begin{tabular}{lcl}
    \verb|(list)| &\Ra& \verb|()|\\
    \verb|(list 1 2 3)| &\Ra& \verb|(1 2 3)|
\end{tabular}

\method{generic-prin}
%
\begin{specargs}
    \item[pair, \classref{cons}] The pair to be output on {\em stream}.
    \item[stream, \classref{stream}] The stream on which the representation is
    to be output.
\end{specargs}
%
\result%
The pair supplied as the first argument.
%
\remarks%
Output the external representation of {\em pair\/} on {\em stream\/} as
described at the beginning of this section.  Uses \genericref{generic-prin} to
produce the external representation of the contents of the \functionref{car} and
\functionref{cdr} slots of {\em pair}.

\method{generic-write}
%
\begin{specargs}
    \item[pair, \classref{cons}] The pair to be output on {\em stream}.
    \item[stream, \classref{stream}] The stream on which the representation is
    to be output.
\end{specargs}
%
\result%
The pair supplied as the first argument.
%
\remarks%
Output the external representation of {\em pair\/} on {\em stream\/} as
described at the beginning of this section.  Uses \genericref{generic-write} to
produce the external representation of the contents of the \functionref{car} and
\functionref{cdr} slots of {\em pair}.

\end{optDefinition}
