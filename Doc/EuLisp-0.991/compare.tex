\sclause{Comparison}
\label{compare}
\index{general}{level-0 modules!compare}
\index{general}{compare!module}
%
\begin{optDefinition}
%
The defined name of this module is {\tt compare}.  There are three binary
functions for comparing objects for equality, \functionref{eq},
\functionref{eql}, \genericref{binary=} and the n-ary \functionref{=} which uses
\genericref{binary=}.  The three binary functions are related in the following
way:
%
\begin{center}
\begin{tabular}{rcccl}
    {\tt (eq {\em a} {\em b})} & $\Rightarrow$ & {\tt (eql {\em a} {\em
            b})} & $\Rightarrow$ & {\tt (binary= {\em a} {\em b})}\\
    {\tt (eq {\em a} {\em b})} & $\not\Leftarrow$ & {\tt (eql {\em a} {\em
            b})} & $\not\Leftarrow$ & {\tt (binary= {\em a} {\em b})}\\
\end{tabular}
\end{center}
%
There are four n-ary function for comparing objects by order, \functionref{<}
and \functionref{>} which are implemented by the generic function
\genericref{binary<}, \functionref{<=} and \functionref{>=} which are
implemented by the generic functions \genericref{binary<} and
\genericref{binary=}.  There is also one binary function for comparing objects
for inequality, \functionref{!=}.  A summary of the comparison functions and the
classes for which they have defined behaviour is given below:

\framebox[\linewidth]
{
\begin{tabular*}{\linewidth}{ll}
    \functionref{eq}: & \classref{object}$\times$\classref{object} \\
    \hline
    \functionref{eql}: & \classref{object}$\times$\classref{object}
    \Ra \functionref{eq} \\
    & \classref{character}$\times$\classref{character} \\
    & \classref{fixed-precision-integer}$\times$ \\
    & \classref{fixed-precision-integer} \\
    & \classref{double-float}$\times$\classref{double-float}
    \Ra \genericref{binary=}\\
    \hline
    \genericref{binary=}: & \classref{object}$\times$\classref{object} \\
    & \classref{character}$\times$\classref{character} \\
    & \classref{null}$\times$\classref{null} \\
    & \classref{number}$\times$\classref{number}
    \Ra \functionref{eql}\\
    & \classref{fixed-precision-integer}$\times$ \\
    & \classref{fixed-precision-integer} \\
    & \classref{double-float}$\times$\classref{double-float} \\
    & \classref{double-float}$\times$\classref{fixed-precision-integer} \\
    & \classref{fixed-precision-integer}$\times$\classref{double-float} \\
    & \classref{cons}$\times$\classref{cons} \\
    & \classref{string}$\times$\classref{string} \\
    & \classref{vector}$\times$\classref{vector} \\
    \hline
    \genericref{binary<}: & \classref{character}$\times$\classref{character} \\
    & \classref{symbol}$\times$\classref{symbol} \\
    & \classref{fixed-precision-integer}$\times$ \\
    & \classref{fixed-precision-integer} \\
    & \classref{double-float}$\times$\classref{double-float} \\
    & \classref{string}$\times$\classref{string} \\
    \hline
    \functionref{=}: & \classref{object}$\times$\classref{object}
    \Ra \genericref{binary=} \\
    \functionref{!=}: & \classref{object}$\times$\classref{object} \\
    \functionref{<}: & \classref{object}$\times$\classref{object}
    \Ra \genericref{binary<} \\
    \functionref{>}: & \classref{object}$\times$\classref{object} \\
    \functionref{<=}: & \classref{object}$\times$\classref{object} \\
    \functionref{>=}: & \classref{object}$\times$\classref{object}
\end{tabular*}
}

\function{eq}
%
\begin{arguments}
    \item[object$_1$] An object.
    \item[object$_2$] An object.
\end{arguments}
%
\result%
Compares {\em object$_1$} and {\em object$_2$} and returns \true{} if they are
the {\em same\/} object, otherwise \nil{}.  {\em Same\/} in this context means
``identifies the same memory location''.
%
\remarks%
In the case of numbers and characters the behaviour of \functionref{eq} might
differ between processors because of implementation choices about internal
representations.  Therefore, \functionref{eq} might return \true{} or \nil{} for
numbers which are \functionref{=} and similarly for characters which are
\functionref{eql}, depending on the implementation
\ttsubindex{eq}{implementation-defined behaviour}
\index{general}{implementation-defined!behaviour of \functionref{eq}}.
%
\examples%
\begin{tabular}{lcl}
    \verb+(eq () nil)+ & \Ra & \verb+t+\\
    \verb+(eq '() 'nil)+ & \Ra & \verb+t+\\
    \verb+(eq 'a 'a)+ & \Ra & \verb+t+\\
    \verb+(eq 'a 'b)+ & \Ra & \verb+()+\\
    \verb+(eq #\a #\a)+ & \Ra & \verb+t+ or \verb+()+\\
    \verb+(eq 3 3)+ & \Ra & \verb+t+ or \verb+()+\\
    \verb+(eq 3 3.0)+ & \Ra & \verb+()+\\
    \verb+(eq 3.0 3.0)+ & \Ra & \verb+t+ or \verb+()+\\
    \verb+(eq (cons 'a 'b) (cons 'a 'c))+ & \Ra & \verb+()+\\
    \verb+(eq (cons 'a 'b) (cons 'a 'b))+ & \Ra & \verb+()+\\
    \verb+(eq '(a . b) '(a . b))+ & \Ra & \verb+t+ or \verb+()+\\
    \verb+(let ((x (cons 'a 'b))) (eq x x))+ & \Ra & \verb+t+\\
    \verb+(let ((x '(a . b))) (eq x x))+ & \Ra & \verb+t+\\
    \verb+(eq "string" "string")+ & \Ra & \verb+t+ or \verb+()+\\
    \verb+(eq #('a 'b) #('a 'b))+ & \Ra & \verb+t+ or \verb+()+\\
    \verb+(let ((x #('a 'b))) (eq x x))+ & \Ra & \verb+t+\\
\end{tabular}

\function{eql}
%
\begin{arguments}
    \item[object$_1$] An object.
    \item[object$_2$] An object.
\end{arguments}
%
\result%
If the class of {\em object$_1$} and of {\em object$_2$} is the same and is a
subclass of \classref{character} or \classref{number}, the result is that of
comparing them under \methodref{binary=}.  Otherwise the result is that of
comparing them under \functionref{eq}.
%
\examples%
Given the same set of examples as for \functionref{eq}, the same result is
obtained except in the following cases:

\begin{tabular}{lcl}
    \verb+(eql #\a #\a)+ & \Ra & \verb+t+\\
    \verb+(eql 3 3)+ & \Ra & \verb+t+\\
    \verb+(eql 3.0 3.0)+ & \Ra & \verb+t+\\
\end{tabular}

\generic{binary=}
%
\begin{arguments}
    \item[object$_1$, \classref{object}] An object.
    \item[object$_2$, \classref{object}] An object.
\end{arguments}
%
\result%
Returns \true{} or \nil{} according to the method for the class(es) of {\em
    object$_1$} and {\em object$_2$}. It is an error if either or both of the
arguments is self-referential.
%
\seealso%
Class specific methods on \genericref{binary=} are defined for
\classref{character}, \classref{list}, \classref{number} (with specialisations
for \classref{fixed-precision-integer} and \classref{double-float}),
\classref{string}, \classref{vectors}.  All other cases are handled by the
default method defined for \classref{object}:

\method{binary=}
%
\begin{specargs}
    \item[object$_1$, \classref{object}] An object.
    \item[object$_2$, \classref{object}] An object.
\end{specargs}
%
\result%
The result is as if \functionref{eql} had been called with the arguments
supplied.

\generic{binary<}
%
\begin{genericargs}
    \item[object$_1$, \classref{object}] An object.
    \item[object$_2$, \classref{object}] An object.
\end{genericargs}
%
\result%
The first argument if it is less than the second, according to the method for
the class of the arguments, otherwise \nil{}.
%
\seealso%
Class specific methods on \genericref{binary<} are defined for
\classref{character}, \classref{string}, \classref{fixed-precision-integer} and
\classref{double-float}.

\function{=}
%
\begin{arguments}
    \item[{number$_1$ \ldots}] A non-empty sequence of numbers.
\end{arguments}
%
\result%
Given one argument the result is \true{}.  Given more than one argument the
result is determined by \genericref{binary=}, returning \true{} if all the
arguments are the same, otherwise \nil{}.

\function{!=}
%
\begin{arguments}
    \item[{number$_1$ \ldots}] A non-empty sequence of numbers.
\end{arguments}
%
\result%
Given one argument the result is \nil{}.  Given more than one argument the result
is determined by \genericref{binary=}, returning \nil{} if all the arguments are
the same, otherwise \true{}.

\function{<}
%
\begin{arguments}
    \item[object$_1$ \ldots] A non-empty sequence of objects.
\end{arguments}
%
\result%
Given one argument the result is \true{}.  Given more than one argument the
result is \true{} if the sequence of objects {\em object$_1$} up to {\em
    object$_n$} is strictly increasing according to the generic function
\genericref{binary<}.  Otherwise, the result is \nil{}.

\function{>}
%
\begin{arguments}
    \item[object$_1$ \ldots] A non-empty sequence of objects.
\end{arguments}
%
\result%
Given one argument the result is \true{}.  Given more than one argument the
result is \true{} if the sequence of objects {\em object$_1$} up to {\em
    object$_n$} is strictly decreasing according to the generic function
\genericref{binary<} applied to the arguments in reverse order.  Otherwise, the
result is \nil{}.

\function{<=}
%
\begin{arguments}
    \item[object$_1$ \ldots] A non-empty sequence of objects.
\end{arguments}
%
\result%
Given one argument the result is \true{}.  Given more than one argument the
result is \true{} if the sequence of objects {\em object$_1$} up to {\em
    object$_n$} is strictly increasing according to the generic function
\genericref{binary<} and \genericref{binary=}.  Otherwise, the result is \nil{}.

\function{>=}
%
\begin{arguments}
    \item[object$_1$ \ldots] A non-empty sequence of objects.
\end{arguments}
%
\result%
Given one argument the result is \true{}.  Given more than one argument the
result is \true{} if the sequence of objects {\em object$_1$} up to {\em
    object$_n$} is strictly decreasing according to the generic function
\genericref{binary<} and \genericref{binary=} applied to the arguments in
reverse order.  Otherwise, the result is \nil{}.

\function{max}
%
\begin{arguments}
    \item[object$_1$ \ldots] A non-empty sequence of objects.
\end{arguments}
%
\result%
The maximal element of the sequence of objects {\em object$_1$} up to {\em
    object$_n$} using the generic function \genericref{binary<}.  Zero arguments
is an error.  One argument returns {\em object$_1$}.

\function{min}
%
\begin{arguments}
    \item[object$_1$ \ldots] A non-empty sequence of objects.
\end{arguments}
%
\result%
The minimal element of the sequence of objects {\em object$_1$} up to {\em
    object$_n$} using the generic function \genericref{binary<}.  Zero arguments
is an error.  One argument returns {\em object$_1$}.
%
\end{optDefinition}
