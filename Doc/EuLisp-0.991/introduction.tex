\begin{introduction}
\label{sec:intro}
\begin{optPrivate}
The behaviour of macros is never explained, especially with respect to how many
times they might be expanded.

The hierarchy of types (classes) is not specified anywhere.  The coercion
ladder, if taken as part of the hierarchy, is upside down in some places (if the
class systems is taken as prototypical of types).

Nothing about the behaviour of \telos\ is specified.  How is the CPL calculated?
When are methods invoked?  What is method combination?  (Note: it seems someone
else wrote this part.  It seems much less smooth than the rest).

Threads are hardly discussed.  How can tail-recursion be specified
linguistically (RPG).

The next paragraph is somewhat awkward.  [jwd]

Why are modules not first-class? (RPG).  Abstraction to optimization is a poor
link (RPG)---action: remove (JAP).  The ...recursive behaviour... definition is
incorrect (RPG).  Implicit here is that {\tt symbol-value} and {\tt dynamic} are
disjoint (RPG).

JWD: We ought to consider whether we really want to require tail recursion
optimization.  It is somewhat difficult to define precisely what the requirement
means (see mail on the Scheme Standard list), and it makes it harder to use
certain implementation strategies (eg, compile to C).  On the other hand, we
don't have any other means for expressing iteration and I suspect many of us are
inclined to follow Scheme.  A Rationale entry is required, at least.

JWD: The Rationale should explain why dynamic variables are not by-module.  It
is possible to implement dynamic variables so that different vars with the same
name can exist in different modules.  We would have to have some way to refer to
(or import) variables in another module.

Rewrote (once again) the description of the binding environments and gave them
new names.  This removes the false distinction between lexical and module
bindings that people complained about at Dec '90 meeting.  Any further
complaints about this must be accompanied by revised wording.
\end{optPrivate}

\begin{optDefinition}
\eulisp\ is a dialect of Lisp and as such owes much to the great body of work
that has been done on language design in the name of Lisp over the last thirty
years.  The distinguishing features of \eulisp\ are (i) the integration of the
classical Lisp type system and the object system into a single class hierarchy
(ii) the complementary abstraction facilities provided by the class and the
module mechanism (iii) support for concurrent execution.

\noindent
Here is a brief summary of the main features of the language.
\begin{itemize}

    \item Classes\index{general}{class} are first-class objects.  The class
    structure integrates the primitive classes\index{general}{class!primitive}
    describing fundamental datatypes, the predefined classes and user-defined
    classes.

    \item Modules\index{general}{module} together with classes are the building
    blocks of both the \eulisp\ language and of applications written in \eulisp.
    The module system exists to limit access to objects by name.  That is,
    modules allow for hidden definitions.  Each module defines a fresh, empty,
    lexical environment\index{general}{module!environments}.

    \item Multiple control threads can be created in \eulisp\ and the
    concurrency model has been designed to allow consistency across a wide range
    of architectures.  Access to shared data can be controlled via locks
    (semaphores). Event-based programming is supported through a generic waiting
    function.

    \item Both functions and continuations are first-class in \eulisp, but the
    latter are not as general as in Scheme because they can only be used in the
    dynamic extent of their creation.  That implies they can only be used once.

    \item A condition mechanism which is fully integrated with both classes and
    threads, allows for the definition of generic handlers and supports both
    propagation of conditions and continuable handling.

    \item Dynamically scoped bindings\index{general}{binding!dynamically scoped}
    can be created in \eulisp, but their use is restricted, as in Scheme.
    \eulisp\ enforces a strong distinction between lexical bindings and dynamic
    bindings by requiring the manipulation of the latter via special forms.

\end{itemize}
\eulisp\ does not claim any particular Lisp dialect as its closest relative,
although parts of it were influenced by features found in
\cl\index{general}{cl@\cl}, \interlisp\index{general}{interlisp@\interlisp},
\lelisp\index{general}{lelisp@\lelisp},
\lisp/VM\index{general}{lispVM@\lisp/VM}, Scheme, and
T\index{general}{T}. \eulisp\ both introduces new ideas and takes from these
Lisps.  It also extends or simplifies their ideas as seen fit.  But this is not
the place for a detailed language comparison.  That can be drawn from the rest
of this text.

\eulisp\ breaks with \lisp\ tradition in describing all its types (in fact,
classes) in terms of an object system.  This is called The \eulisp\ Object
System, or \telos. \telos\ incorporates elements of the Common Lisp Object
System (CLOS)\index{general}{CLOS}~\bref{clos},
ObjVLisp\index{general}{ObjVLisp}~\bref{objvlisp},
Oaklisp\index{general}{Oaklisp}~\bref{oaklisp},
MicroCeyx\index{general}{MicroCeyx}~\bref{lelisp-manual}, and
MCS\index{general}{MCS}~\bref{mcs}.
\end{optDefinition}

\end{introduction}
