\sclause{Classes and Objects}
%
\begin{optDefinition}
\defform{defclass}
\label{defclass-1}
%
\Syntax
\label{defclass-syntax}
\savesyntax\defclassSyntax\vbox{\small\syntax
defclass form
   = '(', 'defclass',
     class name, (* \[\S\ref{defclass}\] *)
     superclass list,
     defclass slot description list,
     {defclass class option}, ')';
superclass list
   = '(', {superclass name}, ')'; (* \[\S\ref{defclass}\] *)
defclass slot description list
   = '(', {defclass slot description}, ')';
defclass slot option
   = 'initform', level 1 expression
   | identifier, level 1 expression
   | defclass slot option; (* \[\S\ref{defclass}\] *)
defclass class option
   = 'class', class name
   | identifier, level 1 expression
   | defclass class option; (* \[\S\ref{defclass}\] *)
\endsyntax}
\syntaxtable{defclass}{\defclassSyntax}

\begin{arguments}
    \item[class-name] A symbol naming a binding to be initialised with the new
    class.

    \item[\group{superclass\/$^*$}] A list of symbols naming bindings of classes
    to be used as the superclasses of the new class (multiple
    inheritance\index{general}{inheritance!multiple}\index{general}{multiple
        inheritance}).  This is different from \defformref{defclass} at level-0,
    where only one superclass may be specified.

    \item[\group{slot\/$^*$}] A list of slot specifications (see
    below), comprising either a {\em slot-name} or a list of a {\em slot-name}
    followed by some {\em slot-option}s.  One of the class options (see below)
    allows the specification of the class of the slot description.

    \item[class-option\/$^*$] A sequence of keys and values (see below).  One of
    the class options (\classref{class}) allows the specification of the class
    of the class being defined.
\end{arguments}
%
\remarks%
This defining form defines a new class.  The resulting class will be
bound to {\em class-name}.  The second argument is a list of
superclasses.  If this list is empty, the superclass is \classref{object}.
The third argument is a list of slots.  The remaining
arguments are class options.  All the slot options and class options
are exactly the same way as for \defformref{defclass} (\ref{defclass}).

\noindent
The {\em slot-option}\/s are interpreted as follows:
\begin{options}
    \item[initarg, symbol]%
    The value of this option is a identifier naming a symbol, which is the name
    of an argument to be supplied in the {\em init-option\/}s of a call to
    \functionref{make} on the new class.  The value of this argument in the call
    to \functionref{make} is the initial value of the slot.  This option must
    only be specified once for a particular slot.  The same initarg name may be
    used for several slots, in which case they will share the same initial value
    if the initarg is given to \functionref{make}.

    \item[initform, form]%
    The value of this option is a form, which is evaluated as the default value
    of the slot, to be used if no initarg is defined for the slot or given to a
    call to \functionref{make}.  The form is evaluated in the lexical
    environment of the call to \defformref{defclass} and the dynamic environment
    of the call to \functionref{make}.  The form is evaluated each time
    \functionref{make} is called and the default value is called for.  The order
    of evaluation of the initforms in all the slots is determined by
    \genericref{initialize}.  This option must only be specified once for a
    particular slot.

    \item[reader, symbol]%
    The value is the identifier of the variable to which the reader function
    will be bound.  The reader function is a means to access the slot.  The
    reader function is a function of one argument, which should be an instance
    of the new class.  No writer function is automatically bound with this
    option.  This option can be specified more than once for a slot, creating
    several readers.  It is an error to specify the same reader, writer, or
    accessor name for two different slots.

    \item[writer, symbol]%
    The value is the identifier of the variable to which the writer function
    will be bound.  The writer function is a means to change the slot value.
    The creation of the writer is analogous to that of the reader function.
    This option can be specified more than once for a slot.  It is an error to
    specify the same reader, writer, or accessor name for two different slots.

    \item[accessor, symbol]%
    The value is the identifier of the variable to which the reader function
    will be bound. In addition, the use of this {\em slot-option} causes that
    the writer function is associated to the reader {\em via} the
    \functionref{setter} mechanism. This option can be specified more than once
    for a slot. It is an error to specify the same reader, writer, or accessor
    name for two different slots.

    % \item[slot-class, class]%
    % The corresponding value is a subclass of \classref{slot}.An
    % implementation conforming at level-1
    % provides the slot description class \classref{local-slot}, for
    % slots particular to instances. Extension modules can define \classref{
    %     shared-slot}, for slots whose values are shared by all the
    % instances of the class. New slot description classes can be defined and
    % used here. This option can only be specified once for a particular
    % slot. Within a class different slots can have different slot description
    % classes.

    \item[{\em identifier}, expression]%
    The symbol named by {\em identifier} and the value of {\em expression} are
    passed to \functionref{make} of the slot description class along with other
    slot options.  The values are evaluated in the lexical and dynamic
    environment of the \defformref{defclass}.  For the language defined slot
    description classes, no slot initargs are defined which are not specified by
    particular \defformref{defclass} slot options.
\end{options}
%
The {\em class-option}\/s are interpreted as follows:
\begin{options}
    \item[initargs, list]%
    The value of this option is a list of identifiers naming symbols, which
    extend the inherited names of arguments to be supplied in the {\em
        init-option\/}s of a call to \functionref{make} on the new class.
    Initargs are inherited by union. The values of all legal arguments in the
    call to \functionref{make} are the initial values of corresponding slots if
    they name a slot or are ignored by the default \genericref{initialize}
    method, otherwise. This option must only be specified once for a
    class.  %\functionref{make}.

    \item[constructor, constructor-spec]%
    Creates a constructor function for the new class.  The constructor
    specification gives the name to which the constructor function will be
    bound, followed by a sequence of legal initargs for the class.  The new
    function creates an instance of the class and fills in the slots according
    to the match between the specified initargs and the given arguments to the
    constructor function.  This option may be specified more than once.

    \item[predicate, symbol]%
    Creates a predicate function for the new class.  The predicate specification
    gives the name to which the predicate function will be bound.  This option
    may be specified any number of times for a class.

    \item[class, class]%
    The value of this option is the class of the new class.  By default, this is
    \classref{class}.  This option must only be specified once for the new
    class.

    \item[{\em identifier}, expression]%
    The symbol named by {\em identifier} and the value of {\em expression} are
    passed to \functionref{make} on the class of the new class.  This list is
    appended to the end of the list that \defformref{defclass} constructs.  The
    values are evaluated in the lexical and dynamic environment of the
    \defformref{defclass}.  This option is used for metaclasses which need extra
    information not provided by the standard options.
\end{options}
%
\end{optDefinition}

\sclause{Generic Functions}

\begin{optDefinition}
\macro{generic-lambda}
\label{generic-lambda-1}
%
\Syntax
\savesyntax\genericLambdaOneSyntax\vbox{\small\syntax
generic lambda form
   = '(', 'generic-lambda', gf lambda list,
     {level 1 init option}, ')';
level 1 init option
   = 'class', class name
   | 'method-class', class name
   | 'method' level 1 method description
   | identifier, level 1 expression;
   | level 0 init option;
level 1 method description
   = '(', {method init option},
     specialized lambda list,  (* \[\ref{defgeneric-0}\] *)
     {form}, ')';
method init option
   = 'class', class name
   | identifier level 1 expression
\endsyntax}
\syntaxtable{generic-lambda}{\genericLambdaOneSyntax}
%
\begin{arguments}
    \item[gf-lambda-list] As level-0.  See section~\ref{defgeneric-0}.

    \item[level-1-init-option\/$^*$] Format as level-0, but with additional
    options, which are defined below.
\end{arguments}
%
\result%
A generic function.
%
\remarks%
The syntax of \macroref{generic-lambda} is an extension of the level-0
syntax allowing additional init-options.  These allow the
specification of the class of the new generic function, which defaults
to \classref{generic-function}, the class of all methods, which defaults
to \classref{method}, and non-standard options. The latter are evaluated
in the lexical and dynamic environment of \macroref{generic-lambda} and
passed to \functionref{make} of the generic function as additional
initialization arguments.  The additional {\em init-option\/}s over
level-0 are interpreted as follows:
%
\begin{options}

    \item[class, gf-class] The class of the new generic function.  This must be
    a subclass of \classref{generic-function}.  The default is
    \classref{generic-function}.

    \item[method-class, method-class] The class of all methods to be defined on
    this generic function.  All methods of a generic function must be instances
    of this class.  The {\em method-class} must be a subclass of
    \classref{method} and defaults to \classref{method}.

    \item[{\em identifier}, expression] The symbol named by {\em identifier} and
    the value of {\em expression} are passed to \functionref{make} as initargs.
    % and each new method.
    The values are evaluated in the lexical and dynamic environment of the
    \defformref{defgeneric}.  This option is used for classes which need extra
    information not provided by the standard options.
\end{options}
%
In addition, method init options can be specified for the individual
methods on a generic function.  These are interpreted as follows:
%
\begin{options}
    \item[class, method-class]%
    The class of the method to be defined. The method class must be a subclass
    of \classref{method} and is, by default, \classref{method}. The value is
    passed to \functionref{make} as the first argument. The symbol and the value
    are not passed as initargs to \functionref{make}.

    \item[{\em identifier}, expression]%
    The symbol named by {\em identifier} and the value of {\em expression} are
    passed to \functionref{make} creating a new method as initargs.  The values
    are evaluated in the lexical and dynamic environment of the
    \macroref{generic-lambda}.  This option is used for classes which need extra
    information not provided by the standard options.
\end{options}
%
\examples
 In the following example an anonymous version of {\tt gf-1} (see
\defformref{defgeneric}) is defined.  In all other respects the resulting object
is the same as {\tt gf-1}.
%
{\small\syntax
(generic-lambda (arg1 (arg2 <class-a>))

  class <another-gf-class>
  class-key-a class-value-a
  class-key-b class-value-b

  method-class <another-method-class-a>

  method (class <another-method-class-b>
          method-class-b-key-a method-class-b-value-a
          ((m1-arg1 <class-b>) (m1-arg2 <class-c>))
          ...)
  method (method-class-a-key-a method-class-a-value-a
          ((m2-arg1 <class-d>) (m2-arg2 <class-e>))
          ...)
  method (class <another-method-class-c>
          method-class-c-key-a method-class-c-value-a
          ((m3-arg1 <class-f>) (m3-arg2 <class-g>))
          ...)
)
\endsyntax}
%
\seealso%
\defformref{defgeneric}.

\defform{defgeneric}
\label{defgeneric-1}
%
\Syntax
\savesyntax\defgenericOneSyntax\vbox{\small\syntax
defgeneric form
   = '(', 'defgeneric',
     gf name, (* \[\ref{defgeneric-0}\] *)
     gf lambda list, (* \[\ref{defgeneric-0}\] *)
     {level 1 init option}, ')';
\endsyntax}
\syntaxtable{defgeneric}{\defgenericOneSyntax}
%
\begin{arguments}
    \item[gf name] As level-0.  See section~\ref{defgeneric-0}.

    \item[gf lambda list] As level-0.  See section~\ref{defgeneric-0}.

    \item[init option\/$^*$] As for \macroref{generic-lambda}, defined above.
    below.
\end{arguments}
%
\remarks%
This defining form defines a new generic function.  The resulting generic
function will be bound to {\em gf-name}.  The second argument is the formal
parameter list.  An error is signalled (condition:
\conditionref{non-congruent-lambda-lists}
\indexcondition{non-congruent-lambda-lists}) if any of the methods defined on
this generic function do not have lambda lists congruent to that of the generic
function.  This applies both to methods defined at the same time as the generic
function and to any methods added subsequently by \macroref{defmethod} or
\genericref{add-method}.  An {\em init-option} is a identifier followed by its
initial value.  The syntax of \defformref{defgeneric} is an extension of the
level-0 syntax.  The rewrite rules for the \defformref{defgeneric} form are
identical to those given in section~\ref{defgeneric-rewrite-rules} except that
{\em level 1 init option} replaces {\em level 0 init option}.
%
\examples
In the following example of the use of \defformref{defgeneric} a generic
function named {\tt gf-1} is defined.  The differences between this
function and {\tt gf-0} (see~\ref{defgeneric-0}) are
\begin{enumerate}
    \item The class of the generic function is specified
    (\theclass{another-gf-class}) along with some init-options related to the
    creation of an instance of that class.

    \item The default class of the methods to be attached to the generic
    function is specified (\theclass{another-method-class-a}) along with an
    init-option related to the creation of an instance of that class.

    \item In addition, some of the methods to be attached are of a different
    method class (\theclass{another-method-class-b} and
    \theclass{another-method-class-c}) also with method specific init-options.
    These method classes are subclasses of \theclass{another-method-class-a}.
\end{enumerate}
%
\begin{optPrivate}
{\small\begin{verbatim}
(defgeneric gf-1 (arg1 (arg2 <class-a>))
  class <another-gf-class>
  class-key-a class-value-a
  class-key-b class-value-b

  method-class <another-method-class>
  method-class-key-a method-class-value-a

  method (((m1-arg1 <class-b>) (m1-arg2 <class-c>))
          ...)
  method (((m2-arg1 <class-d>) (m2-arg2 <class-e>))
          ...)
  method (((m3-arg1 <class-f>) (m3-arg2 <class-g>))
          ...)
)\end{verbatim}}
%
\end{optPrivate}
{\small\syntax
(defgeneric gf-1 (arg1 (arg2 <class-a>))

  class <another-gf-class>
  class-key-a class-value-a
  class-key-b class-value-b

  method-class <another-method-class-a>

  method (class <another-method-class-b>
          method-class-b-key-a method-class-b-value-a
          ((m1-arg1 <class-b>) (m1-arg2 <class-c>))
          ...)
  method (method-class-a-key-a method-class-a-value-a
          ((m2-arg1 <class-d>) (m2-arg2 <class-e>))
          ...)
  method (class <another-method-class-c>
          method-class-c-key-a method-class-c-value-a
          ((m3-arg1 <class-f>) (m3-arg2 <class-g>))
          ...)
)
\endsyntax}

\sclause{Methods}

\macro{method-lambda}
\Syntax
\savesyntax\methodLambdaOneSyntax\vbox{\small\syntax
method lambda form
   = '(', 'method-lambda',
     {method init option}, (* \[\S\ref{generic-lambda-1}\] *)
     specialized lambda list,
     {form}, ')';
\endsyntax}
\syntaxtable{method-lambda}{\methodLambdaOneSyntax}
%
\begin{arguments}
    \item[method init option] A quoted symbol followed by an expression.

    \item[specialized lambda list] As defined under \macroref{generic-lambda}.

    \item[form] An expression.
\end{arguments}
%
\result%
This syntax creates and returns an anonymous method with the given
lambda list and body.  This anonymous method can later be added to a generic
function with a congruent lambda list via the generic function
\genericref{add-method}.  Note that the lambda list can be specialized to
specify the method's domain.  The value of the special initarg \classref{class}
determines the class to instantiate; the rest of the initlist is passed to
\functionref{make} called with this class.  The default method class is
\classref{method}.
%
\remarks%
The additional {\em method-init-option\/}s includes \classref{class}, for
specifying the class of the method to be defined, and non-standard
options, which are evaluated in the lexical and dynamic environment of
\macroref{method-lambda} and passed to \genericref{initialize} of that method.

\macro{defmethod}
\Syntax
\savesyntax\defmethodOneSyntax\vbox{\small\syntax
defmethod form
   = '(', 'defmethod', gf name,
     {method init option}, (* \[\S\ref{generic-lambda-1}\] *)
     specialized lambda list,
     {form}, ')';
\endsyntax}
\syntaxtable{defmethod}{\defmethodOneSyntax}
%
\remarks%
The \macroref{defmethod} form of level-1 extends that of level-0 to accept
{\em method-init-option\/}s.  This allows for the specification of the
method class by means of the \classref{class} init option.  This class must
be a subclass of the method class of the host generic function. The
method class otherwise defaults to that of the host generic function.
In all other respects, the behaviour is as that defined in level-0.

\macro{method-function-lambda}
%
\begin{arguments}
    \item[lambda-list] A lambda list
    \item[form$^*$] A sequence of forms.
\end{arguments}
%
This macro creates and returns an anonymous method function with the given
lambda list and body. This anonymous method function can later be added to a
method using \setterref{method-function}, or as the {\tt function}
initialization value in a call of \functionref{make} on an instance of
\classref{method}.  A function of this type is also returned by the method
accessor \functionref{method-function}.  Only functions created using this macro
can be used as method functions.  Note that the lambda list must not be
specialized; a method's domain is stored in the method itself.

% hb: there are no setters specified yet!
\function{call-method}
%
\begin{arguments}
    \item[method] A method.
    \item[next-methods] A list of methods.
    \item[arg$^*$] A sequence of expressions.
\end{arguments}
%
This function calls the method {\em method} with arguments {\em args}.
The argument {\em next-methods} is a list of methods which are used as
the applicable method list for {\em args}; it is an error if this list
is different from the methods which would be produced by the method
lookup function of the generic function of {\em method}.  If {\em
method} is not attached to a generic function, its behavior is
unspecified.  The {\em next-methods} are used to determine the next
method to call when \keywordref{call-next-method} is called within {\em
method-fn}.

\function{apply-method}
%
\begin{arguments}
    \item[method] A method.
    \item[next-methods] A list of methods.
    \item[form$_1$ \ldots form$_{n-1}$] A sequence of expressions.
    \item[form$_n$] An expression.
\end{arguments}
%
This function is identical to \functionref{call-method} except that its last
argument is a list whose elements are the other arguments to pass to
the method's method function.  The difference is the same as that
between normal function application and \functionref{apply}.
%
\end{optDefinition}
