\sclause{Characters}
\label{character}
\index{general}{character}
\index{general}{level-0 modules!character}
\index{general}{character!module}
\begin{optDefinition}
%
The defined name of this module is {\tt character}.

\syntaxform{character}
%
Character literals\index{general}{literal!character} are denoted by the {\em
    extension\/} glyph, called {\em hash} (\verb+#+), followed by the {\em
    character-extension\/}\index{general}{character!character-extension glyph}
glyph, called {\em reverse solidus\/} (\verb+\+), followed by the name of the
character.  The syntax for the external representation of characters is defined
in syntax table~\ref{character-syntax}.  For most characters, their name is the
same as the glyph associated with the character, for example: the character
``a'' has the name ``a'' and has the external representation \verb+#\a+.
Certain characters in the group named {\em special\/} (see
table~\ref{character-set} and also syntax table~\ref{character-syntax}) form the
syntax category \scref{special-character-token} and are referred to using the
digrams defined in table~\ref{character-digrams}.
%
\begin{table}[h]
\label{character-digrams}
\caption{Character digrams}%
\begin{center}
\begin{tabular}{|ll|}\hline
    Operation & Digram \\
    \hline
    alert & \verb+\a+ \\
    backspace & \verb+\b+ \\
    delete & \verb+\d+ \\
    formfeed & \verb+\f+ \\
    linefeed & \verb+\l+ \\
    newline & \verb+\n+ \\
    return & \verb+\r+ \\
    tab & \verb+\t+ \\
    vertical tab & \verb+\v+ \\
    hex-insertion & \verb+\x+ \\
    string delimiter & \verb+\"+ \\
    string escape & \verb+\\+ \\
    \hline
\end{tabular}
\end{center}
\end{table}
%
Any character which does not have an external representation dealt with by cases
described so far is represented by the digram \verb+#\x+ (see
table~\ref{character-digrams}) followed four hexadecimal digits.  The value of
the hexadecimal number represents the position of the character in the current
character set.  Examples of such character literals are \verb+#\x0000+ and
\verb+#\xabcd+, which denote, respectively, the characters at position 0 and at
position 43981 in the character set current at the time of reading or writing.
The syntax for the external representation of characters is defined in syntax
table~\ref{character-syntax} below:
%
\Syntax
\label{character-syntax}
\defSyntax{character}{
\begin{syntax}
    \scdef{character}: \\
    \>  \scref{literal-character-token} \\
    \>  \scref{special-character-token} \\
    \>  \scref{numeric-character-token} \\
    \scdef{literal-character-token}: \\
    \>  \#\textbackslash{}\scref{letter} \\
    \>  \#\textbackslash{}\scref{decimal-digit} \\
    \>  \#\textbackslash{}\scref{other-character} \\
    \>  \#\textbackslash{}\scref{special-character} \\
    \scdef{special-character-token}: \\
    \>  \#\textbackslash{}\textbackslash{}a \\
    \>  \#\textbackslash{}\textbackslash{}b \\
    \>  \#\textbackslash{}\textbackslash{}d \\
    \>  \#\textbackslash{}\textbackslash{}f \\
    \>  \#\textbackslash{}\textbackslash{}l \\
    \>  \#\textbackslash{}\textbackslash{}n \\
    \>  \#\textbackslash{}\textbackslash{}r \\
    \>  \#\textbackslash{}\textbackslash{}t \\
    \>  \#\textbackslash{}\textbackslash{}v \\
    \>  \#\textbackslash{}\textbackslash{}" \\
    \>  \#\textbackslash{}\textbackslash{}\textbackslash{} \\
    \scdef{numeric-character-token}: \\
    \>  \#\textbackslash{}\textbackslash{}x
        \scref{hexadecimal-digit} \scref{hexadecimal-digit} \\
    \>\>  \scref{hexadecimal-digit} \scref{hexadecimal-digit}
\end{syntax}}%
\showSyntaxBox{character}

%
\begin{note}
    This text refers to the ``current character set'' but defines no means of
    selecting alternative character sets.  This is to allow for future
    extensions and implementation-defined extensions which support more than one
    character set.
\end{note}

\derivedclass{character}{object}
\index{general}{level-0 classes!\theclass{character}}
%
The class of all characters.

\function{character?}
%
\begin{arguments}
    \item[{object}] Object to examine.
\end{arguments}
%
\result%
Returns {\em object\/} if it is a character, otherwise \nil{}.

\method{binary=}{character}
%
\begin{specargs}
    \item[character$_1$, \classref{character}] A character.
    \item[character$_2$, \classref{character}] A character.
\end{specargs}
%
\result%
If {\em character$_1$\/} is the same character as {\em character$_2$\/} the
result is {\em character$_1$}, otherwise the result is \nil{}.

\method{binary<}{character}
%
\begin{specargs}
    \item[character$_1$, \classref{character}] A character.
    \item[character$_2$, \classref{character}] A character.
\end{specargs}
%
\result%
If both characters denote uppercase alphabetic or both denote lowercase
alphabetic, the result is defined by alphabetical order.  If both characters
denote a digit, the result is defined by numerical order.  In these three cases,
if the comparison is \true{}, the result is {\em character$_1$}, otherwise it is
\nil{}.  Any other comparison is an error and the result of such comparisons is
undefined.
%
\examples
\begin{tabular}{lcl}
    \verb+(binary< #\A #\Z)+ & \Ra & \verb+#\A+\\
    \verb+(binary< #\a #\z)+ & \Ra & \verb+#\a+\\
    \verb+(binary< #\0 #\9)+ & \Ra & \verb+#\0+\\
    \verb+(binary< #\A #\a)+ & \Ra & {\em undefined}\\
    \verb+(binary< #\A #\0)+ & \Ra & {\em undefined}\\
    \verb+(binary< #\a #\0)+ & \Ra & {\em undefined}\\
\end{tabular}
%
\seealso%
Method \methodref{binary<}{string} for \class{string}.

\generic{as-lowercase}
%
\begin{genericargs}
    \item[object, \classref{object}] An object to convert to lower case.
\end{genericargs}
%
\result%
An instance of the same class as {\em object\/} converted to lower case
according to the actions of the appropriate method for the class of {\em
    object}.
%
\seealso%
Another method \methodref{as-lowercase}{string} for \classref{string}.

\method{as-lowercase}{character}
%
\begin{specargs}
    \item[character, \classref{character}] A character.
\end{specargs}
%
\result%
If {\em character\/} denotes an upper case character, a character denoting its
lower case counterpart is returned.  Otherwise the result is the argument.

\generic{as-uppercase}
%
\begin{genericargs}
    \item[object, \classref{object}] An object to convert to upper case.
\end{genericargs}
%
\result%
An instance of the same class as {\em object\/} converted to upper case
according to the actions of the appropriate method for the class of {\em
    object}.
%
\seealso%
Another method is defined on \methodref{as-uppercase}{string} for
\classref{string}.

\method{as-uppercase}{character}
%
\begin{specargs}
    \item[character, \classref{character}] A character.
\end{specargs}
%
\result%
If {\em character\/} denotes an lower case character, a character denoting its
upper case counterpart is returned.  Otherwise the result is the argument.

\method{generic-prin}{character}
%
\begin{specargs}
    \item[character, \classref{character}] Character to be ouptut on {\em
        stream}.
    \item[stream, \classref{stream}] Stream on which {\em character\/} is to be
    ouptut.
\end{specargs}
%
\result%
The character {\em character}.
%
\remarks%
Output the interpretation of {\em character\/} on {\em stream}.

\method{generic-write}{character}
%
\begin{specargs}
    \item[character, \classref{character}] Character to be ouptut on {\em
        stream}.
    \item[stream, \classref{stream}] Stream on which {\em character\/} is to be
    ouptut.
\end{specargs}
%
\result%
The character {\em character}.
%
\remarks%
Output external representation of {\em character\/} on {\em stream\/} in the
format \verb+#\+{\em{}name\/} as described at the beginning of this section.
%
\end{optDefinition}
