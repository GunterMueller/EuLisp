\sclause{Characters}
\label{character}
\index{general}{character}
\index{general}{level-0 modules!character}
\index{general}{character!module}
\begin{optDefinition}
%
The defined name of this module is {\tt character}.

\syntaxform{character}
%
Character literals\index{general}{literal!character} are denoted by the {\em
    extension\/} glyph, called {\em hash} (\verb+#+), followed by the {\em
    character-extension\/}\index{general}{character!character-extension glyph}
glyph, called {\em reverse solidus\/} (\verb+\+), followed by the name of the
character.  The syntax for the external representation of characters is defined
in syntax table~\ref{character-syntax}.  For most characters, their name is the
same as the glyph associated with the character, for example: the character
``a'' has the name ``a'' and has the external representation \verb+#\a+.
Certain characters in the group named {\em special\/} (see
table~\ref{character-set} and also syntax table~\ref{character-syntax}) have
symbolic names, for example: the newline character has the name {\em newline\/}
and has the external representation \verb+#\newline+.  These special cases are
the characters in the production {\em special character token} in syntax
table~\ref{character-syntax}.
%
\Syntax
\label{character-syntax}
% a horrible hack
\newbox\characterSyntax
\begingroup

\def\'{\string\'}
\def\^{\string\^}
\def\alert{\string\alert}
\def\backspace{\string\backspace}
\def\delete{\string\delete}
\def\formfeed{\string\formfeed}
\def\linefeed{\string\linefeed}
\def\newline{\string\newline}
\def\return{\string\return}
\def\tab{\string\tab}
\def\space{\string\space}
\def\vertical{\string\vertical}
\def\x{\string\x}

% \global so it doesn't get reset to void when we leave this group...
\global\setbox\characterSyntax\vbox{
\syntax
character token
   = literal character token
   | special character token
   | control character token
   | numeric character token;
literal character token
   = '#\', letter
   | '#\', decimal digit
   | '#\', non-alphabetic;
control character token
   = '#\^' letter;
special character token
   = '#\alert'
   | '#\backspace'
   | '#\delete'
   | '#\formfeed'
   | '#\linefeed'
   | '#\newline'
   | '#\return'
   | '#\tab'
   | '#\space'
   | '#\vertical-tab';
numeric character token
   = '#\x', hex digit
   | '#\x', hex digit, hex digit
   | '#\x', hex digit, hex digit, hex digit
   | '#\x', hex digit, hex digit, hex digit,
     hex digit;
\endsyntax}
\endgroup
\syntaxtable{character}{\characterSyntax}

Any character which does not have a name, and thereby an external representation
dealt with by cases described so far is represented by \verb+#\x+ followed by up
to four hexadecimal digits.  The value of the hexadecimal number represents the
position of the character in the current character set.  Examples of such
character literals are \verb+#\x0+ and \verb+#\xabcd+, which denote,
respectively, the characters at position 0 and at position 43981 in the
character set current at the time of reading or writing.  The syntax for the
external representation of characters is defined in
syntax table~\ref{character-syntax}.
%
\begin{note}
    This text refers to the ``current character set'' but defines no means of
    selecting alternative character sets.  This is to allow for future
    extensions and implementation-defined extensions which support more than one
    character set.
\end{note}

\derivedclass{character}{object}
\index{general}{level-0 classes!\theclass{character}}
%
The class of all characters.

\function{characterp}
%
\begin{arguments}
    \item[{object}] Object to examine.
\end{arguments}
%
\result%
Returns {\em object\/} if it is a character, otherwise \nil{}.

\method{equal}
%
\begin{specargs}
    \item[character$_1$, \classref{character}] A character.
    \item[character$_2$, \classref{character}] A character.
\end{specargs}
%
\result%
If {\em character$_1$\/} is the same character as {\em character$_2$\/} the
result is {\em character$_1$}, otherwise the result is \nil{}.

\method{binary<}
%
\begin{specargs}
    \item[character$_1$, \classref{character}] A character.
    \item[character$_2$, \classref{character}] A character.
\end{specargs}
%
\result%
If both characters denote uppercase alphabetic or both denote lowercase
alphabetic, the result is defined by alphabetical order.  If both characters
denote a digit, the result is defined by numerical order.  In these three cases,
if the comparison is true, the result is {\em character$_1$}, otherwise it is
\nil{}.  Any other comparison is an error and the result of such comparisons is
undefined.
%
\examples
\begin{tabular}{lcl}
    \verb+(binary< #\A #\Z)+ & \Ra & \verb+#\A+\\
    \verb+(binary< #\a #\z)+ & \Ra & \verb+#\a+\\
    \verb+(binary< #\0 #\9)+ & \Ra & \verb+#\0+\\
    \verb+(binary #\A #\a)+ & \Ra & {\em undefined}\\
    \verb+(binary #\A #\0)+ & \Ra & {\em undefined}\\
    \verb+(binary #\a #\0)+ & \Ra & {\em undefined}\\
\end{tabular}
%
\seealso%
Method on \genericref{binary<} for \hyperref[string]{strings}.

%\method{binary=}
%
%\begin{specargs}
%\item[c$_1$, \classref{character}] A character.
%\item[c$_2$, \classref{character}] A character.
%\end{specargs}
%
%\result%
%If {\em c$_1$} and {\em c$_2$} denote the same character, the result
%is {\em c$_1$}, otherwise it is \nil{}.

\generic{as-lowercase}
%
\begin{genericargs}
    \item[object, \classref{object}] An object to convert to lower case.
\end{genericargs}
%
\result%
An instance of the same class as {\em object\/} converted to lower case
according to the actions of the appropriate method for the class of {\em
    object}.
%
\seealso%
Another method is defined on \genericref{as-lowercase} for
\hyperref[string]{strings}.

\method{as-lowercase}
%
\begin{specargs}
    \item[character, \classref{character}] A character.
\end{specargs}
%
\result%
If {\em character\/} denotes an upper case character, a character denoting its
lower case counterpart is returned.  Otherwise the result is the argument.

\generic{as-uppercase}
%
\begin{genericargs}
    \item[object, \classref{object}] An object to convert to upper case.
\end{genericargs}
%
\result%
An instance of the same class as {\em object\/} converted to upper case
according to the actions of the appropriate method for the class of {\em
    object}.
%
\seealso%
Another method is defined on \genericref{as-uppercase} for
\hyperref[string]{strings}.

\method{as-uppercase}
%
\begin{specargs}
    \item[character, \classref{character}] A character.
\end{specargs}
%
\result%
If {\em character\/} denotes an lower case character, a character denoting its
upper case counterpart is returned.  Otherwise the result is the argument.

\method{generic-prin}
%
\begin{specargs}
    \item[character, \classref{character}] Character to be ouptut on {\em
        stream}.
    \item[stream, \classref{stream}] Stream on which {\em character\/} is to be
    ouptut.
\end{specargs}
%
\result%
The character {\em character}.
%
\remarks%
Output the interpretation of {\em character\/} on {\em stream}.

\method{generic-write}
%
\begin{specargs}
    \item[character, \classref{character}] Character to be ouptut on {\em
        stream}.
    \item[stream, \classref{stream}] Stream on which {\em character\/} is to be
    ouptut.
\end{specargs}
%
\result%
The character {\em character}.
%
\remarks%
Output external representation of {\em character\/} on {\em stream\/} in the
format \verb+#\+{\em{}name\/} as described at the beginning of this section.
%
\end{optDefinition}
