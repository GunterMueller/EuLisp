\sclause{Conversion}
\label{convert}
\index{general}{level-0 modules!convert}
\index{general}{convert!module}
%
\begin{optDefinition}
The defined name of this module is {\tt convert}.

The mechanism for the conversion of an instance of one class to an instance of
another is defined by a user-extensible framework which has some similarity to
the {\tt setter} mechanism.

To the user, the interface to conversion is via the function {\tt convert},
which takes an object and some class to which the object is to be converted.
The target class is used to access an associated {\em converter\/} function, in
fact, a generic function, which is applied to the source instance, dispatching
on its class to select the method which implements the appropriate conversion.
Thus, having defined a new class to which it may be desirable to convert
instances of other classes, the programmer defines a generic function:

{\small\syntax
(defgeneric (converter \[{\em new-class}\]) (instance))
\endsyntax}

Hereafter, new converter methods may be defined for {\em new-class\/}
using a similar extended syntax for {\tt defmethod}:

{\small\syntax
(defmethod (converter \[{\em new-class}\])
           ((instance \[{\em other-class}\])))
\endsyntax}

The conversion is implemented by defining methods on the converter for {\em
    new-class\/} which specialize on the source class.  This is also how methods
are documented in this text: by an entry for a method on the converter function
for the target class.  In general, the method for a given source class is
defined in the section about that class, for example, converters from one kind
of collection to another are defined in section~\ref{collection}, converters
from string in section~\ref{string}, etc..

\function{convert}
%
\begin{arguments}
    \item[object] An instance of some class to be converted to an instance of
    {\em class}.
    %
    \item[class] The class to which {\em object\/} is to be converted.
\end{arguments}
%
\result%
Returns an instance of {\em class\/} which is equivalent in some
class-specific sense to {\em object\/}, which may be an instance of any type.
Calls the converter function associated with {\em class\/} to carry out the
conversion operation.  An error is signalled (condition: {\tt
    no-converter}\indexcondition{no-converter}) if there is no associated
function.  An error is signalled (condition: {\tt
    no-applicable-method}\indexcondition{no-applicable-method}) if there is no
method to convert an instance of the class of {\em object\/} to an instance of
{\em class\/}.

\condition{conversion-condition}{condition}
%
This is the general condition class for all conditions arising from conversion
operations.
%
\begin{initoptions}
    \item[source, \classref{object}] The object to be converted into an instance of
    {\em target-class}.
    %
    \item[target-class, \classref{class}] The target class for the conversion
    operation.
\end{initoptions}
%
\remarks%
Should be signalled by {\tt convert} or a converter method.

\function{converter}
%
\begin{arguments}
    \item[target-class] The class whose set of conversion methods is required.
\end{arguments}
%
\result%
The accessor returns the converter function for the class {\em
    target-class}.  The converter is a generic-function with methods specialized
on the class of the object to be converted.

\setter{converter}
%
\begin{arguments}
    \item[target-class] The class whose converter function is to be replaced.
    %
    \item[generic-function] The new converter function.
\end{arguments}
%
\result%
The new converter function.  The setter function replaces the converter
function for the class {\em target-class\/} by {\em generic-function}.  The new
converter function must be an instance of {\tt <generic-function>}.
%
\remarks%
Converter methods from one class to another are defined in the section
pertaining to the source class.
%
\seealso%
Converter methods are defined for collections (\ref{collection}), double
float (\ref{double-float}), fixed precision integer (\ref{fpi}),
string (\ref{string}), symbol (\ref{symbol}), vector (\ref{vector}).
%
\end{optDefinition}
