\clause{Modules}
%
\label{sec:modules}
\index{general}{module}
%
\begin{optDefinition}
The \eulisp\ module scheme has several influences:
LeLisp's\index{general}{LeLisp} module system and module compiler (complice),
Haskell\index{general}{Haskell}, ML\index{general}{Standard
    ML}~\bref{ML-modules}, MIT-Scheme's {\tt make-environment} and T's locales.

All bindings of objects in \eulisp\ reside in some module somewhere.  Also, all
programs in \eulisp\ are written as one or more modules.  Almost every module
imports a number of other modules to make its definition meaningful.  These
imports have two purposes, which are separated in \eulisp: firstly the bindings
needed to process the syntax in which the module is written, and secondly the
bindings needed to resolve the free variables in the module after syntax
expansion.  These bindings are made accessible by specifying which modules are
to be imported for which purpose in a directive at the beginning of each module.
The names of modules are bound in a disjoint binding
environment\index{general}{module!name bindings}\index{general}{binding!of
    module names} which is only accessible via the module definition form.  That
is to say, modules are not first-class.  The body of a module definition
comprises a list of directives followed by a sequence of definitions,
expressions and export forms.

The module mechanism provides abstraction and security in a form complementary
to that provided by the object system.  Indeed, although objects do support data
abstraction, they do not support all forms of information hiding and they are
usually conceptually smaller units than modules.  A module defines a mapping
between a set of names and either local or imported bindings of those names.
Most such bindings are immutable.  The exception are those bindings created by
\defformref{deflocal} which may be modified by both the defining and importing
modules.  There are no implicit imports into a module---not even the special
forms are available in a module that imports nothing.  A module exports nothing
by default.  Mutually referential modules are not possible because a module must
be defined before it can be used.  Hence, the importation dependencies form a
directed acyclic graph.
%
\begin{figure*}[t]
\begin{example}
\label{example:module}
\examplecaption{module directives}
\begin{center}
\begin{minipage}[t]{\textwidth}{
\syntax
(defmodule a-module
  (import
    (module-1                                        ;; import everything from module-1
     (except (binding-a) module-2)                   ;; all but binding-a from module-2
     (only (binding-b) module-3)                     ;; only binding-b from module-3
     (rename
      ((binding-c binding-d) (binding-d binding-c))  ;; all of module-4, but exchange
      module-4))                                     ;; the names of binding-c and binding-d

   syntax
    (syntax-module-1                                 ;; all of the module syntax-module-1
     (rename ((syntax-a syntax-b))                   ;; rename the binding of syntax-a
      syntax-module-2)                               ;; of syntax-module-2 as syntax-b
     (rename ((syntax-c syntax-a))                   ;; rename the binding of syntax-c
      syntax-module-3))                              ;; of syntax-module-3 as syntax-a

   expose
    ((except (binding-e) module-5)                   ;; all but binding-e from module-5
     module-6)                                       ;; export all of module-6

   export
    (binding-1 binding-2 binding-3))                 ;; and three bindings from this module
  ...
  (export local-binding-4)                           ;; a fourth binding from this module
  ...
  (export binding-c)                                 ;; the imported binding binding-c
  ...)
\endsyntax
}
\end{minipage}
\end{center}
\end{example}
\end{figure*}
%
The processing of a module definition uses three environments, which are
initially empty.  These are the top-lexical, the external and the syntax
environments of the module.
%
\begin{description}
    \item[top-lexical] The top-lexical environment comprises all the locally
    defined bindings and all the imported bindings.

    \item[external] The external environment comprises all the exposed
    bindings---bindings from modules being exposed by this module but not
    necessarily imported---and all the exported bindings, which are either local
    or imported.  Thus, the external environment might not be a subset of the
    top-lexical environment because, by virtue of an expose directive, it can
    contain bindings from modules which have not been imported.  This is the
    environment received by any module importing this module.

    \item[syntax] The syntax environment comprises all the bindings available
    for the syntax expansion of the module.
\end{description}
%
Each binding is a pair of a local-name and a module-name.  It is a violation
if any two instances of local-name in any one of these environments have
different module-names.  This is called a name clash.  These environments do not
all need to exist at the same time, but it is simpler for the purposes of
definition to describe module processing as if they do.
\end{optDefinition}

\sclause{Directives}
\index{general}{module!directives}
\label{directives}
%
\begin{optDefinition}
The list of module directives is a sequence of keywords and forms, where the
keywords indicate the interpretation of the forms (see syntax
table~\ref{defmodule-syntax}).  This representation allows for the addition of
further keywords at other levels of the definition and also for
implementation-defined keywords\index{general}{implementation-defined!module
    directives}.  For the keywords given here, there is no defined order of
appearance, nor is there any restriction on the number of times that a keyword
can appear.  Multiple occurrences of any of the directives defined here are
treated as if there is a single directive whose form is the combination of each
of the occurrences.  This definition describes the processing of four keywords,
which are now described in detail.  The syntax of all the directives is given in
syntax syntax table~\ref{defmodule-syntax} and an example of their use appears
in example~\ref{example:module}.
\end{optDefinition}
%
\sclause{Export Directive}
%
\index{general}{module!export}
\begin{optDefinition}
This is denoted by the keyword {\tt export} followed by a sequence of
names of top-lexical bindings---these could be either locally-defined or
imported---and has the effect of making those bindings accessible to any module
importing this module by adding them to the external environment of the module.
A name clash can arise in the external environment from interaction with exposed
modules.
\end{optDefinition}
%
\sclause{Import Directive}
%
\index{general}{module!import}
\label{import}
%
\begin{optDefinition}
This is denoted by the keyword {\tt import} followed by a sequence of {\em
    module descriptor}s (see syntax table~\ref{defmodule-syntax}), being module
names or the filters {\tt except}, {\tt only} and {\tt rename}.  This sequence
denotes the union of all the names generated by each element of the sequence.  A
filter can, in turn, be applied to a sequence of module descriptors, and so the
effect of different kinds of filters can be combined by nesting them.  An import
directive specifies either the importation of a module in its entirety or the
selective importation of specified bindings from a module.

The purpose of this directive is to specify the imported bindings which
constitute part of the top-lexical environment of a module.  These are the
explicit run-time dependencies of the module.  Additional run-time dependencies
may arise as a result of syntax expansion.  These are called implicit run-time
dependencies.

In processing import directives, every name should be thought of as a pair of a
{\em module name} and a {\em local-name}.  Intuitively, a namelist of such pairs
is generated by reference to the module name and then filtered by {\tt except},
{\tt only} and {\tt rename}.  In an import directive, when a namelist has been
filtered, the names are regarded as being defined in the top-lexical environment
of the module into which they have been imported.  A name clash can arise in the
top-lexical environment from interaction between different imported modules.
Elements of an import directive are interpreted as follows:
%
\begin{description}
    \item[{\em module name}] All the exported names from {\em module name}.

    \item[{\tt except}] Filters the names from each {\em module descriptor}
    discarding those specified and keeping all other names.  The {\tt except}
    directive is convenient when almost all of the names exported by a module
    are required, since it is only necessary to name those few that are not
    wanted to exclude them.

    \item[{\tt only}] Filters the names from each {\em module descriptor}
    keeping only those names specified and discarding all other names.  The {\tt
        only} directive is convenient when only a few names exported by a module
    are required, since it is only necessary to name those that are wanted to
    include them.

    \item[{\tt rename}] Filters the names from each {\em module descriptor}
    replacing those with {\em old-name} by {\em new-name}.  Any name not
    mentioned in the replacement list is passed unchanged.  Note that once a
    name has been replaced the new-name is not compared against the replacement
    list again.  Thus, a binding can only be renamed once by a single {\tt
        rename} directive.  In consequence, name exchanges are possible.
\end{description}
\end{optDefinition}
%
\sclause{Expose Directive}
\index{general}{module!expose}
\begin{optDefinition}
This is denoted by the keyword {\tt expose} followed by a list of {\em module
    directive\/}s (see syntax table~\ref{defmodule-syntax}).  The purpose of this
directive is to allow a module to export subsets of the external environments of
various modules without importing them itself.  Processing an expose directive
employs the same model as for imports, namely, a pair of a module-name and a
local-name with the same filtering operations.  When the namelist has been
filtered, the names are added to the external environment of the module begin
processed.  A name clash can arise in the external environment from interaction
with exports or between different exposed modules.  As an example of the use of
{\tt expose}, a possible implementation of the {\tt
    eulisp0}\index{general}{level-0 modules!eulisp0} module is shown in
example~\ref{example:exposes}.
%
\begin{example}
\label{example:exposes}
\examplecaption{module using {\tt expose}}{
\syntax
(defmodule eulisp-level-0
  (expose
    (character collection compare condition convert
     copy double-float elementary-functions event
     formatted-io fixed-precision-integer function
     keyword list lock number object-0 stream string
     symbol syntax-0 table thread vector)))
\endsyntax
}
\end{example}
%
It is also meaningful for a module to include itself in an expose
directive.  In this way, it is possible to refer to all the bindings
in the module being defined.  This is convenient, in combination with
{\tt except} (see \S~\ref{import}), as a way of exporting all but
a few bindings in a module, especially if syntax expansion creates
additional bindings whose names are not known, but should be exported.
\end{optDefinition}
%
\sclause{Syntax Directive}
\label{syntax-directive}
\index{general}{module!syntax}
\begin{optDefinition}
This directive is processed in the same way as an import directive,
except that the bindings are added to the syntax environment.  This
environment is used in the second phase of module processing (syntax
expansion).  These constitute the dependencies for the syntax expansion
of the definitions and expressions in the body of the module.  A name
clash can arise in the syntax environment from interaction between
different syntax modules.

It is important to note that special forms are considered part of the
syntax and they may also be renamed.
\end{optDefinition}

\sclause{Definitions and Expressions}
%
\begin{optDefinition}
Definitions in a module only contain unqualified names---that is, {\em
    local-name}s, using the above terminology.  A top-lexical binding is created
exactly once and shared with all modules that import its exported name from the
module that created the binding.  A name clash can arise in the top-lexical
environment from interaction between local definitions and between local
definitions and imported modules.  Only top-lexical bindings created by
\defformref{deflocal} are mutable---both in the defining module and in any
importing module.  It is a violation to modify an immutable binding.
Expressions, that is non-defining forms, are collected and evaluated in order of
appearance at the end of the module definition process when the top-lexical
environment is complete---that is after the creation and initialization of the
top-lexical bindings.  The exception to this is the \keywordref{progn} form,
which is descended and the forms within it are treated as if the
\keywordref{progn} were not present.  Definitions may only appear either at
top-level within a module definition or inside any number of \keywordref{progn}
forms.  This is specified more precisely in the grammar for a module given
below.
%
\Syntax
\savesyntax\defmoduleZeroSyntax\vbox{
\syntax
module syntax
   = '(', 'defmodule', module name,
     module directives, {level 0 module form}, ')';
module name
   = identifier; (* \[\S\ref{symbol}\] *)
module directives
   = '(', {module directive}, ')';
module directive
   = 'export', '(', {identifier}, ')'
   | 'expose', '(', {module descriptor}, ')'
   | 'import', '(', {module descriptor}, ')'
   | 'syntax', '(', {module descriptor}, ')';
level 0 module form
   = '(', 'export', {identifier}, ')'
   | level 0 expression (* \[\S\ref{control-0}\] *)
   | defining form (* \[\S\ref{control-0}\] *)
   | '(', 'progn', {level 0 module form}, ')';
module descriptor
   = module name
   | module filter;
module filter
   = '(', 'except', '(', {identifier}, ')',
     module descriptor, ')'
   | '(', 'only', '(', {identifier}, ')',
     module descriptor, ')'
   | '(', 'rename', '(', {rename pair}, ')',
     module descriptor, ')';
rename pair
   = '(', identifier, identifier, ')';
\endsyntax
}
%
\savesyntax\defmoduleOneSyntax\vbox{
\syntax
module
   = '(', 'defmodule', module name,
     module directives, {level 1 module form}, ')';
level 1 module form
   = level 1 expression (* \[\S\ref{control-1}\] *)
   | level 0 module form;
\endsyntax
}
\label{defmodule-syntax}
\syntaxtable{module}{\defmoduleZeroSyntax}
%
\end{optDefinition}

\sclause{Module Processing}
%
\begin{optPrivate}
    GN is unhappy with the definition before use requirement.  It could be
    cleaned up by requiring the source to exist but not th e processed module.
\end{optPrivate}
%
\begin{optDefinition}
The following steps summarize the module definition process:
%
\begin{description}
    \item[{\bf directive processing}] This is described in detail in
    \S~\ref{directives}--\ref{syntax-directive}.  This step creates and
    initializes the top-lexical, syntax and external environments.

    \item[{\bf syntax expansion}] \index{general}{macros---see also syntax}
    \index{general}{macro expansion---see also syntax} \index{general}{syntax}
    \index{general}{syntax expansion} The body of the module is expanded
    according to the operators defined in the syntax environment constructed
    from the syntax directive.
    \begin{note}
        The semantics of syntax expansion are still under discussion and will be
        described fully in a future version of the \eulisp\ definition.  In
        outline, however, it is intended that the mechanism should provide for
        hygenic expansion of forms in such a way that the programmer need have
        no knowledge of the expansion-time or run-time dependencies of the
        syntax defining module.  Currently syntax expansion is unhygienic to
        allow a simple syntax for syntax macro definition.
    \end{note}

    \item[{\bf static analysis}] The expanded body of the module is analyzed.
    Names referenced in export forms are added to the external environment.
    Names defined by defining forms are added to the top-lexical environment. It
    is a violation, if a free identifier in an expression or defining form
    does not have a binding in the top-lexical environment.
    \begin{note}
        Additional implementation-defined steps may be added here, such as
        compilation.
    \end{note}

    \item[{\bf initialization}] The top-lexical bindings of the module (created
    above) are initialized by evaluating the defining forms in the body of the
    module in the order they appear.
    \begin{note}
        In this sense, a module can be regarded as a generalization of the
        \keywordref{labels} form of this and other Lisp dialects.
    \end{note}

    \item[{\bf expression evaluation}] The expressions in the body of the module
    are evaluated in the order in which they appear.
\end{description}
%
\end{optDefinition}
%
\sclause{Module Definition}
\label{defmodule}
\ttindex{defmodule}
%
\begin{optDefinition}
%
\syntaxform{defmodule}
\Syntax
\noindent
The syntax of a module and its constituents is defined in
syntax table~\ref{defmodule-syntax}.
%
\begin{arguments}
    \item[module name] A symbol used to name the module.
    \item[module directives] A form specifying the exported names, exposed
    modules, imported modules and syntax modules used by this module.
    \item[module form] One of a defining form, an expression or an export
    directive.
\end{arguments}
%
\remarks%
The \syntaxref{defmodule} form defines a module named by {\em module name}
and associates the name {\em module name} with a module object in the
module binding environment\index{general}{binding!of module
names}\index{general}{module!name bindings}.
\begin{note}
    Intentionally, nothing is defined about any relationship between modules and
    files.
\end{note}
%
\examples
An example module definition with explanatory comments is given in
example~\ref{example:module}.
%
\end{optDefinition}
