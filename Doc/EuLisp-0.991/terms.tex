\clause{Definitions}
\begin{optPrivate}
\end{optPrivate}
\begin{optDefinition}
\label{sec:definitions}
This set of definitions, which are be used throughout this document, is
self-consistent but might not agree with notions accepted in other language
definitions.  The terms are defined in alphabetical rather than dependency order
and where a definition uses a term defined elsewhere in this section it is
written in {\em italics}.  Names in {\tt courier} font refer to entities defined
in the language.
%
\begin{definitions}
    \definition{abstract class}\index{general}{abstract class}
    A Class that by definition has no direct instances.

    \definition{applicable method} \index{general}{applicable method}
    \index{general}{method!applicable}  A \sc{method} is applicable for a
    particular set of arguments if each element in its \sc{domain} is a
    \sc{superclass} of the \sc{class} of the corresponding argument.

    \definition{binding}\index{general}{binding}
    A location containing a value.

    \definition{class}\index{general}{class} A class is an \scref{object} which
    describes the structure and behaviour of a set of \scref{object}s which are
    its \sc{instances}.  A \sc{class} object contains \sc{inheritance}
    information and a set of \scref{slot} \sc{description}s which define the
    structure of its \sc{instances}.  A \sc{class} \scref{object} is an
    \sc{instance} of a \sc{metaclass}.  All \sc{classes} in \eulisp\ are
    \sc{subclasses} of {\classref{object}}, and all \sc{instances} of
    {\classref{class}} are \sc{classes}.

    \definition{class precedence list} \index{general}{class precedence list}
    Each \sc{class} has a linearised list of all its \sc{super-classes},
    \sc{direct} and \sc{indirect}, beginning with the \sc{class} itself and
    ending with the root of the \sc{inheritance} \sc{graph}, the \sc{class}
    \classref{object}.  This list determines the specificity of slot and method
    \sc{inheritance}.  A class's class precedence list may be accessed through
    the \sc{function} \functionref{class-precedence-list}. The rules used to
    compute this list are determined by the \sc{class} of the \sc{class} through
    \sc{methods} of the \sc{generic function}
    \genericref{compute-class-precedence-list}.

    \definition{class option}\index{general}{class option} A keyword and its
    associated value applying to a \sc{class} appearing in a class definition
    form, for example: the {\tt predicate} keyword and its value, which defines
    a predicate \sc{function} for the \sc{class} being defined.

    \definition{closure}\index{general}{closure} A first-class function with
    \sc{free variables} that are bound in the \sc{lexical environment}. Such a
    function is said to be ``closed over'' its free variables.  Example: the
    function returned by the expression {\tt (let ((x 1)) \#'(lambda () x))} is
    a closure since it closes over the free variable {\tt x}.

    \definition{congruent}\index{general}{congruent} A constraint on the form of
    the \scref{lambda-list} of a method, which requires it to have the same
    number of elements as the generic function to which it is to be attached.

    \definition{continuation}\index{general}{continuation} A continuation is a
    function of one parameter which represents the rest of the program.  For
    every point in a program there is the remainder of the program coming after
    that point; this can be viewed as a function of one argument awaiting the
    result of that point.  The \sc{current continuation} is the continuation
    that would be derived from the current point in a program's execution, see
    \specopref{let/cc}.

    \definition{converter function}\index{general}{converter function} The
    generic function associated with a class (the target) that is used to
    project an instance of another class (the source) to an instance of the
    target.

    \definition{defining form}\index{general}{defining form} Any form or
    \sc{macro expression} expanding into a form whose operator is a \sc{defining
        operator}.

    \definition{defining operator}\index{general}{defining operator} One of
    \defopref{defclass}, \defopref{defcondition}, \defopref{defconstant},
    \defopref{defgeneric}, \defopref{deflocal}, \defopref{defmacro},
    \defopref{defun}, or \defopref{defglobal}.

    \definition{direct instance}\index{general}{direct
        instance}\index{general}{instance!direct} A direct instance of a class
    \sc{class$_1$} is any \scref{object} whose most specific \sc{class} is
    \sc{class$_1$}.

    \definition{direct subclass}\index{general}{direct
        subclass}\index{general}{subclass!direct} A \sc{class$_1$} is a direct
    \sc{subclass} of \sc{class$_2$} if \sc{class$_1$} is a \sc{subclass} of
    \sc{class$_2$}, \sc{class$_1$} is not identical to \sc{class$_2$}, and there
    is no other \sc{class$_3$} which is a \sc{superclass} of \sc{class$_1$} and
    a \sc{subclass} of \sc{class$_2$}.

    \definition{direct superclass} \index{general}{direct superclass}
    \index{general}{superclass!direct} A \sc{direct superclass} of a class
    \sc{class$_1$} is any \sc{class} for which \sc{class$_1$} is a direct
    \sc{subclass}.

    \definition{dynamic environment}\index{general}{dynamic environment} The
    \sc{inner} and \sc{top dynamic} environment, taken together, are referred
    to as the dynamic environment.

    \definition{dynamic extent}\index{general}{dynamic extent} A lifetime
    constraint, such that the entity is created on control entering an
    expression and destroyed when control exits the expression.  Thus the entity
    only exists for the time between control entering and exiting the
    expression.

    \definition{dynamic scope}\index{general}{dynamic scope} An access
    constraint, such that the \sc{scope} of the entity is limited to the
    \sc{dynamic extent} of the expression that created the entity.

    \definition{extent}\index{general}{extent} That lifetime for which an entity
    exists.  Extent is constrained to be either \sc{dynamic} or \sc{indefinite}.

    \definition{first-class}\index{general}{first-class} First-class entities
    are those which can be passed as parameters, returned from functions, or
    assigned into a variables.

    \definition{function}\index{general}{function} A function is either a
    \sc{continuation}, a \sc{simple function} or a \sc{generic function}.

    \definition{generic function} \index{general}{generic function} Generic
    functions are \sc{functions} for which the \sc{method} executed depends on
    the \sc{class} of its arguments.  A generic function is defined in terms of
    \sc{methods} which describe the action of the generic function for a
    specific set of argument classes called the method's \sc{domain}.

    \definition{identifier}\index{general}{identifier} An identifier is the
    syntactic representation of a \sc{variable}.

    \definition{indefinite extent}\index{general}{indefinite extent} A lifetime
    constraint, such that the entity exists for ever.  In practice, this means
    for as long as the entity is accessible.

    \definition{indirect instance}\index{general}{indirect
        instance}\index{general}{instance!indirect} An indirect instance of a
    class \sc{class$_1$} is any \scref{object} whose \sc{class} is an
    \sc{indirect} \sc{subclass} of \sc{class$_1$}.

    \definition{indirect subclass}\index{general}{indirect
        subclass}\index{general}{subclass!indirect} A \sc{class$_1$} is an
    indirect subclass of \sc{class$_2$} if \sc{class$_1$} is a \sc{subclass}
    of \sc{class$_2$}, \sc{class$_1$} is not identical to \sc{class$_2$}, and
    there is at least one other \sc{class$_3$} which is a \sc{superclass} of
    \sc{class$_1$} and a \sc{subclass} of \sc{class$_2$}.

    \definition{inheritance graph} \index{general}{inheritance graph} A directed
    labelled acyclic graph whose nodes are \sc{classes} and whose edges are
    defined by the \sc{direct subclass} relations between the nodes.  This
    graph has a distinguished root, the \sc{class} \classref{object}, which is
    a \sc{superclass} of every \sc{class}.

    \definition{inherited slot description}\index{general}{inherited slot
        description} A \scref{slot} \sc{description} is inherited for a
    \sc{class$_1$} if the \scref{slot} \sc{description} is defined for another
    \sc{class$_2$} which is a direct or indirect \sc{superclass} of
    \sc{class$_1$}.

    \definition{initarg} \index{general}{initarg} A \scref{keyword} used in an
    \scref{initlist} to mark the value of some \scref{slot} or additional
    information.  Used in conjunction with \functionref{make} and the other
    \scref{object} initialization functions to initialize the object.  An
    \scref{initarg} may be declared for a \scref{slot} in a class definition
    form using the \keywordref{keyword} \scref{slot-option} or the
    \keywordref{keywords} \sc{class-option}.

    \definition{initform} \index{general}{initform} A form which is evaluated to
    produce a default initial \scref{slot} value.  Initforms are closed in their
    \sc{lexical} environments and the resulting \sc{closure} is called an
    \sc{initfunction}.  An initform may be declared for a \scref{slot} in a
    class definition form using the \keywordref{default} \scref{slot-option}.

    \definition{initfunction} \index{general}{initfunction} A \sc{function} of
    no arguments whose result is used as the default value of a \scref{slot}.
    Initfunctions capture the \sc{lexical} environment of an \sc{initform}
    declaration in a class definition form.

    \definition{initlist} \index{general}{init-list} A list of alternating
    keywords and values which describes some not-yet instantiated object.
    Generally the keywords correspond to the \sc{initargs} of some \sc{class}.

    \definition{inner dynamic}\index{general}{inner dynamic} Inner dynamic
    bindings are created by \specopref{dynamic-let}, referenced by
    \specopref{dynamic} and modified by \specopref{dynamic-setq}.  Inner dynamic
    bindings extend---and can shadow---the dynamically enclosing \sc{dynamic
        environment}.

    \definition{inner lexical}\index{general}{inner lexical} Inner lexical
    bindings are created by \specopref{lambda} and \specopref{let/cc},
    referenced by \sc{variables} and modified by \specopref{setq}.  Inner
    lexical bindings extend---and can shadow---the lexically enclosing
    \sc{lexical environment}.  Note that \specopref{let/cc} creates an immutable
    \sc{binding}.

    \definition{instance} \index{general}{instance} Every \scref{object} is the
    instance of some \sc{class}.  An instance thus describes an \scref{object}
    in relation to its \sc{class}.  An instance takes on the structure and
    behaviour described by its \sc{class}.  An instance can be either
    \sc{direct} or \sc{indirect}.

    \definition{instantiation graph} \index{general}{instantiation graph} A
    directed graph whose nodes are \scref{object}s and whose edges are defined
    by the \sc{instance} relations between the \scref{object}s.  This graph has
    only one cycle, an edge from \classref{class} to itself.  The instantiation
    graph is a tree and \classref{class} is the root.

    \definition{lexical environment}\index{general}{lexical environment} The
    \sc{inner} and \sc{top lexical} environment of a module are together
    referred to as the lexical environment except when it is necessary to
    distinguish between them.

    \definition{lexical scope}\index{general}{lexical scope} An access
    constraint, such that the \sc{scope} of the entity is limited to the
    textual region of the form creating the entity.  See also \sc{lexically
        closer} and \sc{shadow}.

    \definition{macro}\index{general}{macro} A macro is a function distinguished
    by when it is used: macro functions are only used during the syntax
    expansion of modules to transform expressions.

    \definition{metaclass} \index{general}{metaclass} A metaclass is a
    \sc{class} \scref{object} whose \sc{instances} are themselves \sc{classes}.
    All metaclasses in \eulisp\ are \sc{instances} of \classref{class}, which is
    an \sc{instance} of itself.  All metaclasses are also \sc{subclasses} of
    \classref{class}.  \classref{class} is a metaclass.

    \definition{method} \index{general}{method} A method describes the action of
    a \sc{generic-function} for a particular list of argument classes
    called the method's \sc{domain}.  A \sc{method} is thus said to add to the
    behaviour of each of the \sc{classes} in its \sc{domain}.  Methods are not
    \sc{functions} but \scref{object}s which contain, among other information, a
    \sc{function} representing the method's behaviour.

    \definition{method function} \index{general}{method function} A
    \sc{function} which implements the behaviour of a particular \sc{method}.
    Method functions have special restrictions which do not apply to all
    \sc{functions}: their formal parameter bindings are immutable, and the
    special operators \specopref{call-next-method} and \specopref{next-method-p}
    are only valid within the lexical scope of a method function.

    \definition{method specificity} \index{general}{method specificity}
    \index{general}{method!specificity} A domain \sc{domain$_1$} is more
    specific than another \sc{domain$_2$} if the first \sc{class} in
    \sc{domain$_1$} is a \sc{subclass} of the first \sc{class} in
    \sc{domain$_2$}, or, if they are the same, the rest of \sc{domain$_1$} is
    more specific than the rest of \sc{domain$_2$}.

    \definition{multi-method} \index{general}{multi-method} A \sc{method} which
    specializes on more than one argument.

    \definition{new instance}\index{general}{new instance} A newly allocated
    \sc{instance}, which is distinct, but can be isomorphic to other
    \sc{instances}.

    \definition{reflective} \index{general}{reflective} A system which can
    examine and modify its own state is said to be \sc{reflective}.  \eulisp\
    is reflective to the extent that it has explicit \sc{class} objects and
    \sc{metaclasses}, and user-extensible operations upon them.

    \definition{scope}\index{general}{scope} That part of the extent in which a
    given \sc{variable} is accessible.  Scope is constrained to be \sc{lexical},
    \sc{dynamic} or \sc{indefinite}.

    \definition{self-instantiated class} \index{general}{self-instantiated
        class} \index{general}{class!self-instantiated} A \sc{class} which is
    an \sc{instance} of itself.  In \eulisp, \classref{class} is the only
    example of a self-instantiated class.

    \definition{setter function}\index{general}{setter function} The function
    associated with the function that accesses a place in an entity, which
    changes the value stored in that place.

    \definition{simple function}\index{general}{simple function} A function
    comprises at least: an expression, a set of identifiers, which occur in the
    expression, called the parameters and the closure of the expression with
    respect to the \sc{lexical environment} in which it occurs, less the
    parameter identifiers.  Note: this is not a definition of the class
    \classref{simple-function}.

    \definition{slot} \index{general}{slot} A named component of an
    \scref{object} which can be accessed using the slot's \sc{accessor}.  Each
    \scref{slot} of an \scref{object} is described by a \sc{slot description}
    associated with the \sc{class} of the \scref{object}.  When we refer to the
    \sc{structure} of an \scref{object}, this usually means its set of
    \sc{slots}.

    \definition{slot description} \index{general}{slot description} A slot
    description describes a \scref{slot} in the \sc{instances} of a \sc{class}.
    This description includes the \sc{slot's} name, its logical position in
    \sc{instances}, and a way to determine its default value.  A \sc{class's}
    slot descriptions may be accessed through the \sc{function}
    \functionref{class-slots}.  A slot description can be either
    \sc{direct} or \sc{inherited}.

    \definition{slot option} \index{general}{slot option} A keyword and its
    associated value applying to one of the slots appearing in a class
    definition form, for example: the {\tt accessor} keyword and its
    value, which defines a function used to read or write the value of a
    particular slot.

    \definition{slot specification} \index{general}{slot specification} A list
    of alternating keywords and values (starting with a keyword) which
    represents a not-yet-created \scref{slot} \sc{description} during class
    initialization.

    \definition{special form} \index{general}{special form} Any form or
    \sc{macro expression} expanding into a form whose operator is a \sc{special
        operator}.  They are semantic primitives of the language and in
    consequence, any processor (for example, a compiler or a code-walker) need
    be able to process only the special forms of the language and compositions
    of them.

    \definition{special operator} \index{general}{special operator} One of
    \specopref{a-special-form}, \specopref{call-next-handler},
    \specopref{call-next-method}, \specopref{dynamic},
    \specopref{dynamic-let}, \specopref{dynamic-setq}, \specopref{if},
    \specopref{labels}, \specopref{lambda}, \specopref{let/cc},
    \specopref{next-method-p}, \specopref{progn}, \specopref{quote},
    \specopref{setq}, \specopref{unwind-protect}, or
    \specopref{with-handler}.

    \definition{specialize}\index{general}{specialize} A verbal form used to
    describe the creation of a more specific version of some entity.  Normally
    applied to classes, slots and methods.

    \definition{specialize on}\index{general}{specialize on} A verbal form used
    to describe relationship of methods and the classes specified in their
    domains.

    \definition{subclass} \index{general}{subclass} The behaviour and structure
    defined by a class \sc{class$_1$} are inherited by a set of \sc{classes}
    which are termed \sc{subclasses} of \sc{class$_1$}.  A \sc{subclass} can
    be either \sc{direct} or \sc{indirect} or itself.

    \definition{superclass} \index{general}{superclass} A \sc{class$_1$} is a
    superclass of \sc{class$_2$} iff \sc{class$_2$} is a subclass of
    \sc{class$_1$}.  A \sc{superclass} can be either \sc{direct} or
    \sc{indirect} or itself.

    \definition{top dynamic}\index{general}{top dynamic} Top dynamic bindings
    are created by \defopref{defglobal}, referenced by \specopref{dynamic}
    and modified by \specopref{dynamic-setq}.  There is only one \sc{top
        dynamic} environment.

    \definition{top lexical}\index{general}{top lexical} Top lexical bindings
    are created in the \sc{top} \sc{lexical} environment of a module by a
    defining form.  All these bindings are immutable except those created by
    \defopref{deflocal} which creates a mutable top-lexical binding.  All such
    bindings are referenced by \sc{variables} and those made by
    \defopref{deflocal} are modified by \specopref{setq}.  Each module defines
    its own distinct \sc{top lexical} environment.
%
\end{definitions}
\end{optDefinition}
