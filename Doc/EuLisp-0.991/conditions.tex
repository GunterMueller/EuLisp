\clause{Conditions}
\label{condition}
\index{general}{condition}
\index{general}{level-0 modules!condition}
\gdef\module{condition}
%
\begin{optDefinition}
The defined name of this module is {\tt condition}.

The condition system was influenced by the Common Lisp error system
~\bref{Pit-errors} \index{general}{Common Lisp Error System} and the Standard
ML\index{general}{Standard ML} exception mechanism.  It is a simplification of
the former and an extension of the latter.  Following standard practice, this
text defines the actions of functions in terms of their normal behaviour.  Where
an exceptional behaviour might arise, this has been defined in terms of a
condition.  However, not all exceptional situations are errors.  Following
Pitman, we use {\em condition} \index{general}{condition} to be a kind of
occasion in a program when an exceptional situation has been signalled.  An
error is a kind of condition---error and condition are also used as terms for
the objects that represent exceptional situations.  A condition can be signalled
continuably by passing a continuation for the resumption to signal.  If a
continuation is not supplied then the condition cannot be continued.

\noindent
These two categories are characterized as follows:
\begin{enumerate}
    \item \index{general}{condition!continuable} A condition might be signalled
    when some limit has been transgressed and some corrective action is needed
    before processing can resume.  For example, memory zone exhaustion on
    attempting to heap-allocate an item can be corrected by calling the memory
    management scheme to recover dead space.  However, if no space was recovered
    a new kind of condition has arisen.  Another example arises in the use of
    IEEE floating point arithmetic, where a condition might be signalled to
    indicate divergence of an operation.  A condition should be signalled
    continuably when there is a strategy for recovery from the condition.

    \item \index{general}{condition!non-continuable} Alternatively, a condition
    might be signalled when some catastrophic situation is recognized, such as
    the memory manager being unable to allocate more memory or unable to recover
    sufficient memory from that already allocated.  A condition should be
    signalled non-continuably when there is no reasonable way to resume
    processing.
\end{enumerate}
%
A condition class is defined using \defopref{defcondition} (see
\S~\ref{defcondition}).  The definition of a condition causes the
creation of a new class of condition.  A condition is signalled using
the function \functionref{signal}, which has two required arguments and one
optional argument: an instance of a condition, a resume continuation
or the empty list---the latter signifying a non-continuable signal---and a
thread.  A condition can be handled using the special form {\tt
with-handler}, which takes a function---the handler function---and a
sequence of forms to be protected.  The initial condition class
hierarchy is shown in table~\ref{condition-class-hierarchy}.
%
\begin{table}%
\caption{Condition class hierarchy}%
\label{condition-class-hierarchy}%
{\tt%
    \begin{tabbing}%
    00\=00\=00\=00\= \kill%
    \classref{condition} \\
    \>\conditionref{general-condition} \\
    \>\>\conditionref{invalid-operator} \\
    \>\>\conditionref{cannot-update-setter} \\
    \>\>\conditionref{no-setter} \\
    \>\conditionref{environment-condition} \\
    \>\conditionref{arithmetic-condition} \\
    \>\> \conditionref{division-by-zero} \\
    \>\conditionref{conversion-condition} \\
    \>\>\conditionref{no-converter} \\
    \>\conditionref{stream-condition} \\
    \>\>\conditionref{end-of-stream} \\
    \>\conditionref{read-error} \\
    \>\conditionref{thread-condition} \\
    \>\>\conditionref{thread-already-started} \\
    \>\>\conditionref{wrong-thread-continuation} \\
    \>\>\conditionref{wrong-condition-class} \\
    \>\conditionref{generic-function-condition} \\
    \>\>\conditionref{no-next-method} \\
    \>\>\conditionref{non-congruent-lambda-lists} \\
    \>\>\conditionref{incompatible-method-domain} \\
    \>\>\conditionref{no-applicable-method} \\
    \>\>\conditionref{method-domain-clash}
\end{tabbing}%
}%
\end{table}
%
\end{optDefinition}

\clause{Condition Classes}

\begin{optDefinition}
%
\defop{defcondition}
\label{defcondition}
%
\Syntax
\label{defcondition-syntax}
\defSyntax{defcondition}{
\begin{syntax}
    \scdef{defcondition-form}: \\
    \>  ( \defopref{defcondition} \scref{condition-class-name} \\
    \>\>  \scref{condition-superclass-name} \\
    \>\>  \scseqref{slot-name} ) \\
    \scdef{condition-class-name}: \\
    \> \scref{identifier} \\
    \scdef{condition-superclass-name}: \\
    \> \scref{identifier}
\end{syntax}}%
\showSyntaxBox{defcondition}
%
\begin{arguments}
    \item[\scref{condition-class-name}] A symbol naming a binding to be
    initialized with the new condition class.

    \item[\scref{condition-superclass-name}] A symbol naming a binding of a
    class to be used as the superclass of the new condition class.

    \item[\scseqref{slot-name}] A sequence of keywords used to name and to
    identify additional slots begin defined for this condition class.
\end{arguments}
%
\remarks%
This defining form defines a new condition class.  The first argument is the
name to which the new condition class will be bound.  The second is the name of
the superclass of the new condition class and an {\em init-option\/} is an
identifier followed by its (default) initial value.  If {\em superclass-name\/}
is \nil{}, the superclass is taken to be \classref{condition}.  Otherwise {\em
    superclass-name\/} must be \classref{condition}\ or the name of one of its
subclasses.

\derivedclass{condition}{object}
\label{\conditionlabel{condition}}
%
\begin{initoptions}
    \item[message, \classref{string}] A string, containing information which
    should pertain to the situation which caused the condition to be signalled.

    \item[message-arguments, \classref{list}] A list of objects to be used in
    processing the message format string.
\end{initoptions}
%
\remarks%
The class which is the superclass of all condition classes.

\function{condition?}
%
\begin{arguments}
    \item[object] An object to examine.
\end{arguments}
%
\result%
Returns {\em object\/} if it is a condition, otherwise \nil{}.

\method{initialize}{condition}
%
\begin{specargs}
    \item[condition, <condition>] a condition.
    \item[initlist] A list of initialization options as follows:
    \begin{options}
        \item[message, \classref{string}] A string, containing information which
        should pertain to the situation which caused the condition to be
        signalled.

        \item[message-arguments, \classref{list}] A list of objects to be used
        in processing the message format string.
    \end{options}
\end{specargs}
%
\result%
A new, initialized condition.
%
\remarks%
First calls \specopref{call-next-method} to carry out initialization specified
by superclasses then does the \classref{condition} specific initialization.
%

\condition{general-condition}{condition}
%
This is the general condition class for conditions arising from the
execution of programs by the processor.

\condition{domain-condition}{general-condition}
%
\begin{initoptions}
    \item[argument, \classref{object}] An argument, which was not of the
    expected class, or outside a defined range and therefore lead to the
    signalling of this condition.
\end{initoptions}

\condition{range-condition}{general-condition}
%
\begin{initoptions}
    \item[result, \classref{object}] A result, which was not of the expected
    class, or outside a defined range and therefore lead to the signalling of
    this condition.
\end{initoptions}

\condition{environment-condition}{condition}
%
This is the general condition class for conditions arising from the
environment of the processor.

\condition{wrong-condition-class}{thread-condition}
%
\begin{initoptions}
    \item[condition, condition] A condition.
\end{initoptions}
%
Signalled by \functionref{signal} if the given condition is not an instance of
the condition class \conditionref{thread-condition}.

\condition{generic-function-condition}{condition}
%
This is the general condition class for conditions arising from operations in
the object system at level 0. The following direct subclasses of
<generic-function-condition> are defined at level 0:
%
\begin{subclasses}
    \item[no-applicable-method] signalled by a generic function when there is
    no method which is applicable to the arguments.
    %
    \item[incompatible-method-domain] signalled by if the domain of the method
    being added to a generic function is not a subdomain of the generic
    function's domain.
    %
    \item[non-congruent-lambda-lists] signalled if the lambda list of the
    method being added to a generic function is not congruent to that of the
    generic function.
    %
    \item[method-domain-clash] signalled if the method being added to a
    generic function has the same domain as a method already attached to the
    generic function.
    %
    \item[no-next-method] signalled by call-next-method if there is no next
    most specific method.
\end{subclasses}
%
\condition{no-applicable-method}{generic-function-condition}
%
\begin{initoptions}
    \item[generic, function] The generic function which was applied.
    \item[arguments, list] The arguments of the generic function which was
    applied.
\end{initoptions}
%
\remarks%
Signalled by a generic function when there is no method which is applicable to
the arguments.

\condition{incompatible-method-domain}{generic-function-condition}
%
\begin{initoptions}
    \item[generic, function] The generic function to be extended.
    \item[method, method] The method to be added.
\end{initoptions}
%
\remarks%
Signalled by one of \defopref{defgeneric}, \macroref{defmethod} or
\macroref{generic-lambda} if the domain of the method would not be a subdomain
of the generic function's domain.

\condition{non-congruent-lambda-lists}{generic-function-condition}
%
\begin{initoptions}
    \item[generic, function] The generic function to be extended.
    \item[method, method] The method to be added.
\end{initoptions}
%
\remarks%
Signalled by one of \defopref{defgeneric}, \macroref{defmethod} or
\macroref{generic-lambda} if the lambda list of the method is not congruent to
that of the generic function.

\condition{method-domain-clash}{generic-function-condition}
%
\begin{initoptions}
    \item[generic, function] The generic function to be extended.
    \item[methods, list] The methods with the same domain.
\end{initoptions}
%
\remarks%
Signalled by one of \defopref{defgeneric}, \macroref{defmethod} or
\macroref{generic-lambda} if there would be methods with the same domain
attached to the generic function.

\condition{no-next-method}{generic-function-condition}
%
\begin{initoptions}
    % \item[generic] The generic function to which the method belongs.

    \item[method, method] The method which called \specopref{call-next-method}.

    \item[operand-list, list] A list of the arguments to have been passed to the
    next method.
\end{initoptions}
%
\remarks%
Signalled by \specopref{call-next-method} if there is no next most specific
method.
%
\end{optDefinition}

\clause{Condition Signalling and Handling}
\label{cond-hand}
%
\begin{optPrivate}
    Update Pitman reference (EUROPAL paper?).

    RPG: definition of condition is messy.  Also, example 1 is not good (too
    operational).  In general, conditions before classes is hard to follow.

    What happens if the resume continuation \nil{}\/ is called?

    Expand on {\em signaler's intention}.
\end{optPrivate}
%
\begin{optDefinition}
%
Conditions are handled with a function called a {\em handler}
\index{general}{error!handler}.  Handlers are established dynamically and have
dynamic scope and extent.  Thus, when a condition is signalled, the processor
will call the dynamically closest handler.  This can accept, resume or decline
the condition (see {\tt with-handler} for a full discussion and definition of
this terminology).  If it declines, then the next dynamically closest handler is
called, and so on, until a handler accepts or resumes the condition.  It is the
first handler accepting the condition that is used and this may not necessarily
be the most specific.  Handlers are established by the special form
\specopref{with-handler}.

\function{signal}
%
\begin{arguments}
    \item[condition] The condition to be signalled.

    \item[function] The function to be called if the condition is handled and
    resumed, that is to say, the condition is continuable, or \nil{}
    otherwise.

    \item[\optional{thread}] If this argument is not supplied, the condition is
    signalled on the thread which called \functionref{signal}, otherwise, {\em
        thread} indicates the thread on which {\em condition} is to be
    signalled.
\end{arguments}
%
\result%
\functionref{signal} should never return.  It is an error to call
\functionref{signal}'s continuation.
%
\remarks%
Called to indicate that a specified condition has arisen during the execution of
a program.
%
If the third argument is not supplied, \functionref{signal} calls the
dynamically closest handler with {\em condition} and {\em continuation}.  If the
second argument is a subclass of {\tt function}, it is the {\em resume}
continuation to be used in the case of a handler deciding to resume from a
continuable condition.  If the second argument is \nil{}, it indicates that the
condition was signalled as a non-continuable condition---in this way the handler
is informed of the signaler's intention.

If the third argument is supplied, \functionref{signal} registers the specified
condition to be signalled on {\em thread}.  The condition must be an instance of
the condition class \conditionref{thread-condition}, otherwise an error is
signalled (condition class:
\conditionref{wrong-condition-class}\indexcondition{wrong-condition-class}) on
the thread calling \functionref{signal}.  A \functionref{signal} on a determined
thread has no effect on either the signalled or signalling thread except in the
case of the above error.
%
\seealso%
\functionref{thread-reschedule}, \functionref{thread-value}, \specopref{with-handler}.

\specop{call-next-handler}
%
\Syntax
\label{call-next-handler-syntax}
\defSyntax{call-next-handler}{
\begin{syntax}
    \scdef{call-next-handler-form}: \\
    \>  ( \specopref{call-next-handler} )
\end{syntax}}%
\showSyntaxBox{call-next-handler}
%
\remarks%
The \specopref{call-next-handler} special form calls the next enclosing
handler. It is an error to evaluate this form other than within an established
handler function. The \specopref{call-next-handler} special form is normally
used when a handler function does not know how to deal with the class of
condition. However, it may also be used to combine handler function behaviour in
a similar but orthogonal way to \specopref{call-next-method} (assuming a
generic handler function).

\specop{with-handler}
%
\Syntax
\label{with-handler-syntax}
\defSyntax{with-handler}{
\begin{syntax}
    \scdef{with-handler-form}: \\
    \>  ( \specopref{with-handler} \scref{handler-function} \\
    \>\>  \scseqref{form} ) \\
    \scdef{handler-function}: \\
    \>  \scref{level-0-form}
\end{syntax}}%
\showSyntaxBox{with-handler}
%
\begin{arguments}
    \item[handler-function] The result of evaluating the handler function
    expression must be either a function or a generic function.  This function
    will be called if a condition is signalled during the dynamic extent of {\em
        protected-form}s and there are no intervening handler functions which
    accept or resume the condition.  A handler function takes two arguments: a
    condition, and a {\em resume} function/continuation.  The condition is the
    condition object that was passed to \functionref{signal} as its first
    argument.  The {\em resume} continuation is the continuation (or \nil{}) that
    was given to \functionref{signal} as its second argument.

    \item[forms] The sequence of forms whose execution is protected by the {\em
        handler function} specified above.
\end{arguments}
%
\result%
The value of the last form in the sequence of {\em form}s.
%
\remarks%
%
A \specopref{with-handler} form is evaluated in three steps:
\begin{enumerate}
    \item The new {\em handler-function} is evaluated. This now identifies the
    nearest enclosing handler and shadows the previous nearest enclosing
    handler.

    \item The sequence of {\em form}s is evaluated in order and the value of the
    last one is returned as the result of the \specopref{with-handler}
    expression.

    \item the {\em handler-function} is disestablished as the nearest enclosing
    handler, and the previous handler function is restored as the nearest
    enclosing.
\end{enumerate}
%
The above is the normal behaviour of \specopref{with-handler}.  The exceptional
behaviour of \specopref{with-handler} happens when there is a call to
\functionref{signal} during the evaluation of {\em protected-form}.
\functionref{signal} calls the nearest closing {\em handler-function} passing on
the first two arguments given to \functionref{signal}.  The {\em
    handler-function} is executed in the dynamic extent of the call to
\functionref{signal}.  However, any \functionref{signal}s occurring during the
execution of {\em handler-function} are dealt with by the nearest enclosing
handler outside the extent of the form which established {\em handler-function}.
It is an error if there is no enclosing handler. In this circumstance the
identified error is delivered to the configuration to be dealt with in an
implementation-defined way. Errors arising in the dynamic extent of the handler
function are signalled in the dynamic extent of the original
\functionref{signal} but are handled in the enclosing dynamic extent of the
handler.

\examples
%
There are three ways in which a {\em handler-function} can respond:
actions:
\begin{enumerate}
    \item The error is very serious and the computation must be abandoned; this
    is likely to be characterised by a non-local exit from the handler
    function.

    \item The situation can be corrected by the handler, so it does and then
    returns. Thus the call to \functionref{signal} returns with the result
    passed back from the handler function.

    \item The handler function does not know how to deal with the class of
    condition signalled; control is passed explicitly to the next enclosing
    handler via the \specopref{call-next-handler} special form.
\end{enumerate}
%
An illustration of the use of all three cases is given here:
%
\begin{example}
\label{example:with-handler}
\examplecaption{handler actions}
{\codeExample
(defgeneric error-handler (condition)
   method: (((c <serious-error>))
            ... abort thread ...)
   method: (((c <fixable-situation>))
            ... apply fix and return ... )
   method: (((c <condition>) (call-next-handler))))

(with-handler error-handler
      ;; the protected expression
      (something-which-might-signal-an-error))
\endCodeExample}
\end{example}
%
\seealso%
\functionref{signal}.

\function{error}
%
\begin{arguments}
    \item[error-message] a string containing relevant information.

    \item[condition-class] the class of condition to be signalled.

    \item[init-option\/$^*$] a sequence of options to be passed to {\tt
        initialize-instance} when making the instance of condition.
\end{arguments}
%
\result%
The result is \nil{}.
%
\remarks%
The \functionref{error} function signals a non-continuable error.  It calls
\functionref{signal} with an instance of a condition of {\em condition-class}
initialized from {\em init-option}\/s, the {\em error-message} and a {\em
    resume} continuation value of \nil{}, signifying that the condition was not
signalled continuably.

\function{cerror}
%
\begin{arguments}
    \item[error-message] a string containing relevant information.

    \item[condition-class] the class of condition to be signalled.

    \item[init-option\/$^*$] a sequence of options to be passed to {\tt
        initialize-instance} when making the instance of condition.
\end{arguments}
%
\result%
The result is \nil{}.
%
\remarks%
The \functionref{cerror} function signals a continuable error.  It calls
\functionref{signal} with an instance of a condition of {\em condition-class}
initialized from {\em init-option}\/s, the {\em error-message} and a {\em
    resume} continuation value which is the continuation of the
\functionref{cerror} expression.  A non-\nil{}\/ resume continuation signifies
that the condition has been signalled continuably.
%
\end{optDefinition}
