\defModule{vector}{Vectors}
%
\begin{optDefinition}
The defined name of this module is {\tt vector}.  See also
section~\ref{collection} (collections) for further operations on
vectors.

\syntaxform{vector}
%
A vector\index{general}{external representation!vector} is written as
\verb+#(+{\em obj}$_1$ \ldots {\em obj}$_n$\verb+)+.  For example:
\verb+#(1 2 3)+ is a vector of three elements, the integers {\tt 1}, {\tt 2} and
{\tt 3}.  The representations of {\em obj$_i$} are determined by the external
representations defined in other sections of this definition (see
\classref{character}\ (\ref{character}), \classref{int}\
(\ref{fpi}), \classref{float}\ (\ref{float}), \classref{list}\ (\ref{list}),
\classref{string}\ (\ref{string}) and \classref{symbol}\ (\ref{symbol}), as well
as instances of \classref{vector}\ itself.  The syntax for the external
representation of vectors is defined below.
%
\Syntax
\label{vector-syntax}
\defSyntax{vector}{
\begin{syntax}
    \scdef{vector}: \\
    \>  \#( \scseqref{object} )
\end{syntax}}%
\showSyntaxBox{vector}

\derivedclass{vector}{sequence}
%
The class of all instances of \classref{vector}.
%
\begin{initoptions}
    \item[size:, \classref{int}] The number of elements in
    the vector.  Vectors are zero-based and thus the maximum index is {\em
        size-1}.  If not supplied the {\em size\/} is zero.

    \item[fill-value:, \classref{object}] An object with which to initialize the
    vector.  The default fill value is \nil{}.
\end{initoptions}
%
\examples
\begin{tabular}{lcl}
    \verb|(make <vector>)| &\Ra& \verb|#()|\\
    \verb|(make <vector> size: 2)| &\Ra& \verb|#(() ())|\\
    \verb|(make <vector> size: 3| &\Ra& \verb|#(#\a #\a #\a)|\\
    \verb|  fill-value: #\a)|&&\\
\end{tabular}

\function{vector?}
%
\begin{arguments}
    \item[object] Object to examine.
\end{arguments}
%
\result%
Returns {\em object\/} if it is a vector, otherwise \nil{}.

\constant{maximum-vector-index}{integer}
%
\remarks%
This is an implementation-defined constant.  A conforming processor must support
a maximum vector index of at least
32767\index{general}{implementation-defined!maximum vector
    index}\index{general}{conformity-clause!maximum vector index}.

\method{binary=}{vector}
%
\begin{specargs}
    \item[vector$_1$, \classref{vector}] A vector.
    \item[vector$_2$, \classref{vector}] A vector.
\end{specargs}
%
\result%
If the size of {\em vector$_1$} is the same (under {\tt =}) as that of {\em
    vector$_2$}, and the result of the conjunction of the pairwise application
of \genericref{binary=} to the elements of the arguments \true{} the result is
{\em vector$_1$}.  If not the result is \nil{}.

\method{deep-copy}{vector}
%
\begin{specargs}
    \item[vector, \classref{vector}] A vector.
\end{specargs}
%
\result%
Constructs and returns a copy of {\em vector}, in which each element is the
result of calling {\em deep-copy\/} on the corresponding element of {\em
    vector}.

\method{shallow-copy}{vector}
%
\begin{specargs}
    \item[vector, \classref{vector}] A vector.
\end{specargs}
%
\result%
Constructs and returns a copy of {\em vector\/} in which each element is
\functionref{eql} to the corresponding element in {\em vector}.

\method{generic-print}{vector}
%
\begin{specargs}
    \item[vector, \classref{vector}]
    A vector to be ouptut on stream.
    \item[stream, \classref{stream}]
    A stream on which the representation is to be output.
\end{specargs}
%
\result%
The vector supplied as the first argument.
%
\remarks%
Output the external representation of {\em vector\/} on {\em stream\/} as
described in the introduction to this section.  Calls the generic function again
to produce the external representation of the elements stored in the vector.

\method{generic-write}{vector}
%
\begin{specargs}
    \item[vector, \classref{vector}] A vector to be ouptut on stream.
    \item[stream, \classref{stream}] A stream on which the representation is to
    be output.
\end{specargs}
%
\remarks%
Output the external representation of {\em vector\/} on {\em stream\/} as
described in the introduction to this section.  Calls the generic function again
to produce the external representation of the elements stored in the vector.
%
\end{optDefinition}
