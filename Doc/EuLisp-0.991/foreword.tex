\begin{foreword}
\label{history}
\begin{optDefinition}
The \eulisp\ group first met in September 1985 at IRCAM in Paris to discuss the
idea of a new dialect of Lisp, which should be less constrained by the past than
Common Lisp and less minimalist than Scheme.  Subsequent meetings formulated the
view of \eulisp\ that was presented at the 1986 ACM Conference on Lisp and
Functional Programming held at MIT, Cambridge, Massachusetts \bref{desiderata}
and at the European Conference on Artificial Intelligence (ECAI-86) held in
Brighton, Sussex \bref{stoyan}.  Since then, progress has not been steady, but
happening as various people had sufficient time and energy to develop part of
the language.  Consequently, although the vision of the language has in the most
part been shared over this period, only certain parts were turned into physical
descriptions and implementations.  For a nine month period starting in January
1989, through the support of INRIA, it became possible to start writing the
\eulisp\ definition.  Since then, affairs have returned to their previous state,
but with the evolution of the implementations of \eulisp\ and the background of
the foundations laid by the INRIA-supported work, there is convergence to a
consistent and practical definition.
\end{optDefinition}

\begin{optDefinition}
The acknowledgments for this definition fall into three categories:
intellectual, personal, and financial.

The ancestors of \eulisp\ (in alphabetical order) are
\cl\index{general}{cl@\cl}~\bref{CLtl},
\interlisp\index{general}{interlisp@\interlisp}~\bref{interlisp},
\lelisp\index{general}{lelisp@\lelisp}~\bref{le-lisp},
\lisp/VM\index{general}{lispVM@\lisp/VM}~\bref{lisp/vm}, Scheme~\bref{scheme-3},
and T\index{general}{T}~\bref{t-manual} \bref{t-book}.  Thus, the authors of
this report are pleased to acknowledge both the authors of the manuals and
definitions of the above languages and the many who have dissected and extended
those languages in individual papers.  The various papers on Standard
ML\index{general}{Standard ML}~\bref{std-ml} and the draft report on
Haskell\index{general}{Haskell}~\bref{haskell} have also provided much useful
input.

The writing of this report has, at various stages, been supported by Bull S.A.,
Gesellschaft f\"ur Mathematik und Datenverarbeitung (GMD, Sankt Augustin), Ecole
Polytechnique (LIX), ILOG S.A., Institut National de Recherche en Informatique
et en Automatique (INRIA), University of Bath, and Universit\'e Paris VI (LITP).
The authors gratefully acknowledge this support.  Many people from European
Community countries have attended and contributed to \eulisp\ meetings since
they started, and the authors would like to thank all those who have helped in
the development of the language.

In the beginning, the work of the \eulisp\ group was supported by the
institutions or companies where the participants worked, but in 1987 DG XIII
(Information technology directorate) of the Commission of the European
Communities agreed to support the continuation of the working group by funding
meetings and providing places to meet.  It can honestly be said that without
this support \eulisp\ would not have reached its present state.  In addition,
the \eulisp\ group is grateful for the support of:

British Council in France (Alliance programme),
British Council in Spain (Acciones Integradas programme),
British Council in Germany (Academic Research Collaboration programme),
British Standards Institute,
Deutscher Akademischer Austauschdienst (DAAD),
Departament de Llenguatges i Sistemes Inform\`atics (LSI, Universitat
Polit\`ecnica de Catalunya),
Fraunhofer Gesellschaft Institut f\"ur Software und Systemtechnik,
Gesellschaft f\"ur Mathematik und Datenverarbeitung (GMD),
ILOG S.A.,
Insiders GmbH,
Institut National de Recherche en Informatique et en Automatique (INRIA),
Institut de Recherche et de Coordination Acoustique Musique (IRCAM),
Ministerio de Educacion y Ciencia (MEC),
Rank Xerox France,
Science and Engineering Research Council (UK),
Siemens AG,
University of Bath,
University of Technology, Delft,
University of Edinburgh,
Universit\"at Erlangen
and
Universit\'e Paris VI (LITP).

The following people (in alphabetical order) have contributed in
various ways to the evolution of the language:
Giuseppe Attardi,
Javier B\'ejar,
Neil Berrington,
Russell Bradford,
Harry Bretthauer,
Peter Broadbery,
Christopher Burdorf,
J\'er\^ome Chailloux,
Odile Chenetier,
Thomas Christaller,
Jeff Dalton,
Klaus D\"a{\ss}ler,
Harley Davis,
David DeRoure,
John Fitch,
Richard Gabriel,
Brigitte Glas,
Nicolas Graube,
Dieter Kolb,
J\"urgen Kopp,
Antonio Moreno,
Eugen Neidl,
Greg Nuyens,
Pierre Parquier,
Keith Playford,
Willem van der Poel,
Christian Queinnec,
Nitsan Seniak,
Enric Sesa,
Herbert Stoyan,
and
Richard Tobin.

The editors of the \eulisp\ definition wish particularly to acknowledge the work
of Harley Davis on the first versions of the description of the object system.
The second version was largely the work of Harry Bretthauer, with the assistance
of J\"urgen Kopp, Harley Davis and Keith Playford.

Julian Padget ({\bf \tt jap@maths.bath.ac.uk})\\
School of Mathematical Sciences\\
University of Bath\\
Bath, Avon, BA2 7AY, UK\\

\noindent
Harry Bretthauer ({\bf \tt harry.bretthauer@gmd.de})
GMD mbH\\
Postfach 1316\\
53737 Sankt Augustin\\
Germany\\

\noindent
editors.
\end{optDefinition}
\end{foreword}
