%%% ----------------------------------------------------------------------------
%%% Streams
\defModule{stream}{Streams}
%
\begin{optDefinition}
The defined name of this module is {\tt stream}.

The aim of the stream design presented here is an open architecture for
programming with streams, which should be applicable when the interface to some
object can be characterized by either serial access to, or delivery of,
objects.

The two specific objectives are: (i) transfer of objects between a process and
disk storage; (ii) transfer of objects between one process and another.

The fundamental purpose of a stream object in the scheme presented here is to
provide an interface between two objects through the two functions
\functionref{read}, for streams from which objects are received, and
\functionref{write}, for streams to which objects are sent.

%%%  ---------------------------------------------------------------------------
%%%  Stream classes
\ssclause{Stream classes}

% ------------------------------------------------------------------------------
\derivedclass{stream}{object}
% ------------------------------------------------------------------------------
%
This is the root of the stream class hierarchy and also defines the basic
stream class.
%
\begin{initoptions}
    \item[read-action, \classref{function}] A function which is called by the
    \classref{stream} \methodref{generic-read}{stream} method. The accessor for
    this slot is called {\tt stream-read-action}.
    %
    \item[write-action, \classref{function}] A function which is called by the
    \classref{stream} \methodref{generic-write}{stream} method. The accessor for
    this slot is called {\tt stream-write-action}.
\end{initoptions}
%
The following accessor functions are defined for \classref{stream}
%
\begin{functions}
    \item[stream-lock] A lock, to be used to allow exclusive access to a stream.
    %
    \item[stream-source] An object to which the stream is connected and from
    which input is read.
    %
    \item[stream-sink] An object to which the stream is connected and to which
    ouptut is written.
    %
    \item[stream-buffer] An object which is used to buffer data by some
    subclasses of \classref{stream}. Its default value is \nil{}.
    %
    \item[stream-buffer-size] The maximum number of objects that can be stored
    in {\em stream-buffer}. Its default value is 0.
\end{functions}
%
\noindent
The transaction unit of \classref{stream} is \classref{object}.

% ------------------------------------------------------------------------------
\function{stream?}
% ------------------------------------------------------------------------------
\begin{arguments}
  \item[object, \classref{object}] The object to be examined.
\end{arguments}
%
\result%
Returns {\em object\/} if it is a stream, otherwise \nil{}.

% ------------------------------------------------------------------------------
\function{from-stream}
% ------------------------------------------------------------------------------
A constructor function of one argument for \classref{stream} which
returns a stream whose {\tt stream-read-action} is the given argument.

% ------------------------------------------------------------------------------
\function{to-stream}
% ------------------------------------------------------------------------------
A constructor function of one argument for \classref{stream} which returns a
stream whose {\tt stream-write-action} is the given argument.

% ------------------------------------------------------------------------------
\derivedclass{buffered-stream}{stream}
% ------------------------------------------------------------------------------
%
This class specializes \classref{stream} by the use of a buffer which may grow
arbitrarily large. The transaction unit of \classref{buffered-stream} is
\classref{object}.

% ------------------------------------------------------------------------------
\derivedclass{fixed-buffered-stream}{buffered-stream}
% ------------------------------------------------------------------------------
%
This class specializes \classref{buffered-stream} by placing a bound on the
growth of the buffer. The transaction unit of \classref{fixed-buffered-stream}
is \classref{object}.

% ------------------------------------------------------------------------------
\derivedclass{file-stream}{fixed-buffered-stream}
% ------------------------------------------------------------------------------
%
This class specializes \classref{fixed-buffered-stream} by providing an
interface to data stored in files on disk. The transaction unit of
\classref{file-stream} is \classref{character}. The following additional
accessor functions are defined for \classref{file-stream}:
%
\begin{functions}
    \item[file-stream-filename] The path identifying the file system object
    associated with the stream.
    %
    \item[file-stream-mode] The mode of the connection between the stream and
    the file system object (usually either read or write).
    %
    \item[file-stream-buffer-position] A key identifying the current position in
    the stream's buffer.
\end{functions}

% ------------------------------------------------------------------------------
\function{file-stream?}
% ------------------------------------------------------------------------------
\begin{arguments}
  \item[object, \classref{object}] The object to be examined.
\end{arguments}
%
\result%
Returns {\em object\/} if it is a \classref{file-stream} otherwise \nil{}.

% ------------------------------------------------------------------------------
\derivedclass{string-stream}{buffered-stream}
% ------------------------------------------------------------------------------
%
The class of the default string stream.

% ------------------------------------------------------------------------------
\function{string-stream?}
% ------------------------------------------------------------------------------
\begin{arguments}
    \item[object, \classref{object}] The object to be examined.
\end{arguments}
%
\result%
Returns {\em object\/} if it is a \classref{string-stream} otherwise \nil{}.

%%%  ---------------------------------------------------------------------------
%%%  Stream operators
\ssclause{Stream operators}

% ------------------------------------------------------------------------------
\function{connect}
% ------------------------------------------------------------------------------
\begin{arguments}
    \item[source] The source object from which the stream will read data.
    \item[sink] The sink object to which the stream will write data.
    \item[\optional{options}] An optional argument for specifying
    implementation-defined options.
\end{arguments}
%
\result%
The return value is \nil{}.
%
\remarks%
Connects {\em source\/} to {\em sink\/} according to the class-specific
behaviours of \genericref{generic-connect}.

% ------------------------------------------------------------------------------
\generic{generic-connect}
% ------------------------------------------------------------------------------
\begin{genericargs}
    \item[source, \classref{object}] The source object from which the stream
    will read data.
    \item[sink, \classref{object}] The sink object to which the stream will
    write data.
    \item[\optional{options}, \classref{list}] An optional argument for
    specifying implementation-defined options.
\end{genericargs}
%
\remarks%
Generic form of \functionref{connect}.

% ------------------------------------------------------------------------------
\method{generic-connect}{stream}
% ------------------------------------------------------------------------------
\begin{specargs}
    \item[source, \classref{stream}] The stream which is to be the source of
    {\em sink}.
    \item[sink, \classref{stream}] The stream which is to be the sink of {\em
        source}.
    \item[options, \classref{list}] A list of implementation-defined options.
\end{specargs}
%
\result%
The return value is \nil{}.
%
\remarks%
Connects the source of {\em sink\/} to {\em source\/} and the sink of
{\em source\/} to {\em sink}.

% ------------------------------------------------------------------------------
\method{generic-connect}{path}
% ------------------------------------------------------------------------------
\begin{specargs}
    \item[source, \theclass{path}] A path name.
    \item[sink, \classref{file-stream}] The stream via which data will be
    received from the file named by \scref{path}.
    \item[options, \classref{list}] A list of implementation-defined options.
\end{specargs}
%
\result%
The return value is \nil{}.
%
\remarks%
Opens the object identified by the path {\em source\/} for reading and
connects {\em sink\/} to it. Hereafter, {\em sink\/} may be used for reading
data from {\em sink\/}, until {\em sink\/} is disconnected or
reconnected. Implementation-defined options for the opening of files may be
specified using the third argument.
%
\seealso%
\functionref{open-input-file}.

% ------------------------------------------------------------------------------
\method{generic-connect}{file-stream}
% ------------------------------------------------------------------------------
\begin{specargs}
    \item[source, \classref{file-stream}] The stream via which data will be sent
    to the file named by \scref{path}.
    \item[sink, \theclass{path}] A path name.
    \item[options, \classref{list}] A list of implementation-defined options.
\end{specargs}
%
\result%
The return value is \nil{}.
%
\remarks%
Opens the object identifed by the path {\em sink\/} for writing and
connects {\em source\/} to it. Hereafter, {\em source\/} may be used for writing
data to {\em sink}, until {\em source\/} is disconnected or
reconnected. Implementation-defined options for the opening of files may be
specified using the third argument.
%
\seealso%
\functionref{open-output-file}.

% ------------------------------------------------------------------------------
\generic{reconnect}
% ------------------------------------------------------------------------------
\begin{genericargs}
    \item[s1, \classref{stream}] A stream.
    \item[s2, \classref{stream}] A stream.
\end{genericargs}
%
\result%
The return value is \nil{}.
%
\remarks%
Transfers the source and sink connections of {\em s1\/} to {\em s2},
leaving {\em s1\/} disconnected.

% ------------------------------------------------------------------------------
\method{reconnect}{stream}
% ------------------------------------------------------------------------------
\begin{specargs}
    \item[s1, \classref{stream}] A stream.
    \item[s2, \classref{stream}] A stream.
\end{specargs}
%
\result%
The return value is \nil{}.
%
\remarks%
Implements the \genericref{reconnect} operation for objects of class
\classref{stream}.

% ------------------------------------------------------------------------------
\generic{disconnect}
% ------------------------------------------------------------------------------
\begin{genericargs}
    \item[s, \classref{stream}] A stream.
\end{genericargs}
%
\result%
The return value is \nil{}.
%
\remarks%
Disconnects the stream {\em s\/} from its source and/or its sink.

% ------------------------------------------------------------------------------
\method{disconnect}{stream}
% ------------------------------------------------------------------------------
\begin{specargs}
    \item[s, \classref{stream}] A stream.
\end{specargs}
%
\result%
The return value is \nil{}.
%
\remarks%
Implements the diconnect operation for objects of class \classref{stream}.

% ------------------------------------------------------------------------------
\method{disconnect}{file-stream}
% ------------------------------------------------------------------------------
\begin{specargs}
    \item[s, \classref{file-stream}] A file stream.
\end{specargs}
%
\result%
The return value is \nil{}.
%
\remarks%
Implements the diconnect operation for objects of class
\classref{file-stream}. In particular, this involves closing the file associated
with the stream {\em s}.

%%%  ---------------------------------------------------------------------------
%%%  Stream objects
\ssclause{Stream objects}

% ------------------------------------------------------------------------------
\instance{stdin}{file-stream}
% ------------------------------------------------------------------------------
\remarks%
The standard input stream, which is a file-stream and whose transaction unit is
therefore character. In Posix compliant configurations, this object is
initialized from the Posix \instanceref{stdin} object. Note that although
\instanceref{stdin} itself is a constant binding, it may be connected to
different files by the \genericref{reconnect} operation.

% ------------------------------------------------------------------------------
\instance{lispin}{stream}
% ------------------------------------------------------------------------------
\remarks%
The standard lisp input stream, and its transaction unit is object. This stream
is initially connected to \instanceref{stdin} (although not necessarily
directly), thus a \functionref{read} operation on \instanceref{lispin} will case
characters to be read from \instanceref{stdin} and construct and return an
object corresponding to the next lisp expression. Note that although
\instanceref{lispin} itself is a constant binding, it may be connected to
different source streams by the \genericref{reconnect} operation.

% ------------------------------------------------------------------------------
\instance{stdout}{file-stream}
% ------------------------------------------------------------------------------
\remarks%
The standard output stream, which is a file-stream and whose transaction unit is
therefore character. In Posix compliant configurations, this object is
initialized from the Posix \instanceref{stdout} object. Note that although
\instanceref{stdout} itself is a constant binding, it may be connected to
different files by the \genericref{reconnect} operation.

% ------------------------------------------------------------------------------
\instance{stderr}{file-stream}
% ------------------------------------------------------------------------------
\remarks%
The standard error stream, which is a file-stream and whose transaction unit is
therefore character. In Posix compliant configurations, this object is
initialized from the Posix \instanceref{stderr} object. Note that although
\instanceref{stderr} itself is a constant binding, it may be connected to
different files by the \genericref{reconnect} operation.

%%%  ---------------------------------------------------------------------------
%%%  Buffer management
\ssclause{Buffer management}
\label{Buffer-management}

% ------------------------------------------------------------------------------
\generic{fill-buffer}
% ------------------------------------------------------------------------------
\begin{genericargs}
    \item[stream, \classref{buffered-stream}] A stream.
\end{genericargs}
%
\result%
The buffer associated with {\em stream\/} is refilled from its {\em source\/}.
Returns a count of the number of items read.
%
\remarks%
This function is guaranteed to be called when an attempt is made to read from a
buffered stream whose buffer is either empty, or from which all the items have
been read.

% ------------------------------------------------------------------------------
\method{fill-buffer}{buffered-stream}
% ------------------------------------------------------------------------------
\begin{specargs}
    \item[stream, \classref{buffered-stream}] A stream.
\end{specargs}

% ------------------------------------------------------------------------------
\method{fill-buffer}{file-stream}
% ------------------------------------------------------------------------------
\begin{specargs}
    \item[stream, \classref{file-stream}] A stream.
\end{specargs}

% ------------------------------------------------------------------------------
\generic{flush-buffer}
% ------------------------------------------------------------------------------
\begin{genericargs}
    \item[stream, \classref{buffered-stream}] A stream.
\end{genericargs}
%
\result%
The contents of the buffer associated with {\em stream} is flushed to
its sink. If this operation succeeds, a \true{} value is returned, otherwise the
result is \nil{}.
%
\remarks%
This function is guaranteed to be called when an attempt is made to
write to a buffered stream whose buffer is full.

% ------------------------------------------------------------------------------
\method{flush-buffer}{buffered-stream}
% ------------------------------------------------------------------------------
\begin{specargs}
    \item[stream, \classref{buffered-stream}] A stream.
\end{specargs}
%
\result%
The contents of the buffer associated with {\em stream} is flushed to
its sink. If this operation succeeds, a \true{} value is returned, otherwise the
result is \nil{}.
%
\remarks%
Implements the \genericref{flush-buffer} operation for objects of class
\classref{buffered-stream}.

% ------------------------------------------------------------------------------
\method{flush-buffer}{file-stream}
% ------------------------------------------------------------------------------
\begin{specargs}
    \item[stream, \classref{file-stream}] A stream.
\end{specargs}
%
\result%
The contents of the buffer associated with {\em stream} is flushed to
its sink. If this operation succeeds, a \true{} value is returned, otherwise the
result is \nil{}.
%
\remarks%
Implements the \genericref{flush-buffer} operation for objects of
\classref{file-stream}. This method is called both when the buffer is full and
after a newline character is written to the buffer.

% ------------------------------------------------------------------------------
\condition{end-of-stream}{stream-condition}
% ------------------------------------------------------------------------------
\begin{initoptions}
    \item[stream, \classref{stream}] A stream.
\end{initoptions}
%
\remarks%
Signalled by the default end of stream action, as a consequence of a
read operation on {\em stream\/}, when it is at end of stream.
%
\seealso%
\genericref{generic-read}.

% ------------------------------------------------------------------------------
\generic{end-of-stream}
% ------------------------------------------------------------------------------
\begin{genericargs}
    \item[stream, \classref{buffered-stream}] A stream.
\end{genericargs}
%
\remarks%
This function is guaranteed to be called when a read operation
encounters the end of {\em stream\/} and the {\tt eos-error?} argument to read
has a non-\nil{}\/ value.

% ------------------------------------------------------------------------------
\method{end-of-stream}{buffered-stream}
% ------------------------------------------------------------------------------
\begin{specargs}
    \item[stream, \classref{buffered-stream}] A stream.
\end{specargs}
%
\remarks%
Signals the end of stream condition.

% ------------------------------------------------------------------------------
\method{end-of-stream}{file-stream}
% ------------------------------------------------------------------------------
\begin{specargs}
    \item[stream, \classref{file-stream}] A stream.
\end{specargs}
%
\remarks%
Disconnects {\em stream\/} and signals the end of stream condition.

%%%  ---------------------------------------------------------------------------
%%%  Reading from streams
\ssclause{Reading from streams}

% ------------------------------------------------------------------------------
\condition{read-error}{condition}
% ------------------------------------------------------------------------------
%
\begin{genericargs}
    \item[stream, \classref{stream}] A stream.
\end{genericargs}
%
\remarks%
Signalled by a \functionref{read} operation which fails in some manner other
than when it is at end of stream.

% ------------------------------------------------------------------------------
\function{read}
% ------------------------------------------------------------------------------
\begin{arguments}
    \item[\optional{stream}] A stream.
    \item[\optional{eos-error?}] A boolean.
    \item[\optional{eos-value}] Value to be returned to indicate end of stream.
\end{arguments}
%
\result%
That of calling \genericref{generic-read} with the arguments supplied or
defaulted as described.
%
\remarks%
The {\em stream\/} defaults to \instanceref{lispin}, {\em
    eos-error?\/} defaults to \nil{}\/ and {\em eos-value\/} defaults to {\tt
    eos-default-value}.

% ------------------------------------------------------------------------------
\generic{generic-read}
% ------------------------------------------------------------------------------
\begin{genericargs}
    \item[stream, \classref{stream}] A stream.
    \item[eos-error?, \classref{object}] A boolean.
    \item[eos-value, \classref{object}] Value to be returned to indicate end of
    stream.
\end{genericargs}
%
\result%
The next transaction unit from {\em stream}.
%
\remarks%
If the end of {\em stream\/} is encountered and the value of {\em eos-error?}
is \nil{}, the result is {\em eos-value\/}. If the end of {\em stream} is
encountered and the value of {\tt eos-error?} is non-\nil{}, the function
\methodref{end-of-stream}{stream} is called with the argument {\em stream}.

% ------------------------------------------------------------------------------
\method{generic-read}{stream}
% ------------------------------------------------------------------------------
\begin{specargs}
    \item[stream, \classref{stream}] A stream.
    \item[eos-error?, \classref{object}] A boolean.
    \item[eos-value, \classref{object}] Value to be returned to indicate end of
    stream.
\end{specargs}
%
\result%
That of calling the {\em read-action\/} of stream with the arguments {\em
    stream\/}, {\em eos-error?\/} and {\em eos-value}.  Returns \true.
%
\remarks%
Implements the \genericref{generic-read} operation for objects of class
\classref{stream}.

% ------------------------------------------------------------------------------
\method{generic-read}{buffered-stream}
% ------------------------------------------------------------------------------
\begin{specargs}
    \item[stream, \classref{buffered-stream}] A buffered stream.
    \item[eos-error?, \classref{object}] A boolean.
    \item[eos-value, \classref{object}] Value to be returned to indicate end of
    stream.
\end{specargs}
%
\result%
The next object stored in the stream buffer.  If the buffer is empty,
the function \genericref{fill-buffer} is called. If the refilling operation did
not succeed, the end of stream action is carried out as described under
\genericref{generic-read}.  Returns \true.
%
\remarks%
Implements the \genericref{generic-read} operation for objects of class
\classref{buffered-stream}.

% ------------------------------------------------------------------------------
\method{generic-read}{file-stream}
% ------------------------------------------------------------------------------
\begin{specargs}
    \item[stream, \classref{file-stream}] A file stream.
    \item[eos-error?, \classref{object}] A boolean.
    \item[eos-value, \classref{object}] Value to be returned to indicate end of
    stream.
\end{specargs}
%
\result%
The next object stored in the stream buffer.  If the buffer is empty,
the function \genericref{fill-buffer} is called. If the refilling operation did
not succeed, the end of stream action is carried out as described under
\genericref{generic-read}.  Returns \true.
%
\remarks%
Implements the \genericref{generic-read} operation for objects of class
\classref{file-stream}.

%%%  ---------------------------------------------------------------------------
%%%  Writing to streams
\ssclause{Writing to streams}

% ------------------------------------------------------------------------------
\generic{generic-write}
% ------------------------------------------------------------------------------
\begin{genericargs}
    \item[object, \classref{object}] An object to be written to {\em stream}.
    \item[stream, \classref{stream}] Stream to which {\em object\/} is to be
    written.
\end{genericargs}
%
\result%
Returns {\em object}.
%
\remarks%
Outputs the external representation of {\em object\/} on the output stream
{\em stream}.
%
\seealso%
The following \genericref{generic-write} methods are defined:
\methodref{generic-write}{character}, \methodref{generic-write}{symbol},
\methodref{generic-write}{keyword},
\methodref{generic-write}{fpi},
\methodref{generic-write}{double-float}, \methodref{generic-write}{null},
\methodref{generic-write}{cons}, \methodref{generic-write}{list},
\methodref{generic-write}{string}, \methodref{generic-write}{vector},
\methodref{generic-write}{stream}, \methodref{generic-write}{buffered-stream}
and \methodref{generic-write}{file-stream}.

% ------------------------------------------------------------------------------
\method{generic-write}{stream}
% ------------------------------------------------------------------------------
\begin{specargs}
    \item[object, \classref{object}] An object to be written to {\em stream}.
    \item[stream, \classref{stream}] Stream to which {\em object\/} is to be
    written.
\end{specargs}

% ------------------------------------------------------------------------------
\method{generic-write}{buffered-stream}
% ------------------------------------------------------------------------------
\begin{specargs}
    \item[object, \classref{object}] An object to be written to {\em stream}.
    \item[stream, \classref{buffered-stream}] Stream to which {\em object\/} is
    to be written.
\end{specargs}

% ------------------------------------------------------------------------------
\method{generic-write}{file-stream}
% ------------------------------------------------------------------------------
\begin{specargs}
    \item[object, \classref{object}] An object to be written to {\em stream}.
    \item[stream, \classref{file-stream}] Stream to which {\em object\/} is
    to be written.
\end{specargs}

% ------------------------------------------------------------------------------
\function{swrite}
% ------------------------------------------------------------------------------
\begin{arguments}
    \item[stream] Stream to which {\em object\/} is to be written.
    \item[object] An object to be written to {\em stream}.
\end{arguments}
%
\result%
Returns {\em stream}.
%
\remarks%
Outputs the external representation of {\em object\/} on the output
stream {\em stream\/} using \genericref{generic-write}.
%
\seealso%
\genericref{generic-write}.

% ------------------------------------------------------------------------------
\function{write}
% ------------------------------------------------------------------------------
\begin{arguments}
    \item[object] An object to be written to {\em stream}.
\end{arguments}
%
\result%
Returns \instanceref{stdout}.
%
\remarks%
Outputs the external representation of {\em object\/} on \instanceref{stdout}
using \genericref{generic-write}.
%
\seealso%
\functionref{swrite}, \genericref{generic-write}.

%%%  ---------------------------------------------------------------------------
%%%  Additional functions
\ssclause{Additional functions}

% ------------------------------------------------------------------------------
\function{read-line}
% ------------------------------------------------------------------------------
\begin{arguments}
    \item[stream] A stream.
    \item[\optional{eos-error?}] A boolean.
    \item[\optional{eos-value}] Value to be returned to indicate end of stream.
\end{arguments}
%
\result%
A string.
%
\remarks%
Reads a line (terminated by a newline character or the end of the
stream) from the stream of characters which is {\em stream}.  Returns
the line as a string, discarding the terminating newline, if any.  If
the stream is already at end of stream, then the stream action is
called: the default stream action is to signal an error: (condition
class: \conditionref{end-of-stream}\indexcondition{end-of-stream}).

% ------------------------------------------------------------------------------
\generic{generic-print}
% ------------------------------------------------------------------------------
\begin{genericargs}
    \item[object, \classref{object}] An object to be output on {\em stream}.
    \item[stream, \classref{stream}] A character stream on which {\em
        object\/} is to be output.
\end{genericargs}
%
\result%
Returns {\em object}.
%
\remarks%
Outputs the external representation of {\em object\/} on the output stream {\em
    stream\/}.
%
\seealso%
\functionref{prin}.  The following \genericref{generic-write} methods are
defined: \methodref{generic-write}{character},
\methodref{generic-write}{symbol}, \methodref{generic-write}{keyword},
\methodref{generic-write}{fpi},
\methodref{generic-write}{double-float}, \methodref{generic-write}{null},
\methodref{generic-write}{cons}, \methodref{generic-write}{list},
\methodref{generic-write}{string} and \methodref{generic-write}{vector}.

% ------------------------------------------------------------------------------
\function{sprint}
% ------------------------------------------------------------------------------
\begin{arguments}
    \item[stream] A character stream on which {\em object\/} is
    to be output.
    \item[{\optional{object$_1$ object$_2$ ...}}] A sequence of objects to be
    output on {\em stream}.
\end{arguments}
%
\result%
Returns {\em stream}.
%
\remarks%
Outputs the external representation of {\em object$_1$ object$_2$ ...} on the
output stream {\em stream\/} using \genericref{generic-print} for each object.
%
\seealso%
\genericref{generic-print}.

% ------------------------------------------------------------------------------
\function{print}
% ------------------------------------------------------------------------------
\begin{arguments}
    \item[{\optional{object$_1$ object$_2$ ...}}] A sequence of objects to be
    output on \instanceref{stdout}.
\end{arguments}
%
\result%
Returns \instanceref{stdout}.
%
\remarks%
Outputs the external representation of {\em object$_1$ object$_2$ ...} on the
output stream \instanceref{stdout} using \genericref{sprint} for each object.
%
\seealso%
\functionref{sprint} and \genericref{generic-print}.

% ------------------------------------------------------------------------------
\function{sflush}
% ------------------------------------------------------------------------------
\begin{arguments}
    \item[stream] A stream to flush.
\end{arguments}
%
\result%
Returns {\em stream}.
%
\remarks%
{\tt sflush} causes any buffered data for the stream to be written to
the stream. The stream remains open.

% ------------------------------------------------------------------------------
\function{flush}
% ------------------------------------------------------------------------------
%
\result%
Returns \instanceref{stdout}.
%
\remarks%
{\tt flush} causes any buffered data for \instanceref{stdout} to be written to
\instanceref{stdout}.
%
\seealso%
\functionref{sflush}.

% ------------------------------------------------------------------------------
\function{sprin-char}
% ------------------------------------------------------------------------------
\begin{arguments}
    \item[stream] A stream.
    \item[char] Character to be written to {\em stream}.
    \item[\optional{times}] Integer count.
\end{arguments}
%
\result%
Outputs {\em char} on {\em stream}.  The optional count {\em times\/} defaults
to 1.

% ------------------------------------------------------------------------------
\function{prin-char}
% ------------------------------------------------------------------------------
\begin{arguments}
    \item[char] Character to be written to \instanceref{stdout}.
    \item[\optional{times}] Integer count.
\end{arguments}
%
\result%
Outputs {\em char} on \instanceref{stdout}. The optional count {\em times\/}
defaults to 1.
%
\seealso%
\functionref{sprin-char}.

% ------------------------------------------------------------------------------
\function{sread}
% ------------------------------------------------------------------------------
\begin{arguments}
    \item[stream] A stream.
    \item[\optional{eos-error?}] A boolean.
    \item[\optional{eos-value}] Value to be returned to indicate end of stream.
\end{arguments}

%%% ---------------------------------------------------------------------------
%%%  Convenience forms
\ssclause{Convenience forms}

% ------------------------------------------------------------------------------
\function{open-input-file}
% ------------------------------------------------------------------------------
\begin{arguments}
    \item[path] A path identifying a file system object.
\end{arguments}
%
\result%
Allocates and returns a new \classref{file-stream} object whose source
is connected to the file system object identified by \scref{path}.

% ------------------------------------------------------------------------------
\function{open-output-file}
% ------------------------------------------------------------------------------
\begin{arguments}
    \item[path] A path identifying a file system object.
\end{arguments}
%
\result%
Allocates and returns a new \classref{file-stream} object whose sink is
connected to the file system object identified by \scref{path}.

% ------------------------------------------------------------------------------
\specop{with-input-file}
% ------------------------------------------------------------------------------
\Syntax
\defSyntax{with-input-file}{
\begin{syntax}
    \scdef{with-input-file-form}: \ra{} \classref{object} \\
    \>  ( \specopref{with-input-file} \scref{path} \\
    \>\>  \scref{body} ) \\
    \scdef{path}: \\
    \>  \scref{string}
\end{syntax}}%
\showSyntaxBox{with-input-file}

% ------------------------------------------------------------------------------
\specop{with-output-file}
% ------------------------------------------------------------------------------
\Syntax
\defSyntax{with-output-file}{
\begin{syntax}
    \scdef{with-output-file-form}: \ra{} \classref{object} \\
    \>  ( \specopref{with-output-file} \scref{path} \\
    \>\>  \scref{body} )
\end{syntax}}%
\showSyntaxBox{with-output-file}

% ------------------------------------------------------------------------------
\specop{with-source}
% ------------------------------------------------------------------------------
\Syntax
\defSyntax{with-source}{
\begin{syntax}
    \scdef{with-source-form}: \ra{} \classref{object} \\
    \>  ( \specopref{with-source} ( \scref{identifier} \scref{form} ) \\
    \>\>  \scref{body} )
\end{syntax}}%
\showSyntaxBox{with-source}

% ------------------------------------------------------------------------------
\specop{with-sink}
% ------------------------------------------------------------------------------
\Syntax
\defSyntax{with-sink}{
\begin{syntax}
    \scdef{with-sink-form}: \ra{} \classref{object} \\
    \>  ( \specopref{with-sink} ( \scref{identifier} \scref{form} ) \\
    \>\>  \scref{body} )
\end{syntax}}%
\showSyntaxBox{with-sink}

\end{optDefinition}
%
\begin{optPrivate}
%
%%%  ---------------------------------------------------------------------------
%%%  Posix bindings
\ssclause{Posix bindings}

% ------------------------------------------------------------------------------
\function{fopen}
% ------------------------------------------------------------------------------

Arguments

stream
A stream to connect to a file.
path
A path identifying a file system object.
mode
How the file is to be opened.

Connects stream to the file system object identified by path according to the
specified mode. The mode is specified by one of the following symbols:

r
Opens file for reading.
w
Opens file for writing, creating the file if it does not exist.
a
Opens file for append output, creating the file if it does not exist. If
the file exists, the stream position points to the end of the file, so any
output adds to the file.
r+
Opens the file for update. Both input and output can be performed on the
stream. Any existing data in the file is preserved.
w+
Opens the file for update. Both input and output can be performed on the
stream. Any existing data in the file is eliminated; the file is re-created
if it exists.
a+
Opens the file for update. The file can be read at any location, but any
output to the file occurs starting from its current end.

The result is stream.

% ------------------------------------------------------------------------------
\function{fclose}
% ------------------------------------------------------------------------------

Arguments

stream
A stream.

Closes the file stream stream and returns (). The stream is disconnected from
the file system object with which it was associated and may no longer be used
for read or write operations. The stream buffer is flushed to the file system
before the file is closed.

% ------------------------------------------------------------------------------
\function{fcntl}
% ------------------------------------------------------------------------------

Arguments

stream
A stream.
[cmds]
fcntl commands as defined below.

This function returns or sets information about an open file stream. The symbol
command determines the action of fcntl and in some cases, the third argument
command-values must also be supplied. The command can be any one of the
following symbols, which are a subset of those defined in 1990:??? (POSIX C
API):

F_GETFL
This is the default value. In this case, fcntl returns a list of symbols or
integers, in no particular order, describing the stream's attributes. These
values can be among the following:

O_APPEND
The file is open in append mode.
O_NONBLOCK
Waiting for data does not cause blocking.
O_RDONLY
The file is open for read only.
O_RDWR
The file is open for read and write.
O_WRONLY
The file is open for write only.
Other implementation-defined values specified as integers.

F_SETFL
In this case the the third argument command-values must be a list of
attributes from the set described above for F_GETFL. However, according to
1990:???, only the values O_APPEND and O_NONBLOCK can be modified.
\end{optPrivate}

%%% ----------------------------------------------------------------------------
