\sclause{Keywords}
\label{keyword}
\index{general}{keyword}
\index{general}{keyword!module}
\index{general}{level-0 modules!keyword}
%
\begin{optDefinition}
The defined name of this module is {\tt keyword}.
%
\syntaxform{keyword}
\index{general}{keyword}
\index{general}{keyword!definition of}
%
The syntax of keywords is very similar to that of identifiers and of symbols,
including all the escape conventions, but are distinguished by a colon {\tt (:)}
suffix:
%
\Syntax
\defSyntax{keyword}{
\begin{syntax}
    \scdef{keyword}: \\
    \>  \scref{identifier}:
\end{syntax}}%
\showSyntaxBox{keyword}%

It is an error to use a keyword where an identifier is expected, such as, for
example, in lambda parameter lists or in let binding forms.

{\em The matter of keywords appering in lambda parameter lists, for example,
    {\tt rest:}, instead of the dot notation, is currently an open issue.}

Operationally, the most important aspect of keywords is that each is unique, or,
stated the other way around: the result of processing every syntactic token
comprising the same sequence of characters which denote a keyword is the same
object. Or, more briefly, every keyword with the same name denotes the same
keyword. A consequence of this guarantee is that keywords may be compared using
\functionref{eq}.

\derivedclass{keyword}{name}
\index{general}{level-0 classes!\theclass{keyword}}
%
The class of all instance of \classref{keyword}.
%
\begin{initoptions}
    \item[string, string] The string containing the characters to be used to
    name the keyword. The default value for string is the empty string, thus
    resulting in the keyword with no name, written \verb+|:|+.
\end{initoptions}
%
{\em What is the defined behaviour if the last character of string is colon?}

\function{keywordp}
%
\begin{arguments}
    \item[object] Object to examine.
\end{arguments}
%
\result%
Returns {\em object\/} if it is a keyword.

\function{keyword-name}
%
\begin{arguments}
    \item[keyword] A keyword.
\end{arguments}
%
\result%
Returns a {\em string\/} which is \genericref{equal} to that given as the
argument to the call to \functionref{make} which created {\em keyword}. It is an
error to modify this string.

\function{keyword-exists-p}
%
\begin{arguments}
    \item[string] A string containing the characters to be used to determine the
    existence of a keyword with that name.
\end{arguments}
%
\result%
Returns the keyword whose name is {\em string\/} if that keyword has already
been constructed by \functionref{make}. Otherwise, returns \nil{}.

\method{generic-prin}
%
\begin{specargs}
    \item[keyword, \classref{keyword}] The keyword to be output on {\em stream}.
    %
    \item[stream, \classref{stream}] The stream on which the representation is to be
    output.
\end{specargs}
%
\result%
The keyword supplied as the first argument.
%
\remarks%
Outputs the external representation of {\em keyword\/} on {\em stream\/} as
described in the section on symbols, interpreting each of the characters in the
name.

\method{generic-write}
%
\begin{specargs}
    \item[keyword, \classref{keyword}] The keyword to be output on {\em stream}.
    %
    \item[stream, \classref{stream}] The stream on which the representation is to be
    output.
\end{specargs}
%
\result%
The keyword supplied as the first argument.
%
\remarks%
Outputs the external representation of {\em keyword\/} on {\em stream\/} as
described in the section on symbols. If any characters in the name would not
normally be legal constituents of a keyword, the output is preceded and
succeeded by multiple-escape characters.
%
\examples
\begin{tabular}{lcl}
    \verb|(write (make <keyword> 'string "abc"))| &\Ra& \verb+abc:+\\
    \verb|(write (make <keyword> 'string "a c"))| &\Ra& \verb+|a c:|+\\
    \verb|(write (make <keyword> 'string ").("))| &\Ra& \verb+|).(:|+\\
\end{tabular}

\converter{string}
%
\begin{specargs}
    \item[keyword, \classref{keyword}] A keyword to be converted to a string.
\end{specargs}
%
\result%
A string.
%
\remarks%
This function is the same as \functionref{keyword-name}.  It is defined for the
sake of symmetry.
%
\end{optDefinition}
