\sclause{Symbols}
\index{general}{symbol}
\index{general}{symbol!module}
\label{symbol}
\index{general}{level-0 modules!symbol}
%
\begin{optDefinition}
The defined name of this module is {\tt symbol}.
%
\syntaxform{symbol}
%
A \scref{symbol} is a literal \scref{identifier} and hence has the same syntax
\ref{identifier-syntax}:

\Syntax
\label{symbol-syntax}
\defSyntax{symbol}{
\begin{syntax}
    \scdef{symbol}: \\
    \>  \scref{identifier} \\
\end{syntax}}%
\showSyntaxBox{symbol}%

Because there are two escaping mechanisms and because the first character of a
token affects the interpretation of the remainder, there are many ways in which
to input the same \scref{identifier}.  If this same identifier is used as a
literal, \ie a \scref{symbol}, the results of processing each token denoting the
\scref{identifier} will be \functionref{eq} to one another.  For example, the
following tokens all denote the same \scref{symbol}:
%
\begin{center}
\verb+|123|+, \verb+\123+, \verb+|1|23+, \verb+||123+, \verb+||||123+
\end{center}
%
which will be output by the function \functionref{write} as \verb+|123|+.  If
output by \functionref{write}, the representation of the \scref{symbol} will
permit reconstruction by \functionref{read}---escape characters are
preserved---so that equivalence is maintained between \functionref{read} and
\functionref{write} for \scref{symbol}s.  For example: \verb+|a(b|+ and
\verb+abc.def+ are two \scref{symbol}s as output by \functionref{write} such
that \functionref{read} can read them as two \scref{symbol}s.  If output by
\functionref{prin}, the escapes necessary to re-\functionref{read} the
\scref{symbol} will not be included.  Thus, taking the same examples,
\functionref{prin} outputs \verb+a(b+ and \verb+abc.def+, which
\functionref{read} interprets as the \scref{symbol} \verb+a+ followed by the
start of a list, the \scref{symbol} \verb+b+ and the \scref{symbol}
\verb+abc.def+.

Computationally, the most important aspect of \scref{symbol}s is that each is
unique, or, stated the other way around: the result of processing
every syntactic token comprising the same sequence of characters which
denote an identifier is the same object.  Or, more briefly, every
identifier with the same name denotes the same \scref{symbol}.

\derivedclass{symbol}{name}
\index{general}{level-0 classes!\theclass{symbol}}
%
The class of all instances of \classref{symbol}.
%
\begin{initoptions}
    \item[string, string] The string containing the characters to be used to
    name the symbol.  The default value for string is the empty string, thus
    resulting in the symbol with no name, written \verb+||+.
\end{initoptions}

\function{symbol?}
%
\begin{arguments}
    \item[object] Object to examine.
\end{arguments}
%
\result%
Returns {\em object\/} if it is a symbol.

\function{gensym}
%
\begin{arguments}
%
    \item[\optional{string}] A string contain characters to be prepended
    to the name of the new symbol.
\end{arguments}
%
\result%
Makes a new symbol whose name, by default, begins with the character \verb+#\g+
and the remaining characters are generated by an implementation-defined
mechanism\index{general}{processor-defined!\functionref{gensym} names}.
Optionally, an alternative prefix string for the name may be specified.  It is
guaranteed that the resulting symbol did not exist before the call to
\functionref{gensym}.

\function{symbol-name}
%
\begin{arguments}
    \item[symbol] A symbol.
\end{arguments}
%
\result%
Returns a {\em string\/} which is \methodref{binary=}{string} to that given as
the argument to the call to \functionref{make} which created {\em symbol}.  It
is an error to modify this string.

\function{symbol-exists?}
%
\begin{arguments}
    \item[string] A string containing the characters to be used to determine the
    existence of a symbol with that name.
\end{arguments}
%
\result%
Returns the symbol whose name is {\em string\/} if that symbol has
already been constructed by \functionref{make}.  Otherwise, returns \nil{}.

\method{generic-prin}{symbol}
%
\begin{specargs}
    \item[symbol, \classref{symbol}] The symbol to be output on {\em stream}.

    \item[stream, \classref{stream}] The stream on which the representation is
    to be output.
\end{specargs}
%
\result%
The symbol supplied as the first argument.
%
\remarks%
Outputs the external representation of {\em symbol\/} on {\em stream\/}
as described in the introduction to this section, interpreting each of the
characters in the name.

\method{generic-write}{symbol}
%
\begin{specargs}
    \item[symbol, \classref{symbol}] The symbol to be output on {\em stream}.

    \item[stream, \classref{stream}] The stream on which the representation is
    to be output.
\end{specargs}
%
\result%
The symbol supplied as the first argument.
%
\remarks%
Outputs the external representation of {\em symbol\/} on {\em stream\/}
as described in the introduction to this section.  If any characters in the name
would not normally be legal constituents of an identifier or symbol, the output
is preceded and succeeded by multiple-escape characters.
%
\examples
\begin{tabular}{lcl}
    \verb|(write (make <symbol> 'string "abc"))| &\Ra& \verb+abc+\\
    \verb|(write (make <symbol> 'string "a c"))| &\Ra& \verb+|a c|+\\
    \verb|(write (make <symbol> 'string ").("))| &\Ra& \verb+|).(|+\\
\end{tabular}

\converter{string}
%
\begin{specargs}
    \item[symbol, \classref{symbol}] A symbol to be converted to a string.
\end{specargs}
%
\result%
A string.
%
\remarks%
This function is the same as \functionref{symbol-name}.  It is defined for the
sake of symmetry.
%
\end{optDefinition}
