\sclause{Symbols}
\index{general}{symbol}
\index{general}{symbol!module}
\label{symbol}
\index{general}{level-0 modules!symbol}
%
\begin{optPrivate}
    Fixed definition of {\tt special-form-p} after RJB's observation and (later)
    KMP's CL fix to be called {\tt special-operator-p}.
\end{optPrivate}
%
\begin{optDefinition}
The defined name of this module is {\tt symbol}.
%
\syntaxform{symbol}
%
The syntax of symbols also serves as the syntax for identifiers.
Identifiers\index{general}{identifier} in \eulisp\ are very similar lexically
to identifiers in other Lisps and in other programming languages.
Informally, an identifier\index{general}{identifier!definition of} is a
sequence of {\em alphabetic}, {\em digit\/} and {\em other\/}
characters starting with a character that is not a {\em digit}.
Characters which are {\em special\/} (see section~\ref{syntax}) must
be escaped if they are to be used in the names of identifiers.
However, because the common notations for arithmetic operations are
the glyphs for plus (\verb-+-) and minus (\verb+-+), which are also
used to indicate the sign of a number, these glyphs are classified as
identifiers\index{general}{identifier!peculiar identifiers} in their own
right as well as being part of the syntax of a number.

Sometimes, it might be desirable to incorporate characters in an
identifier that are normally not legal constituents.  The aim of
escaping in identifiers is to change the meaning of particular
characters so that they can appear where they are not otherwise
acceptable.  Identifiers containing characters that are not ordinarily
legal constituents can be written by delimiting the sequence of
characters by {\em multiple-escape}, the glyph for which is called
{\em vertical bar\/} (\verb+|+).  The {\em multiple-escape\/} denotes
the beginning of an escaped {\em part\/} of an identifier and the next
{\em multiple-escape\/} denotes the end of an escaped part of an
identifier.  A single character that would otherwise not be a legal
constituent can be written by preceding it with {\em single-escape},
the glyph for which is called {\em reverse solidus\/} (\verb+\+).
Therefore, {\em single-escape\/} can be used to incorporate the {\em
multiple-escape\/} or the {\em single-escape\/} character in an
identifier, delimited (or not) by {\em multiple-escape\/}s.  For
example, \verb+|).(|+ is the identifier whose name contains the three
characters \verb+#\)+, \verb+#\.+ and \verb+#\(+, and \verb+a|b|+ is
the identifier whose name contains the characters \verb+#\a+ and
\verb+#\b+.  The sequence \verb+||+ is the identifier with no name,
and so is \verb+||||+, but \verb+|\||+ is the identifier whose name
contains the single character \verb+|+, which can also be written
\verb+\|+, without delimiting {\em multiple-escape\/}s.

Any identifier can be used as a literal, in which cases it denotes a
{\em symbol}.  Because there are two escaping mechanisms and because
the first character of a token affects the interpretation of the
remainder, there are many ways in which to input the same identifier.
If this same identifier is used as a literal the results of processing
each token denoting the identifier will be \functionref{eq} to one another.
For example, the following tokens all denote the same symbol:
%
\begin{center}
\verb+|123|+, \verb+\123+, \verb+|1|23+, \verb+||123+, \verb+||||123+
\end{center}
%
which will be output by the function \functionref{write} as \verb+|123|+.  If
output by \functionref{write}, the representation of the symbol will permit
reconstruction by \functionref{read}---escape characters are preserved---so that
equivalence is maintained between \functionref{read} and \functionref{write} for
symbols.  For example: \verb+|a(b|+ and \verb+abc.def+ are two symbols as output
by \functionref{write} such that \functionref{read} can read them as two
symbols.  If output by \functionref{prin}, the escapes necessary to
re-\functionref{read} the symbol will not be included.  Thus, taking the same
examples, \functionref{prin} outputs \verb+a(b+ and \verb+abc.def+, which
\functionref{read} interprets as the symbol \verb+a+ followed by the start of a
list, the symbol \verb+b+ and the symbol \verb+abc.def+.
%
\Syntax
\label{symbol-syntax}
\defSyntax{symbol}{
\begin{syntaxx}
    \scdef{symbol}: \\
    \>  \scref{identifier} \\
    \scdef{identifier}: \\
    \>  \scref{normal-identifier} \\
    \>   \scref{peculiar-identifier} \\
    \scdef{normal-identifier}: \\
    \>  \scref{normal-initial} \scseqref{normal-constituent} \\
    \scdef{normal-initial}: \\
    \>  \scref{letter} \\
    \>  \scref{normal-other-character} \\
    \scdef{normal-constituent}: \\
    \>  \scref{letter} \\
    \>  \scref{decimal-digit} \\
    \>  \scref{other-character} \\
    \scdef{peculiar-identifier}: \\
    \> \{+ | -\} \\
    \>\>  \scgopt{\scref{peculiar-constituent} \scseqref{normal-constituent}} \\
    \> . \scref{peculiar-constituent} \scseqref{normal-constituent} \\
    \scdef{peculiar-constituent}: \\
    \> \scref{letter} \\
    \> \scref{other-character}
\end{syntaxx}}%
\showSyntaxBox{symbol}%

Computationally, the most important aspect of symbols is that each is
unique, or, stated the other way around: the result of processing
every syntactic token comprising the same sequence of characters which
denote an identifier is the same object.  Or, more briefly, every
identifier with the same name denotes the same symbol.

\derivedclass{symbol}{name}
\index{general}{level-0 classes!\theclass{symbol}}
%
The class of all instances of \classref{symbol}.
%
\begin{initoptions}
    \item[string, string] The string containing the characters to be used to
    name the symbol.  The default value for string is the empty string, thus
    resulting in the symbol with no name, written \verb+||+.
\end{initoptions}

\function{symbolp}
%
\begin{arguments}
    \item[object] Object to examine.
\end{arguments}
%
\result%
Returns {\em object\/} if it is a symbol.

\function{gensym}
%
\begin{arguments}
%
    \item[\optional{string}] A string contain characters to be prepended
    to the name of the new symbol.
\end{arguments}
%
\result%
Makes a new symbol whose name, by default, begins with the character \verb+#\g+
and the remaining characters are generated by an implementation-defined
mechanism\index{general}{processor-defined!\functionref{gensym} names}.
Optionally, an alternative prefix string for the name may be specified.  It is
guaranteed that the resulting symbol did not exist before the call to
\functionref{gensym}.

\function{symbol-name}
%
\begin{arguments}
    \item[symbol] A symbol.
\end{arguments}
%
\result%
Returns a {\em string\/} which is \methodref{equal} to that given as the
argument to the call to \functionref{make} which created {\em symbol}.  It is an
error to modify this string.

\function{symbol-exists-p}
%
\begin{arguments}
    \item[string] A string containing the characters to be used to determine the
    existence of a symbol with that name.
\end{arguments}
%
\result%
Returns the symbol whose name is {\em string\/} if that symbol has
already been constructed by \functionref{make}.  Otherwise, returns \nil{}.

\method{generic-prin}
%
\begin{specargs}
    \item[symbol, \classref{symbol}] The symbol to be output on {\em stream}.

    \item[stream, \classref{stream}] The stream on which the representation is
    to be output.
\end{specargs}
%
\result%
The symbol supplied as the first argument.
%
\remarks%
Outputs the external representation of {\em symbol\/} on {\em stream\/}
as described in the introduction to this section, interpreting each of the
characters in the name.

\method{generic-write}
%
\begin{specargs}
    \item[symbol, \classref{symbol}] The symbol to be output on {\em stream}.

    \item[stream, \classref{stream}] The stream on which the representation is
    to be output.
\end{specargs}
%
\result%
The symbol supplied as the first argument.
%
\remarks%
Outputs the external representation of {\em symbol\/} on {\em stream\/}
as described in the introduction to this section.  If any characters in the name
would not normally be legal constituents of an identifier or symbol, the output
is preceded and succeeded by multiple-escape characters.
%
\examples
\begin{tabular}{lcl}
    \verb|(write (make <symbol> 'string "abc"))| &\Ra& \verb+abc+\\
    \verb|(write (make <symbol> 'string "a c"))| &\Ra& \verb+|a c|+\\
    \verb|(write (make <symbol> 'string ").("))| &\Ra& \verb+|).(|+\\
\end{tabular}

\converter{string}
%
\begin{specargs}
    \item[symbol, \classref{symbol}] A symbol to be converted to a string.
\end{specargs}
%
\result%
A string.
%
\remarks%
This function is the same as \functionref{symbol-name}.  It is defined for the
sake of symmetry.
%
\end{optDefinition}
