\clause{Syntax}
\label{syntax}
\index{general}{syntax}
\begin{optDefinition}
%
Case\index{general}{case sensitivity} is distinguished in each of characters,
strings and identifiers, so that {\tt variable-name} and {\tt Variable-name} are
different, but where a character is used in a positional number representation
({\em e.g.} \verb+#\x3Ad+) the case is ignored.  Thus, case is also significant
in this definition and, as will be observed later, all the special form and
standard function names are lower case.  In this section, and throughout this
text, the names for individual character glyphs are those used in ISO/IEC DIS
646:1990.

The minimal character set\index{general}{character set}
\index{general}{character!minimal character set} to support \eulisp\ is defined
in syntax table~\ref{character-set}.  The language as defined in this text uses
only the characters given in this table.  Thus, left hand sides of the
productions in this table define and name groups of characters which are used
later in this definition: {\em decimal digit}, {\em upper letter}, {\em lower
    letter}, {\em letter}, {\em other character} and {\em special character}.
Any character not specified here is classified under {\em other character},
which permits its use as an initial or a constituent character of an identifier
(see section~\ref{symbol}).
%
\Syntax
\label{character-set}
\defSyntax{character-set}{
\begin{syntaxx}
    \scdef{decimal-digit}: one of \\
    \>0 1 2 3 4 5 6 7 8 9 \\
    \scdef{upper-letter}: one of \\
    \>A B C D E F G H I J K L M \\
    \>N O P Q R S T U V W X Y Z \\
    \scdef{lower-letter}: one of \\
    \>a b c d e f g h i j k l m \\
    \>n o p q r s t u v w x y z \\
    \scdef{letter}: \\
    \>\scref{upper-letter} \\
    \>\scref{lower-letter} \\
    \scdef{other-character}: one of\\
    \>* / < = > + - . \\
    \scdef{special-character}: one of\\
    \>; ' , \textbackslash{} " \# ( ) ` | @ \\
    \scdef{level-0-character}: \\
    \>\scref{decimal-digit} \\
    \>\scref{letter} \\
    \>\scref{other-character} \\
    \>\scref{special-character}
\end{syntaxx}}
\showSyntaxBox{character-set}
%
\end{optDefinition}
%
\sclause{Whitespace and Comments}
\begin{optPrivate}
    \verb+\#tab+ is omitted from whitespace because it is not a standard
    character.

    JWD: Is \verb+#\tab+ whitespace?  Section 2.4.2 does not include it, but the
    formal syntax in section A.8 does.  In Common Lisp, it is a "semi-standard"
    character, which may indicate that it ought to be somewhat of a special
    case.

    Need to be able to allow whitespace as a constituent of numbers.  Leave for
    later!  Define an extended input syntax??
\end{optPrivate}
%
\begin{optDefinition}
Whitespace characters\index{general}{whitespace} are space and
newline.  The newline character is also used to represent end of record for
configurations providing such an input model, thus, a reference to newline in
this definition should also be read as a reference to end of record.  The only
use of whitespace is to improve the legibility of programs for human readers.
Whitespace separates tokens and is only significant in a string or when it
occurs escaped within an identifier.

A comment\index{general}{comment} is introduced by the {\em
    comment-begin}\index{general}{comment!{\em comment-begin} glyph} glyph,
called {\em semicolon} (\verb+;+) and continues up to, but does not include, the
end of the line.  Hence, a comment cannot occur in the middle of a token because
of the whitespace in the form of the newline.  Thus a comment is equivalent to
whitespace.
%
\begin{note}
    There is no notation in \eulisp\ for block comments.
\end{note}
\end{optDefinition}
%
\sclause{Objects}
\begin{optDefinition}
The syntax of the classes of objects that can be read by \eulisp\ is defined in
the section of this definition corresponding to the class as defined below:
%
\Syntax
\label{object-syntax}
\defSyntax{object}{
\begin{syntaxx}
    \scdef{object}: \\
    \>  \scref{character} \>\>\>\S\ref{character} \\
    \>  \scref{float}     \>\>\>\S\ref{float} \\
    \>  \scref{integer}   \>\>\>\S\ref{integer} \\
    \>  \scref{list}      \>\>\>\S\ref{list} \\
    \>  \scref{string}    \>\>\>\S\ref{string} \\
    \>  \scref{symbol}    \>\>\>\S\ref{symbol} \\
    \>  \scref{vector}    \>\>\>\S\ref{vector}
\end{syntaxx}}
\showSyntaxBox{object}

The syntax for identifiers \index{general}{identifier!syntax}
\index{general}{syntax!identifier} corresponds to that for symbols.
%
\end{optDefinition}
