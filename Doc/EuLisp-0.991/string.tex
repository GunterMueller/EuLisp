\sclause{Strings}
\index{general}{string}
\index{general}{string!module}
\index{general}{level-0 modules!string}
\label{string}
\begin{optDefinition}
The defined name of this module is {\tt string}.  See also
section~\ref{collection} (collections) for further operations on
strings.
%
\syntaxform{string}
%
String literals\index{general}{external representation!string} are
delimited by the glyph called {\em quotation mark\/} (\verb+"+).  For
example, \verb+"abcd"+.

Sometimes it might be desirable to include string delimiter characters
in strings.  The aim of escaping in strings\index{general}{string!escaping
in} is to fulfill this need.  The {\em
string-escape\/}\index{general}{string!string-escape glyph} glyph is
defined as {\em reverse solidus\/} (\verb+\+).  String escaping can also
be used to include certain other characters that would otherwise be
difficult to denote.  The set of named special characters (see
sections~\ref{syntax} and~\ref{character}) have a particular syntax to
allow their inclusion in strings.  To allow arbitrary characters to
appear in strings, the hex-insertion digram is followed by an integer
denoting the position of the character in the current character set.
Some examples of string literals appear in
Table~\ref{example:string-literal}.  The syntax for the external
representation of strings is defined below.
%
\Syntax
\label{string-syntax}
\newbox\stringSyntax
\begingroup

\def\a{\string\a}
\def\b{\string\b}
\def\d{\string\d}
\def\f{\string\f}
\def\l{\string\l}
\def\n{\string\n}
\def\r{\string\r}
\def\t{\string\t}
\def\v{\string\v}
\def\x{\string\x}
\def\'{\string\'}
\def\"{\string\"}
\def\\{\string\\}

% \global so it doesn't get reset to void when we leave this group...
\global\setbox\stringSyntax\vbox{\small\syntax
string token
   = '"' {string constituent} '"';
string constituent
   = normal string constituent
   | digram string constituent
   | numeric string constituent;
normal string constituent
   = level 0 character - ( '"' | '\' )
digram string constituent
   = '\a' (* \[\em alert\] *)
   | '\b' (* \[\em backspace\] *)
   | '\d' (* \[\em delete\] *)
   | '\f' (* \[\em formfeed\] *)
   | '\l' (* \[\em linefeed\] *)
   | '\n' (* \[\em newline\] *)
   | '\r' (* \[\em return\] *)
   | '\t' (* \[\em tab\] *)
   | '\v' (* \[\em vertical tab\] *)
   | '\"' (* \[\em string delimiter\] *)
   | '\\' ;  (* \[\em string escape\] *)
numeric string constituent
   = '\x' hex digit
   | '\x' hex digit, hex digit
   | '\x' hex digit, hex digit, hex digit
   | '\x' hex digit, hex digit, hex digit,
     hex digit;
\endsyntax}
\endgroup
\syntaxtable{string}{\stringSyntax}
%
\begin{example}
    \label{example:string-literal}
    \examplecaption{Examples of string literals}
    \begin{center}
        \begin{tabular}{|ll|}\hline
            Example & Contents\\\hline
            \verb+"a\nb"+ & \verb+#\a+, \verb+#\newline+ and \verb+#\b+\\
            \verb+"c\\"+ & \verb+#\c+ and \verb+#\\+\\
            \verb+"\x1 "+ & \verb+#\x1+ followed by \verb+#\space+\\
            \verb+"\xabcde"+ & \verb+#\xabcd+ followed by \verb+#\e+\\
            \verb+"\x1\x2"+ & \verb+#\x1+ followed by \verb+#\x2+\\
            \verb-"\x12+"- & \verb+#\x12+ followed by \verb-#\+-\\
            \verb+"\xabcg"+ & \verb+#\xabc+ followed by \verb+#\g+\\
            \verb+"\x00abc"+ & \verb+#\xab+ followed by \verb+#\c+\\\hline
        \end{tabular}
    \end{center}
\end{example}
%
\begin{note}
    At present this document refers to the ``current character set'' but
    defines no means of selecting alternative character sets.  This is to
    allow for future extensions and implementation-defined extensions
    which support more than one character set.
\end{note}
%
The function \functionref{write} outputs a re-readable form of any escaped
characters in the string.  For example, \verb+"a\n\\b"+ (input
notation) is the string containing the characters \verb+#\newline+,
\verb+#\a+, \verb+#\\+ and \verb+#\b+.  The function \functionref{write}
produces \verb+"a\n\\b"+, whilst \functionref{prin} produces
%
\begin{verbatim}
a
\b
\end{verbatim}
%
The function \functionref{write} outputs characters which do not have a glyph
associated with their position in the character set as a hex insertion
in which all four hex digits are specified, even if there are leading
zeros, as in the last example in Table~\ref{example:string-literal}.
The function \functionref{prin} outputs the interpretation of the characters
according to the definitions in section~\ref{character} without the
delimiting quotation marks.
%
\derivedclass{string}{character-sequence}
\index{general}{level-0 classes!\theclass{string}}
%
The class of all instances of \classref{string}.
%
\begin{initoptions}
%
\item[size, \classref{fixed-precision-integer}]
The number of characters in the string.  Strings are zero-based and
thus the maximum index is {\em size-1}.  If not specified the {\em
size\/} is zero.
%
\item[fill-value, \classref{character}]
A character with which to initialize the string.  The default fill
character is \verb|#\x0|.
%
\end{initoptions}
%
\examples
%
\begin{tabular}{lcl}
\verb|(make <string>)| &\Ra& \verb|""|\\
\verb|(make <string> 'size 2)| &\Ra& \verb|"\x0000\x0000"|\\
\verb|(make <string> 'size 3| &\Ra& \verb|"aaa"|\\
\verb|  'fill-value #\a)|&&\\
\end{tabular}
%
\function{stringp}
%
\begin{arguments}
    %
    \item[object] Object to examine.
    %
\end{arguments}
%
\result
Returns {\em object\/} if it is a string, otherwise \nil.
%
\converter{symbol}
\begin{specargs}
    \item[string, \classref{string}] A string to be converted to a symbol.
\end{specargs}
%
\result
If the result of \functionref{symbol-exists-p} when applied to {\em string\/}
is a symbol, that symbol is returned.  If the result is \nil, then
a new symbol is constructed whose name is {\em string}.  This new
symbol is returned.
%
\method{equal}
%
\begin{specargs}
    \item[string$_1$, \classref{string}] A string.
    \item[string$_2$, \classref{string}] A string.
\end{specargs}
%
\result
If the size of {\em string$_1$} is the same (under {\tt =}) as that of
{\em string$_2$}, then the result is the conjunction of the pairwise
application of \genericref{equal} to the elements of the arguments.  If not
the result is \nil.
%
\method{deep-copy}
%
\begin{specargs}
    \item[string, \classref{string}] A string.
\end{specargs}
%
\result
Constructs and returns a copy of {\em string\/} in which each element
is \functionref{eql} to the corresponding element in {\em string}.
%
\method{shallow-copy}
%
\begin{specargs}
    \item[string, \classref{string}] A string.
\end{specargs}
%
\result
Constructs and returns a copy of {\em string\/} in which each element
is \functionref{eql} to the corresponding element in {\em string}.
%
\method{binary<}
%
\begin{specargs}
    \item[string$_1$, \classref{string}] A string.
    \item[string$_2$, \classref{string}] A string.
\end{specargs}
%
\result%
If the second argument is longer than the first, the result is \nil.  Otherwise,
if the sequence of characters in {\em string$_1$} is pairwise less than that in
{\em string$_2$} according to \genericref{binary<} the result is \true.
Otherwise the result is \nil.  Since it is an error to compare lower case, upper
case and digit characters with any other kind than themselves, so it is an error
to compare two strings which require such comparisons and the results are
undefined.
%
\examples
\begin{tabular}{lcl}
\verb|(< "a" "b")| &\Ra& \verb|t|\\
\verb|(< "b" "a")| &\Ra& \verb|()|\\
\verb|(< "a" "a")| &\Ra& \verb|()|\\
\verb|(< "a" "ab")| &\Ra& \verb|t|\\
\verb|(< "ab" "a")| &\Ra& \verb|()|\\
\verb|(< "A" "B")| &\Ra& \verb|t|\\
\verb|(< "0" "1")| &\Ra& \verb|t|\\
\verb|(< "a1" "a2")| &\Ra& \verb|t|\\
\verb|(< "a1" "bb")| &\Ra& \verb|t|\\
\verb|(< "a1" "ab")| &\Ra& {\em undefined}
\end{tabular}
%
\seealso
Method on \genericref{binary<} (\ref{compare}) for characters (\ref{character}).
%
\method{as-lowercase}
%
\begin{specargs}
    \item[string, \classref{string}] A string.
\end{specargs}
%
\result
Returns a copy of {\em string\/} in which each character denoting an
upper case character, is replaced by a character denoting its lower
case counterpart.  The result must not be \functionref{eq} to {\em string}.
%
\method{as-uppercase}
%
\begin{specargs}
    \item[string, \classref{string}] A string.
\end{specargs}
%
\result
Returns a copy of {\em string\/} in which each character denoting an
lower case character, is replaced by a character denoting its upper
case counterpart.  The result must not be \functionref{eq} to {\em string}.
%
\method{generic-prin}
\begin{specargs}
    \item[string, \classref{string}] String to be ouptut on {\em stream}.
    \item[stream, \classref{stream}] Stream on which {\em string} is to be ouptut.
\end{specargs}
%
\result
The string {\em string}.
%
Output external representation of {\em string\/} on {\em stream\/} as
described in the introduction to this section, interpreting each of
the characters in the string.  The opening and closing quotation marks
are not output.
%
\method{generic-write}
\begin{specargs}
    \item[string, \classref{string}] String to be ouptut on {\em stream}.
    \item[stream, \classref{stream}] Stream on which {\em string\/} is to be ouptut.
\end{specargs}
%
\result
The string {\em string}.
%
Output external representation of {\em string\/} on {\em stream\/} as
described in the introduction to this section, replacing single
characters with escape sequences if necessary.  Opening and closing
quotation marks are output.
\end{optDefinition}
