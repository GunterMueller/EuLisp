\sclause{Low Level Allocation Primitives}
%
\begin{optDefinition}
This module provides primitives which are necessary to implement new allocation
methods portably. However, they should be defined in such a way that objects
cannot be destroyed unintentionally.  In consequence it is an error to use
\functionref{primitive-class-of}, \functionref{primitive-ref} and their setters
on objects not created by \functionref{primitive-allocate}.

\function{primitive-allocate}
%
\begin{arguments}
    \item[class] A class.
    \item[size] An integer.
\end{arguments}
%
\result%
An instance of the first argument.
%
\remarks%
This function returns a new instance of the first argument which has a
vector-like structure of length {\em size}. The components of the new instance
can be accessed using \functionref{primitive-ref} and updated using
\setterref{primitive-ref}.  It is intended to be used in new
\genericref{allocate} methods defined for new metaclasses.

\function{primitive-class-of}
%
\begin{arguments}
    \item[object] An object created by \functionref{primitive-allocate}.
\end{arguments}
%
\result%
A class.
%
\remarks%
This function returns the class of an object. It is similar to
\functionref{class-of}, which has a defined behaviour on any object. It is an
error to use \functionref{primitive-class-of} on objects which were not created
by \functionref{primitive-allocate}.

\setter{primitive-class-of}
%
\begin{arguments}
    \item[object] An object created by \functionref{primitive-allocate}.
    \item[class] A class.
\end{arguments}
%
\result%
The {\em class}.
%
\remarks%
This function supports portable implementations of
\begin{enumerate}
    \item dynamic classification like {\tt change-class} in CLOS.
    \item automatic instance updating of redefined classes.
\end{enumerate}

\function{primitive-ref}
%
\begin{arguments}
    \item[object] An object created by \functionref{primitive-allocate}.
    \item[index] The index of a component.
\end{arguments}
%
\result%
An object.
%
\remarks%
Returns the value of the objects component corresponding to the supplied index.
It is an error if {\em index} is outside the index range of {\em object}.  This
function is intended to be used when defining new kinds of accessors for new
metaclasses.

\setter{primitive-ref}
%
\begin{arguments}
    \item[object] An object created by \functionref{primitive-allocate}.
    \item[index] The index of a component.
    \item[value] The new value, which can be any object.
\end{arguments}
%
\result%
The new value.
%
\remarks%
Stores and returns the new value as the objects component corresponding to the
supplied index.  It is an error if {\em index} is outside the index range of
{\em object}.  This function is intended to be used when defining new kinds of
accessors for new metaclasses.

\end{optDefinition}
