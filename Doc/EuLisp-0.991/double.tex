\defModule{double-float}{Double Precision Floats}
%
\begin{optDefinition}
\noindent

The defined name of this module is {\tt double}.  Arithmetic operations for
\classref{double-float}\ are defined by methods on the generic functions defined
in the compare module (\ref{compare}):
%
\begin{flushleft}
    \genericref{binary=},\ttindex{binary=}\indexmeth{binary=}
    \genericref{binary<},\ttindex{binary<}\indexmeth{binary<}
\end{flushleft}
%
\noindent
the number module (\ref{number}):
%
\begin{flushleft}
    \genericref{binary+},\ttindex{binary+}\indexmeth{binary+}
    \genericref{binary-},\ttindex{binary-}\indexmeth{binary-}
    \genericref{binary*},\ttindex{binary*}\indexmeth{binary*}
    \genericref{binary/},\ttindex{binary/}\indexmeth{binary/}
    \genericref{binary-mod},\ttindex{binary-mod}\indexmeth{binary-mod}
    \genericref{negate},\ttindex{negate}\indexmeth{negate}
    \genericref{zero?}\ttindex{zero?}\indexmeth{zero?}
\end{flushleft}
%
\noindent
the float module (\ref{float}):
\begin{flushleft}
\genericref{ceiling},\ttindex{ceiling}\indexmeth{ceiling}
\genericref{floor},\ttindex{floor}\indexmeth{floor}
\genericref{round},\ttindex{round}\indexmeth{round}
\genericref{truncate}\ttindex{truncate}\indexmeth{truncate}
\end{flushleft}
%
\noindent
and the elementary functions module (\ref{elementary-functions}):
\begin{flushleft}
\genericref{acos},\ttindex{acos}\indexmeth{acos}
\genericref{asin},\ttindex{asin}\indexmeth{asin}
\genericref{atan},\ttindex{atan}\indexmeth{atan}
\genericref{atan2},\ttindex{atan2}\indexmeth{atan2}
\genericref{cos},\ttindex{cos}\indexmeth{cos}
\genericref{sin},\ttindex{sin}\indexmeth{sin}
\genericref{tan},\ttindex{tan}\indexmeth{tan}
\genericref{cosh},\ttindex{cosh}\indexmeth{cosh}
\genericref{sinh},\ttindex{sinh}\indexmeth{sinh}
\genericref{tanh},\ttindex{tanh}\indexmeth{tanh}
\genericref{exp},\ttindex{exp}\indexmeth{exp}
\genericref{log},\ttindex{log}\indexmeth{log}
\genericref{log10},\ttindex{log10}\indexmeth{log10}
\genericref{pow},\ttindex{pow}\indexmeth{pow}
\genericref{sqrt}\ttindex{sqrt}\indexmeth{sqrt}
\end{flushleft}
%
\noindent
The behaviour of these functions is defined in the modules noted
above.

\derivedclass{double-float}{float}
%
The class of all double precision floating point numbers.

The syntax for the exponent of a double precision floating point is given below:

\defSyntax{double-float}{
\begin{syntax}
    \scdef{double-exponent}: \\
    \>  d \scoptref{sign} \scref{decimal-integer} \\
    \>  D \scoptref{sign} \scref{decimal-integer}
\end{syntax}}%
\showSyntaxBox{double-float}

The general syntax for floating point numbers is given in syntax
table~\ref{float-syntax}.

\function{double-float?}
%
\begin{arguments}
    \item[object] Object to examine.
\end{arguments}
%
\result%
Returns {\em object\/} if it is a double float, otherwise \nil{}.
%
\seealso%
\functionref{float?} (\ref{number}).

\constant{most-positive-double-float}{double-float}
\index{general}{implementation-defined!most positive double precision float}
\index{general}{conformity-clause!most positive double precision float}
%
\remarks%
The value of \constantref{most-positive-double-float} is that positive double
precision floating point number closest in value to (but not equal to) positive
infinity that the processor provides.

\constant{least-positive-double-float}{double-float}
\index{general}{implementation-defined!least positive double precision float}
\index{general}{conformity-clause!least positive double precision float}
%
\remarks%
The value of \constantref{least-positive-double-float} is that positive double
precision floating point number closest in value to (but not equal to) zero that
the processor provides.

\constant{least-negative-double-float}{double-float}
\index{general}{implementation-defined!least negative double precision float}
\index{general}{conformity-clause!least negative double precision float}
%
\remarks%
The value of \constantref{least-negative-double-float} is that negative double
precision floating point number closest in value to (but not equal to) zero that
the processor provides.  Even if the processor provide negative zero, this value
must not be negative zero.

\constant{most-negative-double-float}{double-float}
\index{general}{implementation-defined!most negative double precision float}
\index{general}{conformity-clause!most negative double precision float}
%
\remarks%
The value of \constantref{most-negative-double-float} is that negative double
precision floating point number closest in value to (but not equal to) negative
infinity that the processor provides.

\converter{string}
%
\begin{specargs}
    \item[x, \classref{double-float}] A double precision float.
\end{specargs}
%
\result%
Constructs and returns a string, the characters of which correspond to the
external representation of {\em x\/} as produced by {\tt generic-print}, namely
that specified in the syntax as {\em {\tt[}sign{\tt]}float format 3}.

\converter{fpi}
%
\begin{specargs}
    \item[x, \classref{double-float}] A double precision float.
\end{specargs}
%
\result%
A fixed precision integer.

\remarks%
This function is the same as the \classref{double-float} method of
\genericref{round}.  It is defined for the sake of symmetry.

\method{generic-print}{double-float}
%
\begin{specargs}
    \item[double, \classref{double-float}]%
    The double float to be output on {\em stream}.
    \item[stream, \classref{stream}]%
    The stream on which the representation is to be output.
\end{specargs}
%
\result%
The double float supplied as the first argument.
%
\remarks%
Outputs the external representation of {\em double\/} on {\em stream}, as an
optional sign preceding the syntax defined by {\em float format 3}.  Finer
control over the format of the output of floating point numbers is provided by
some of the formatting specifications of {\tt format} (see
section~\ref{formatted-io}).

\method{generic-write}{double-float}
%
\begin{specargs}
    \item[double, \classref{double-float}]%
    The double float to be output on {\em stream}.
    \item[stream, \classref{stream}]%
    The stream on which the representation is to be output.
\end{specargs}
%
\result%
The double float supplied as the first argument.
%
\remarks%
Outputs the external representation of {\em double\/} on {\em stream}, as an
optional sign preceding the syntax defined by {\em float format 3}.  Finer
control over the format of the output of floating point numbers is provided by
some of the formatting specifications of {\tt format} (see
section~\ref{formatted-io}).
%
\end{optDefinition}
