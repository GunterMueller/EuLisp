\sclause{Floating Point Numbers}
\index{general}{float}
\index{general}{float!module}
\label{float}
\index{general}{level-0 modules!float}
%
\begin{optDefinition}
\noindent
The defined name of this module is {\tt float}.  This module defines the
abstract class \classref{float}\ and the behaviour of some generic functions on
floating point numbers.  Further operations on numbers are defined in the
numbers module (\ref{number}) and further operations on floating point numbers
are defined in the elementary functions module (\ref{elementary-functions}).  A
concrete float class is defined in the double float module (\ref{double-float}).

\syntaxform{float}
%
The syntax for the external representation of floating point literals is defined
in Table~\ref{float-syntax}.  The representation used by \functionref{write} and
\functionref{prin} is that of a sign, a whole part and a fractional part without
an exponent, namely that defined by {\em float format 3}.  Finer control over
the format of the output of floating point numbers is provided by some of the
formatting specifications of \functionref{format} (section~\ref{formatted-io}).
%
\Syntax
\label{float-syntax}
\savesyntax\floatSyntax\vbox{\small\syntax
float
   = [sign] unsigned float [exponent];
sign
   = '+' | '-';
unsigned float
   = float format 1
   | float format 2
   | float format 3;
float format 1
   = decimal integer, '.'; (* \[\S\ref{integer}\] *)
float format 2
   = '.', decimal integer; (* \[\S\ref{integer}\] *)
float format 3
   = float format 1, decimal integer; (* \[\S\ref{integer}\] *)
exponent
   = double exponent; (* \[\S\ref{double-float}\] *)
\endsyntax}
\syntaxtable{float}{\floatSyntax}
%
A floating point number\index{general}{external representation!floating point}
has six forms of external representation depending on whether either or both the
whole and the fractional part are specified and on whether an exponent is
specified.  In addition, a positive floating point number is optionally preceded
by a plus sign and a negative floating point number is preceded by a minus sign.
For example:
%
\verb|+123.| ({\em float format 1\/}),
\verb|-.456| ({\em float format 2\/}),
\verb|123.456| ({\em float format 3\/}); and with exponents:
\verb|+123456.D-3|,
\verb|1.23455D2|,
\verb|-.123456D3|.

\derivedclass{float}{number}
\index{general}{level-0 classes!\theclass{float}}
%
The abstract class which is the superclass of all floating point
numbers.
%
\function{floatp}
\begin{arguments}
    \item[objext] Object to examine.
\end{arguments}
%
\result%
Returns {\em object\/} if it is a floating point number, otherwise \nil.

\generic{ceiling}
%
\begin{genericargs}
    \item[float, \classref{float}] A floating point number.
\end{genericargs}
%
\result%
Returns the smallest integral value not less than {\em float\/} expressed as a
float of the same class as the argument.

\generic{floor}
%
\begin{genericargs}
    \item[float, \classref{float}] A floating point number.
\end{genericargs}
%
\result%
Returns the largest integral value not greater than {\em float\/} expressed as a
float of the same class as the argument.

\generic{round}
%
\begin{arguments}
    \item[float] A floating point number.
\end{arguments}
%
\result%
Returns the integer whose value is closest to {\em float}, except in the case
when {\em float\/} is exactly half-way between two integers, when it is rounded
to the one that is even.

\generic{truncate}
%
\begin{arguments}
    \item[float] A floating point number.
\end{arguments}
%
\result%
Returns the greatest integer value whose magnitude is less than or equal to {\em
    float}.
%
\end{optDefinition}
