\sclause{Collections}
\index{general}{collection}
\label{collection}
\index{general}{level-0 modules!collection}
\index{general}{collection!module}
%
\begin{optPrivate}
    Some terms which may need defining: natural order indexed objects versus
    stretchy objects
\end{optPrivate}
%
\begin{optDefinition}
%
The defined name of this module is {\tt collection}.  A {\em collection\/} is
defined as an instance of one of \classref{list}, \classref{string},
\classref{vector}, \classref{table}\ or any user-defined class for which a
method is added to any of the collection manipulation functions.  Collection
does not name a class and does not form a part of the class hierarchy.  This
module defines a set of operators on collections as generic functions and
default methods with the behaviours given here.

When iterating over a single collection, the order in which elements are
processed might not be important, but as soon as more than one collection is
involved, it is necessary to specify how the collections are
aligned\index{general}{collection!alignment} so that it is clear which elements
of the collections will be processed together.  This is quite straightforward in
the cases of \classref{list}, \classref{string}\ and \classref{vector}, since
there is an intuitive {\em natural\/} order for the elements which allows them
to be identified by a non-negative integer.  Thus, when iterating over a
combination of any of these, all the elements at index position $i$ will be
processed together, starting with the elements at position $0$ and finishing
with those at position $n-1$ where $n$ is the size of the smallest collection in
the combination.  The subset of collections which have natural order is called
{\em sequence\/} and members of this set can be identified by the predicate {\tt
    sequence?}, while collections in general can be identified by {\tt
    collection?}.

Collection alignment is more complicated when tables are involved since they use
explicit keys rather than natural order to identify their elements.  In any
iteration over a combination of collections including a table or some tables,
the set of keys used is the intersection of the keys of the tables and the
implicit keys of the other collection classes present; this identifies the
elements of the collections with common keys.  Thus, for an iteration to process
any elements from the combination of a collection with natural order and a
table, the table must have some integer keys and they must be in the range
$[0\ldots{}size)$ of the collection with natural order.

A conforming level-0 implementation must define methods on these functions to
support operations on lists (\ref{list}), strings (\ref{string}), tables
(\ref{table}), vector (\ref{vector}) and any combination of these.

\derivedclass{collection}{object}
\index{general}{level-0 classes!\theclass{collection}}
%
The class of all collections.

\derivedclass{sequence}{collection}
\index{general}{level-0 classes!\theclass{sequence}}
%
The class of all sequences, the subset of collections which have natural order.

\derivedclass{character-sequence}{sequence}
\index{general}{level-0 classes!\theclass{sequence}}
%
The class of all sequences of characters {\em e.g.} \classref{string}.

\condition{collection-condition}{condition}
%
This is the condition class for all collection processing conditions.

\generic{accumulate}
%
\begin{genericargs}
    \item[function, \classref{function}] A function of two arguments.
%
    \item[obj, \classref{object}] The object which is the initial value for the
    accumulation operation.
%
    \item[collection, \classref{collection}] The collection which is the subject
    of the accumulation operation.
\end{genericargs}
%
\result%
The result is the result of the application of {\em function\/} to the
accumulated result and successive elements of {\em collection}.  The initial
value of the accumulated result is supplied by {\em obj}.
%
\examples
Note that the order of the elements in the result of the second example depends
on the hashing algorithm of the implementation and does not prescribe the result
that any particular implementation must give.
%
\begin{tabular}{lcl}
    \verb+(accumulate * 1 #(1 2 3 4 5))+ & \Ra & \verb+120+\\
    \begin{minipage}[t]{\columnwidth}
        {\tt\begin{tabbing}
                (a\=ccumulate\\
                \>(lambda (a v) (cons v a))\\
                \>()\\
                \>(m\=ake <table>\\
                \>  \>'entries\\
                \> \>'((1 . b) (0 . a) (2 . c))))
            \end{tabbing}}
    \end{minipage}
    & \Ra &
    \verb+(c b a)+\\
\end{tabular}

\generic{accumulate1}
%
\begin{genericargs}
    \item[function, \classref{function}] A function of two arguments.
%
    \item[collection, \classref{collection}] The collection which is the subject
    of the accumulation operation.
\end{genericargs}
%
\result%
The result is the result of the application of {\em function\/} to the
accumulated result and successive elements of {\em collection\/} starting with
the second element.  The initial value of the accumulated result is the first
element of {\em collection}.  The terms first and second correspond to the
appropriate elements of a natural order collection, but no elements in
particular of an explicit key collection.  If {\em collection\/} is empty, the
result is \nil{}.
%
\examples
\begin{tabular}{lcl}
\begin{minipage}[t]{\columnwidth}
{\tt
    \begin{tabbing}
        (a\=ccumulate1\\
        \>(l\=ambda (a v)\\
        \>  \>(if (evenp v) (cons v a) a))\\
        \>'(1 2 3 4 5))
    \end{tabbing}
}
\end{minipage}
& \Ra &
\verb+(4 2)+\\
\end{tabular}

\generic{all?}
%
\begin{genericargs}
    \item[function, \classref{function}] A function to be used as a predicate on
    the elements of the collection(s).
    %
    \item[collection, \classref{collection}] A collection.
    %
    \item[\optional{more-collections}] More collections.
\end{genericargs}
%
\result%
The {\em function\/} is applied to argument lists constructed from corresponding
successive elements of {\em collection\/} and {\em more-collections}.  If the
result is \true{} for all elements of {\em collection\/} and {\em
    more-collections} the result of \genericref{all?} is \true{} otherwise
\nil{}.
%
\examples
\begin{tabular}{lcl}
    \verb+(all? even? #(1 2 3 4))+ & \Ra & \nil{}\\
    \verb+(all? even? #(2 4 6 8))+ & \Ra & \true{}
\end{tabular}
%
\seealso%
\genericref{any?}.

\generic{any?}
%
\begin{genericargs}
    \item[function, \classref{function}] A function to be used as a predicate on
    the elements of the collection(s).
%
    \item[collection, \classref{collection}] A collection.
%
    \item[\optional{more-collections}] More collections.
\end{genericargs}
%
\result%
The {\em function\/} is applied to argument lists constructed from corresponding
successive elements of {\em collection\/} and {\em more-collections}.  If the
result is \true{}, the result of \genericref{any?} is \true{} and there are no
further applications of {\em function\/} to elements of {\em collection\/} and
{\em more-collections}.  If any of the collections is exhausted, the result of
\genericref{any?} is \nil{}.
%
\examples
\begin{tabular}{lcl}
\verb+(any? even? #(1 2 3 4))+ & \Ra & \true{}\\
\begin{minipage}[t]{\columnwidth}
{\tt
    \begin{tabbing}
        (l\=et ((x (list 1 2 3 4)))\\
        \>(any? > x (cdr x)))
    \end{tabbing}}
\end{minipage}
& \Ra & \nil{}\\
\begin{minipage}[t]{\columnwidth}
{\tt
    \begin{tabbing}
        (a\=ny?\\
        \>(lambda (a b) (= (\% a b) 0))\\
        \>\verb|#(3 2) '(1 0))|
    \end{tabbing}}
\end{minipage}
& \Ra & \true{}
\end{tabular}
%
\seealso%
\genericref{all?}.

\generic{collection?}
%
\begin{genericargs}
    \item[object, \classref{object}] An object to examine.
\end{genericargs}
%
\result%
Returns \true{} if {\em object\/} is a collection, otherwise \nil{}.
%
\remarks%
This predicate does not return {\em object\/} because \nil{}\/ is a
collection.

\generic{concatenate}
%
\begin{genericargs}
    \item[collection, \classref{collection}] A collection.
%
    \item[\optional{more-collections}] More collections.
\end{genericargs}
%
\result%
The result is an object of the same class as {\em collection}.
%
\remarks%
The contents of the result object depend on whether {\em collection\/}
has natural order or not:
\begin{enumerate}
    \item If {\em collection\/} has natural order then the size of the result is
    the sum of the sizes of {\em collection\/} and {\em more-collections}.  The
    result collection is initialized with the elements of {\em collection\/}
    followed by the elements of each of {\em more-collections\/} taken in turn.
    If any element cannot be stored in the result collection, for example, if
    the result is a string and some element is not a character, an error is
    signalled (condition class: {\tt
        collection-condition}\indexcondition{collection-condition}).
%
    \item If {\em collection\/} does not have natural order, then the result
    will contain associations for each of the keys in {\em collection\/} and
    {\em more-collections}.  If any key occurs more than once, the associated
    value in the result is the value of the last occurrence of that key after
    processing {\em collection\/} and each of {\em more-collections\/} taken in
    turn.
\end{enumerate}
%
\examples
\begin{tabular}{lcl}
\verb+(concatenate #(1) '(2) "bc")+ & \Ra & \verb+#(1 2 #\b #\c))+\\
\verb+(concatenate "a" '(#\b))+ & \Ra & \verb+"ab"+\\
\begin{minipage}[t]{0.6\columnwidth}
{\tt\begin{tabbing}
(c\=oncatenate\\
  \>(make <table>)\\
  \>'(a b)\\
  \>"c")\\
\end{tabbing}}\end{minipage}
& \Ra & %
\begin{minipage}[t]{0.3\columnwidth}
\begin{verbatim}
#<table:
  0 -> "c",
  1 -> b>
\end{verbatim}
\end{minipage}
\end{tabular}

\generic{delete}
%
\begin{genericargs}
    \item[object, \classref{object}] Object to be removed.
    %
    \item[collection, \classref{collection}] A collection.
    %
    \item[\optional{test}] The function to be used to compare {\em object|/} and
    the elements of {\em collection\/}.
\end{genericargs}
%
\result%
If there is an element of {\em collection\/} such that test returns \true{}
when applied to {\em object|/} and that element, then the result is the modified
{\em collection\/}, less that element. Otherwise, the result is {\em
    collection\/}.
%
\remarks%
\genericref{delete} is destructive. The {\em test\/} function defaults
to \functionref{eql}.

\generic{do}
%
\begin{genericargs}
    \item[function, \classref{function}] A function.
    %
    \item[collection, \classref{collection}] A collection.
    %
    \item[\optional{more-collections}] More collections.
\end{genericargs}
%
\result%
The result is \nil{}.  This operator is used for side-effect only.  The {\em
    function\/} is applied to argument lists constructed from corresponding
successive elements of {\em collection\/} and {\em more-collections\/} and the
result is discarded.  Application stops if any of the collections is exhausted.
%
\examples
\begin{tabular}{lcl}
    \verb|(do prin '(1 b #\c))| &\Ra& \verb|1bc|\\
    \verb|(do write '(1 b #\c))| &\Ra& \verb|1b#\c|\\
\end{tabular}

\generic{element}
%
\begin{genericargs}
    \item[collection, \classref{collection}] The object to be accessed or
    updated.
    %
    \item[key, \classref{object}] The object identifying the key of the element
    in {\em collection}.
\end{genericargs}
%
\result%
The value associated with {\em key\/} in {\em collection}.
%
\examples
\begin{tabular}{lcl}
\verb+(element "abc" 1)+ & \Ra & \verb+#\b+\\
\verb+(element '(a b c) 1)+ & \Ra & \verb+b+\\
\verb+(element #(a b c) 1)+ & \Ra & \verb+b+\\
\begin{minipage}[t]{\columnwidth}
{\tt\begin{tabbing}
(e\=lement\\
  \>(m\=ake <table> fill-value: 'b)\\
  \>1)
\end{tabbing}}\end{minipage}
& \Ra & \verb+b+\\
\end{tabular}

\setter{element}
%
\begin{genericargs}
    \item[collection, \classref{collection}] The object to be accessed or
    updated.
    %
    \item[key, \classref{object}] The object identifying the key of the element
    in {\em collection}.
    %
    \item[value, \classref{object}] The object to replace the value associated
    with {\em key\/} in {\em collection\/} (for setter).
\end{genericargs}
%
\result%
The argument supplied as {\em value\/}, having updated the association
of {\em key\/} in {\em collection\/} to refer to {\em value}.
% Note: can't say ``is an error'' to reference a key outside range of
% collection because that conflicts with tables.

\generic{empty?}
%
\begin{genericargs}
    \item[collection, \classref{collection}] The object to be examined.
\end{genericargs}
%
\result%
Returns \true{} if {\em collection\/} is the object identified with the
empty object for that class of collection.
%
\examples
\begin{tabular}{lcl}
    \verb+(emptyp "")+ & \Ra & \true{}\\
    \verb+(emptyp ())+ & \Ra & \true{}\\
    \verb+(emptyp #())+ & \Ra & \true{}\\
    \verb+(emptyp (make <table>))+ & \Ra & \true{}\\
\end{tabular}

\generic{fill}
%
\begin{genericargs}
    \item[collection, \classref{collection}] A collection to be (partially)
    filled.
    %
    \item[object, \classref{object}] The object with which to fill {\em
        collection}.
    %
    \item[\optional{keys}] The keys with which {\em object\/} is to be
    associated.
    %
\end{genericargs}
%
\result%
The result is \nil{}.
%
\remarks%
This function side-effects {\em collection\/} by updating the values
associated with each of the specified {\em keys\/} with {\em obj}.  If
no {\em keys\/} are specified, the whole collection is filled with
{\em obj}.  Otherwise, the key specification can take two forms:
%
\begin{enumerate}
    \item A collection, in which case the values of the collection are taken to
    be the keys of {\em collection\/} to be associated with {\em obj}.
    %
    \item Two fixed precision integers, denoting the start and end keys,
    respectively, in a natural order collection to be associated with {\em obj}.
    An error is signalled (condition class: {\tt
        collection-condition}\index{general}{collection-condition}) if {\em
        collection\/} does not have natural order.  It is an error if the start
    and end do not specify an ascending sub-interval of the interval $[0,size)$,
    where {\em size\/} is that of {\em collection}.
\end{enumerate}

\generic{find-key}
%
\begin{genericargs}
    \item[collection, \classref{collection}] A collection.
    %
    \item[test, \classref{function}] A function.
    %
    \item[\optional{skip}] An integer.
    %
    \item[\optional{failure}] An integer.
\end{genericargs}
%
\result%
The function {\em test\/} is applied to successive elements of {\em
    collection}. If {\em test\/} returns \true{} when applied to an element,
then the result of \genericref{find-key} is the key associated with that
element.
%
\remarks%
The value {\em skip\/}, which defaults to zero, indicates how many successful
tests are to be made before returning a result. The value {\em failure}, which
defaults to \nil{}, is returned if no key satisfying the test was found. Note
that {\em skip\/} and {\em failure\/} are positional arguments and that {\em
    skip\/} must be specified if {\em failure\/} is specified.

\generic{first}
%
\begin{genericargs}
    \item[sequence, \classref{sequence}] A sequence.
\end{genericargs}
%
\result%
The result is contents of index position 0 of {\em sequence}.

\generic{last}
%
\begin{genericargs}
    \item[sequence, \classref{sequence}] A sequence.
\end{genericargs}
%
\result%
The result is last element of {\em sequence}.

\generic{key-sequence}
%
\begin{genericargs}
    \item[collection, \classref{collection}] A collection.
\end{genericargs}
%
\result%
The result is a collection comprising the keys of {\em collection}.

\generic{map}
%
\begin{genericargs}
    \item[function, \classref{function}] A function.
    %
    \item[collection, \classref{collection}] A collection.
    %
    \item[\optional{more-collections}] More collections.
\end{genericargs}

\result%
The result is an object of the same class as {\em collection}.  The
elements of the result are computed by the application of {\em function\/} to
argument lists constructed from corresponding successive elements of {\em
    collection\/} and {\em more-collections}.  Application stops if any of the
collections is exhausted.
%
\examples
\begin{tabular}{lcl}
\verb+(map cons #(1 2) '(3))+ & \Ra & \verb+#((1 . 3))+\\
\begin{minipage}[t]{\columnwidth}
{\tt\begin{tabbing}
(m\=ap\\
  \>(lambda (f) (f 1 2))\\
  \>\verb|#(+ - * %))|
\end{tabbing}}\end{minipage}
&\Ra& \verb|#(3 -1 2 1)|
\end{tabular}

\generic{member}
%
\begin{genericargs}
    \item[object, \classref{object}] The object to be searched for in {\em
        collection}.
    %
    \item[collection, \classref{collection}] The collection to be searched.
    %
    \item[\optional{test}] The function to be used to
    compare {\em object\/} and the elements of {\em collection}.
\end{genericargs}
%
\result%
Returns \true{} if there is an element of {\em collection\/} such that the
result of the application of {\em test\/} to {\em object\/} and that element is
\true{}.  If {\em test\/} is not supplied, \functionref{eql} is used by default.
Note that \true{} denotes any value that is not \nil{}\/ and that the class of
the result depends on the class of {\em collection}.  In particular, if {\em
    collection\/} is a list, the result of \genericref{member} is a list.
%
\examples
\begin{tabular}{lcl}
    \verb+(member #\b "abc")+ & \Ra & \true{}\\
    \verb+(member 'b '(a b c))+ & \Ra & \verb+(b c)+\\
    \verb+(member 'b #(a b c))+ & \Ra & \true{}\\
    \begin{minipage}[t]{\columnwidth}
        {\tt\begin{tabbing}
                (m\=ember\\
                \>'b\\
                \>(m\=ake <table>\\
                \>  \>'entries\\
                \>  \>'((1 . b) (0 . a) (2 . c))))
            \end{tabbing}}\end{minipage}
    & \Ra & \true{}\\
\end{tabular}

\generic{remove}
%
\begin{genericargs}
    \item[object, \classref{object}] Object to be removed.
    %
    \item[collection, \classref{collection}] A collection.
    %
    \item[\optional{test}] The function to be used to compare {\em object\/} and
    the elements of {\em collection}.
\end{genericargs}
%
\result%
If there is an element of {\em collection\/} such that test returns \true{} when
applied to {\em object\/} and that element, then the result is a shallow copy of
{\em collection\/} less that element. Otherwise, the result is {\em collection}.
%
\remarks%
The test function defaults to \functionref{eql}.

\generic{reverse}
%
\begin{genericargs}
    \item[collection, \classref{collection}] A collection.
\end{genericargs}
%
\result%
The result is an object of the same class as {\em collection\/} whose
elements are the same as those in {\em collection}, but in the reverse order
with respect to the natural order of {\em collection}.  If {\em collection\/}
does not have natural order, the result is equal to the argument.
%
\examples
\begin{tabular}{lcl}
    \verb+(reverse "abc")+ & \Ra & \verb+"cba"+\\
    \verb+(reverse '(1 2 3))+ & \Ra & \verb+(3 2 1)+\\
    \verb+(reverse #(a b c))+ & \Ra & \verb+#(c b a)+
\end{tabular}

\generic{reverse!}
%
\begin{genericargs}
    \item[collection, \classref{collection}] A collection.
\end{genericargs}
%
\result%
Destructively reverses the order of the elements in {\em collection\/} (see
\genericref{reverse}) and returns it.

\generic{sequence?}
%
\begin{genericargs}
    \item[object, \classref{object}] An object to examine.
\end{genericargs}
%
\result%
Returns \true{} if {\em object\/} is a sequence (has natural order),
otherwise \nil{}.
%
\remarks%
This predicate does not return {\em object\/} because \nil{}\/ is a
sequence.

\generic{size}
%
\begin{genericargs}
    \item[collection, \classref{collection}] The object to be examined.
\end{genericargs}
%
\result%
An integer which denotes the size of {\em collection\/} according to
the method for the class of {\em collection}.
%
\examples
\begin{tabular}{lcl}
    \verb+(size "")+ & \Ra & \verb+0+\\
    \verb+(size ())+ & \Ra & \verb+0+\\
    \verb+(size #())+ & \Ra & \verb+0+\\
    \verb+(size (make <table>))+ & \Ra & \verb+0+\\
    \verb+(size "abc")+ & \Ra & \verb+3+\\
    \verb+(size (cons 1 ()))+ & \Ra &\verb+1+\\
    \verb+(size (cons 1 . 2))+ & \Ra &\verb+1+\\
    \verb+(size (cons 1 (cons 2 . 3)))+ & \Ra & \verb+2+\\
    \verb+(size '(1 2 3))+ & \Ra & \verb+3+\\
    \verb+(size #(a b c))+ & \Ra & \verb+3+\\
    \verb+(size (make <table> 'entries '((0 . a)))+ & \Ra & \verb+1+\\
\end{tabular}

\generic{slice}
%
\begin{genericargs}
    \item[sequence, \classref{sequence}] A sequence.
    %
    \item[start, \classref{int}] The index of the first
    element of the slice.
    \item[end, \classref{int}] The index of the last
    element of the slice.
\end{genericargs}
%
\result%
The result is new sequence of the same class as {\em sequence} containing the
elements of {\em sequence} from {\em start} up to but not including {\em end}.
%
\examples
\begin{tabular}{lcl}
    \verb+(slice '(a b c d) 1 3)+ & \Ra & \verb+(b c)+
\end{tabular}

\generic{sort}
%
\begin{genericargs}
    \item[sequence, \classref{sequence}] A sequence.
    \item[comparator, \classref{function}] A function.
\end{genericargs}
%
\result%
The result of sort is a new {\em sequence} comprising the elements of {\em
    sequence\/} ordered according to {\em comparator}.
%
\remarks%
Methods on this function are only defined for \classref{list} and
\classref{vector}.

\generic{sort!}
%
\begin{genericargs}
    \item[sequence, \classref{sequence}] A sequence.
    \item[comparator, \classref{function}] A function.
\end{genericargs}
%
\result%
Destructively sorts the elements of {\em sequence} (see \genericref{sort}) and
returns it.
%
\remarks%
Methods on this function are only defined for \classref{list} and
\classref{vector}.

\converter{list}
%
\begin{specargs}
    \item[collection, \classref{collection}] A collection to be converted into a
    list.
\end{specargs}
%
\result%
If {\em collection\/} is a list, the result is the argument.  Otherwise
a list is constructed and returned whose elements are the elements of {\em
    collection}.  If {\em collection\/} has natural order, then the elements
will appear in the result in the same order as in {\em collection}.  If {\em
    collection\/} does not have natural order, the order in the resulting list
is undefined.
%
\seealso%
Conversion (\ref{convert}).

\converter{string}
%
\begin{specargs}
    \item[collection, \classref{collection}] A collection to be converted into a
    string.
\end{specargs}
%
\result%
If {\em collection\/} is a string, the result is the argument.
Otherwise a string is constructed and returned whose elements are the elements
of {\em collection\/} as long as {\em all\/} the elements of {\em collection\/}
are characters.  An error is signalled (condition class: {\tt
    conversion-condition}\indexcondition{conversion-condition}) if any element
of {\em collection\/} is not a character.  If {\em collection\/} has natural
order, then the elements will appear in the result in the same order as in {\em
    collection}.  If {\em collection\/} does not have natural order, the order
in the resulting string is undefined.
%
\seealso%
Conversion (\ref{convert}).

\converter{table}
%
\begin{specargs}
    \item[collection, \classref{collection}] A collection to be converted into a
    table.
\end{specargs}
%
\result%
If {\em collection\/} is a table, the result is the argument.  Otherwise
a table is constructed and returned whose elements are the elements of {\em
    collection}.  If {\em collection\/} has natural order, then the elements
will be stored under integer keys in the range $[0\ldots{}size)$, otherwise the
keys used will be the keys associated with the elements of {\em collection}.
%
\seealso%
Conversion (\ref{convert}).

\converter{vector}
%
\begin{specargs}
    \item[collection, \classref{collection}] A collection to be converted into a
    vector.
\end{specargs}
%
\result%
If {\em collection\/} is a vector, the result is the argument.
Otherwise a vector is constructed and returned whose elements are the elements
of {\em collection}.  If {\em collection\/} has natural order, then the elements
will appear in the result in the same order as in {\em collection}.  If {\em
    collection\/} does not have natural order, the order in the resulting vector
is undefined.
%
\seealso%
Conversion (\ref{convert}).
%
\end{optDefinition}
