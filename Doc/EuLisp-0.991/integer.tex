\sclause{Integers}
\label{integer}
\index{general}{integer}
\index{general}{integer!module}
\index{general}{level-0 modules!integer}
%
\begin{optDefinition}
The defined name of this module is {\tt integer}.  This module defines
the abstract class \classref{integer}\ and the behaviour of some generic
functions on integers.  Further operations on numbers are defined in
the numbers module (\ref{number}).  A concrete integer class is
defined in the fixed precision integer module (\ref{fpi}).

\syntaxform{integer}

A positive integer\index{general}{external representation!integer} is has its
external representation as a sequence of digits optionally preceded by a plus
sign.  A negative integer\index{general}{external representation!integer} is
written as a sequence of digits preceded by a minus sign.  For example,
\verb+1234567890+, \verb+-456+, \verb-+1959-.

Integer literals have an external representation in any base up to base
36\index{general}{base}\index{general}{base!limitation on input}.  For
convenience, base 2\index{general}{binary
    literals}\index{general}{literal!binary}, base 8\index{general}{octal
    literals}\index{general}{literal!octal} and base
16\index{general}{hexadecimal literals}\index{general}{literal!hexadecimal} have
distinguished notations---\verb+#b+, \verb+#o+ and \verb+#x+, respectively.  For
example: \verb+1234+, \verb+#b10011010010+, \verb+#o2322+ and \verb+#x4d2+ all
denote the same value.

The general notation\index{general}{literal!arbitrary
    base}\index{general}{base!arbitrary base literals} for an arbitrary base is
\verb+#+{\em base\/}\verb+r+, where {\em base\/} is an unsigned decimal number.
Thus, the above examples may also be written: \verb+#10r1234+,
\verb+#2r10011010010+, \verb+#8r2322+, \verb+#16r4d2+ or \verb+#36rya+.  The
reading of any number is terminated on encountering a character which cannot be
a constituent of that number.  The syntax for the external representation of
integer literals is defined below.

\Syntax
\label{integer-syntax}
\defSyntax{integer}{
\begin{syntax}
    \scdef{integer}: \\
    \>  \scoptref{sign} \scref{unsigned-integer} \\
    \scdef{sign}: one of \\
    \>  + - \\
    \scdef{unsigned-integer}: \\
    \>   \scref{binary-integer} \\
    \>   \scref{octal-integer} \\
    \>   \scref{decimal-integer} \\
    \>   \scref{hexadecimal-integer} \\
    \>   \scref{specified-base-integer} \\
    \scdef{binary-integer}: \\
    \>  \#b \scSeqref{binary-digit} \\
    \scdef{binary-digit}: one of\\
    \>  0 1 \\
    \scdef{octal-integer}: \\
    \>  \#o \scSeqref{octal-digit} \\
    \scdef{octal-digit}: one of \\
    \>  0 1 2 3 4 5 6 7 \\
    \scdef{decimal-integer}: \\
    \>  \scSeqref{decimal-digit} \\
    \scdef{hexadecimal-integer}: \\
    \>  \#x \scSeqref{hexadecimal-digit} \\
    \scdef{hexadecimal-digit}: \\
    \>  \scref{decimal-digit} \\
    \>  \scref{hex-lower-letter} \\
    \>  \scref{hex-upper-letter} \\
    \scdef{hex-lower-letter}: one of \\
    \>  a b c d e f \\
    \scdef{hex-upper-letter}: one of \\
    \>  A B C D E F \\
    \scdef{specified-base-integer}: \\
    \>  \# \scref{base-specification} r \\
    \>  \scref{specified-base-digit} \\
    \>  \scseqref{specified-base-digit} \\
    \scdef{base-specification}: \\
    \>  \{ 2 | 3 | 4 | 5 | 6 | 7 | 8 | 9 \} \\
    \>  \{ 1 | 2 \} \scref{decimal-digit} \\
    \> 3 \{ 0 | 1 | 2 | 3 | 4 | 5 | 6 \} \\
    \scdef{specified-base-digit}: \\
    \>  \scref{decimal-digit} \\
    \>  \scref{letter}
\end{syntax}}%
\showSyntaxBox{integer}

\begin{note}
    At present this text does not define a class integer with variable
    precision.  It is planned this should appear in a future version at level-1
    of the language.  The class will be named
    \classref{variable-precision-integer}.  The syntax given here is applicable
    to both fixed and variable precision integers.
\end{note}

\derivedclass{integer}{number}
\index{general}{level-0 classes!\theclass{integer}}
%
The abstract class which is the superclass of all integer numbers.

\function{integer?}
%
\begin{arguments}
    \item[object] Object to examine.
\end{arguments}
%
\result
Returns {\em object} if it is an integer, otherwise \nil{}.

\generic{even?}
%
\begin{arguments}
    \item[integer, \classref{integer}] An integer.
\end{arguments}
%
\result
Returns \true\/ if two divides {\em integer}, otherwise \nil{}.

\function{odd?}
%
\begin{arguments}
    \item[integer] An integer.
\end{arguments}
%
\result
Returns the equivalent of the logical negation of \genericref{even?} applied to
{\em integer}.
%
\end{optDefinition}
