\sclause{Integers}
\label{integer}
\index{general}{integer}
\index{general}{integer!module}
\index{general}{level-0 modules!integer}
%
\begin{optDefinition}
The defined name of this module is {\tt integer}.  This module defines
the abstract class \classref{integer}\ and the behaviour of some generic
functions on integers.  Further operations on numbers are defined in
the numbers module (\ref{number}).  A concrete integer class is
defined in the fixed precision integer module (\ref{fpi}).

\syntaxform{integer}

A positive integer\index{general}{external representation!integer} is has its
external representation as a sequence of digits optionally preceded by a plus
sign.  A negative integer\index{general}{external representation!integer} is
written as a sequence of digits preceded by a minus sign.  For example,
\verb+1234567890+, \verb+-456+, \verb-+1959-.

Integer literals have an external representation in any base up to base
36\index{general}{base}\index{general}{base!limitation on input}.  For
convenience, base 2\index{general}{binary
    literals}\index{general}{literal!binary}, base 8\index{general}{octal
    literals}\index{general}{literal!octal} and base
16\index{general}{hexadecimal literals}\index{general}{literal!hexadecimal} have
distinguished notations---\verb+#b+, \verb+#o+ and \verb+#x+, respectively.  For
example: \verb+1234+, \verb+#b10011010010+, \verb+#o2322+ and \verb+#x4d2+ all
denote the same value.

The general notation\index{general}{literal!arbitrary
    base}\index{general}{base!arbitrary base literals} for an arbitrary base is
\verb+#+{\em base\/}\verb+r+, where {\em base\/} is an unsigned decimal number.
Thus, the above examples may also be written: \verb+#10r1234+,
\verb+#2r10011010010+, \verb+#8r2322+, \verb+#16r4d2+ or \verb+#36rya+.  The
reading of any number is terminated on encountering a character which cannot be
a constituent of that number.  The syntax for the external representation of
integer literals is defined below.

\Syntax
\label{integer-syntax}
\savesyntax\integerSyntax\vbox{
\syntax
integer
   = [sign] unsigned integer;
sign
   = '+' | '-';
unsigned integer
   = binary integer
   | octal integer
   | decimal integer
   | hexadecimal integer
   | specified base integer
binary integer
   = '#b', binary digit, {binary digit};
binary digit
   = '0' | '1';
octal integer
   = '#o', octal digit, {octal digit};
octal digit
   = '0' | '1' | '2' | '3' | '4' | '5' | '6' | '7';
decimal integer
   = decimal digit, {decimal digit};
hexadecimal integer
   = '#x', hexadecimal digit, {hexadecimal digit};
hexadecimal digit
   = decimal digit
   | hex lower letter
   | hex upper letter;
hex lower letter
   = 'a' | 'b' | 'c' | 'd' | 'e' | 'f';
hex upper letter
   = 'A' | 'B' | 'C' | 'D' | 'E' | 'F';
specified base integer
   = '#', base specification, 'r',
     specified base digit,
     {specified base digit};
base specification
   = decimal digit - ('0' | '1')
   | ( '1' | '2' ), decimal digit
   | '3', ('0' | '1' | '2' | '3' | '4' | '5' | '6');
specified base digit
   = decimal digit | letter;
\endsyntax}
\syntaxtable{integer}{\integerSyntax}

\begin{note}
    At present this text does not define a class integer with variable
    precision.  It is planned this should appear in a future version at level-1
    of the language.  The class will be named
    \classref{variable-precision-integer}.  The syntax given here is applicable
    to both fixed and variable precision integers.
\end{note}

\derivedclass{integer}{number}
\index{general}{level-0 classes!\theclass{integer}}
%
The abstract class which is the superclass of all integer numbers.

\function{integerp}
%
\begin{arguments}
    \item[object] Object to examine.
\end{arguments}
%
\result
Returns {\em object} if it is an integer, otherwise \nil{}.

\generic{evenp}
%
\begin{arguments}
    \item[integer, \classref{integer}] An integer.
\end{arguments}
%
\result
Returns \true\/ if two divides {\em integer}, otherwise \nil{}.

\function{oddp}
%
\begin{arguments}
    \item[integer] An integer.
\end{arguments}
%
\result
Returns the equivalent of the logical negation of \genericref{evenp} applied to
{\em integer}.
%
\end{optDefinition}
