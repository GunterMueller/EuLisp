\sclause{Object Introspection}
%
\begin{optDefinition}
\noindent
The only reflective capability which every object possesses is the ability to
find its class.

\function{class-of}
%
\begin{arguments}
    \item[object] An object.
\end{arguments}
%
\result%
The class of the object.
%
\remarks%
The function \functionref{class-of} can take any \lisp\ object as argument and
returns an instance of \classref{class} representing the class of that entity.
%
\end{optDefinition}

\sclause{Class Introspection}
%
\begin{optDefinition}
\noindent
Standard classes are not redefinable and support single inheritance only.
General multiple inheritance can be provided by extensions.  Nor is it possible
to use a class as a superclass which is not defined at the time of class
definition.  Again, such forward reference facilities can be provided by
extensions. The distinction between metaclasses and non-metaclasses is made
explicit by a special class, named \theclass{metaclass}, which is the class of
all metaclasses. This is different from ObjVlisp, where whether a class is a
metaclass depends on the superclass list of the class in question.  It is
implementation-defined whether \theclass{metaclass} itself is specializable or
not. This implies that implementations are free to restrict the instantiation
tree (excluding the selfinstantiation loop of \theclass{metaclass}) to a depth
of three levels.

The minimum information associated with a class metaobject is:
%
\begin{enumerate}
    \item The class precedence list, ordered most specific first, beginning with
    the class itself.
    \item The list of (effective) slot descriptions.
    \item The list of (effective) initargs.
\end{enumerate}
%
Standard classes support local slots only. Shared slots can be provided by
extensions.  The minimal information associated with a slot description
metaobject is:
%
\begin{enumerate}
    \item The name, which is required to perform inheritance computations.
    \item The initfunction, called by default to compute the initial slot value
    when creating a new instance.
    \item The reader, which is a function to read the corresponding slot value
    of an instance.
    \item The writer, which is a function to write the corresponding slot of an
    instance.
    \item The initarg, which is a symbol to access the value which can be
    supplied to a \functionref{make} call in order to initialize the
    corresponding slot in a newly created object.
\end{enumerate}
%
The metaobject classes defined for slot descriptions at level-1 are shown in
table~\ref{level-1-class-hierarchy}.
%
\begin{table}
\caption{Level-1 class hierarchy}
\label{level-1-class-hierarchy}
{\tt
\begin{tabbing}
    00\=00\=00\=00\= \kill
    \classref{class} \\
    % not in youtoo \>\classref{built-in-class} \\
    \>\classref{simple-class} \\
    \>\classref{function-class}  \\
    \classref{slot} \\
    \>\classref{local-slot} \\
    \classref{function} \\
    \>\classref{generic-function} \\
    \>\>\classref{simple-generic-function} \\
    \classref{method} \\
    \>\classref{simple-method}
\end{tabbing}
}
\end{table}

\function{class-precedence-list}
%
\begin{arguments}
    \item[class] A class.
\end{arguments}
%
\result%
A list of classes, starting with {\em class} itself, succeeded by the
superclasses of {\em class} and ending with \classref{object}. This list is
equivalent to the result of calling \functionref{compute-class-precedence-list}.
%
\remarks%
The class precedence list is used to control the inheritance of slots and
methods.

\function{class-slots}
%
\begin{arguments}
    \item[class] A class.
\end{arguments}
%
\result%
A list of slots, one for each of the slots of an instance
of {\em class}.
%
\remarks%
The slots determine the instance size (number of slots)
and the slot access.

\function{class-initargs}
%
\begin{arguments}
    \item[class] A class.
\end{arguments}
%
\result%
A list of symbols, which can be used as legal keywords to initialize
instances of the class.
%
\remarks%
The initargs correspond to the keywords specified in the {\tt initarg}
slot-option or the {\tt initargs} class-option when the
class and its superclasses were defined.
%
\end{optDefinition}

\sclause{Slot Introspection}
%
\begin{optDefinition}
\class{slot}

The abstract class of all slot descriptions.

\derivedclass{local-slot}{slot}

The class of all local slot descriptions.
%
\begin{initoptions}
    \item[name, string] The name of the slot.%; useful for debugging only.
    \item[reader, function] The function to access the slot.
    \item[writer, function] The function to update the slot.
    \item[initfunction, function] The function to compute the initial value in
    the absence of a supplied value.
    \item[initarg, symbol] The key to access a supplied initial value.
\end{initoptions}

The default value for all initoptions is \nil{}.

\function{slot-name}
%
\begin{arguments}
    \item[slot] A slot description.
\end{arguments}
%
\result%
The symbol which was used to name the slot when the class, of which
the slot is part, was defined.
%
\remarks%
The slot description name is used to identify a slot description in a
class. It has no effect on bindings.

\function{slot-initfunction}
%
\begin{arguments}
    \item[slot] A slot description.
\end{arguments}
%
\result%
A function of no arguments that is used to compute the initial value of the
slot in the absence of a supplied value.

\function{slot-slot-reader}
%
\begin{arguments}
    \item[slot] A slot description.
\end{arguments}
%
\result%
A function of one argument that returns the value of the slot
in that argument.

\function{slot-slot-writer}
%
\begin{arguments}
    \item[slot] A slot description.
\end{arguments}
%
\result%
A function of two arguments that installs the second argument as the value of
the slot in the first argument.
\end{optDefinition}

\sclause{Generic Function Introspection}
%
\begin{optDefinition}
The default generic dispatch scheme is class-based; that is, methods
are class specific.  The default argument precedence order is
left-to-right.

The minimum information associated with a generic function metaobject is:

\begin{enumerate}
    \item The domain, restricting the domain of each added method to a
    sub-domain.
    \item The method class, restricting each added method to be an instance
    of that class.
    \item The list of all added methods.
    \item The method look-up function used to collect and sort the
    applicable methods for a given domain.
    \item The discriminating function used to perform the generic dispatch.
\end{enumerate}

\function{generic-function-domain}
%
\begin{arguments}
    \item[generic-function] A generic function.
\end{arguments}
%
\result%
A list of classes.
%
\remarks%
This function returns the domain of a generic function. The domains of all
methods attached to a generic function are constrained to be within this domain.
In other words, the domain classes of each method must be subclasses of the
corresponding generic function domain class.  It is an error to modify this
list.

\function{generic-function-method-class}
%
\begin{arguments}
    \item[generic-function] A generic function.
\end{arguments}
%
\result%
This function returns the class which is the class of all methods of the generic
function.  Each method attached to a generic function must be an instance of
this class. When a method is created using \macroref{defmethod},
\macroref{method-lambda}, or by using the {\tt method} generic function
option in a \defopref{defgeneric} or \macroref{generic-lambda}, it will be an
instance of this class by default.

\function{generic-function-methods}
%
\begin{arguments}
    \item[generic-function] A generic function.
\end{arguments}
%
\result%
This function returns a list of the methods attached to the generic function.
The order of the methods in this list is undefined.  It is an error to modify
this list.

\function{generic-function-method-lookup-function}
%
\begin{arguments}
    \item[generic-function] A generic function.
\end{arguments}
%
\result%
A function.
%
\remarks%
This function returns a function which, when applied to the arguments given to
the generic function, returns a sorted list of applicable methods.  The order of
the methods in this list is determined by
\genericref{compute-method-lookup-function}.
%How many arguments does this function take --- 1 or n?  hb: n

\function{generic-function-discriminating-function}
%
\begin{arguments}
    \item[generic-function] A generic function.
\end{arguments}
%
\result%
A function.
%
\remarks%
This function returns a function which may be applied to the same arguments as
the generic function.  This function is called to perform the generic dispatch
operation to determine the applicable methods whenever the generic function is
called, and call the most specific applicable method function.  This function is
created by \genericref{compute-discriminating-function}.

\end{optDefinition}

\sclause{Method Introspection}
\begin{optDefinition}
The minimal information associated with a method metaobject is:

\begin{enumerate}
    \item The domain, which is a list of classes.
    \item The function comprising the code of the method.
    \item The generic function to which the method has been added, or \nil{}\/ if
    it is attached to no generic function.
\end{enumerate}
%
The metaobject classes for generic functions defined at level-1 are
shown in table~\ref{level-1-class-hierarchy}.

\function{method-domain}
%
\begin{arguments}
    \item[method] A method.
\end{arguments}
%
\result%
A list of classes defining the domain of a method.

\function{method-function}
%
\begin{arguments}
    \item[method] A method.
\end{arguments}
%
\result%
This function returns a function which implements the method.  The returned
function which is called when {\em method} is called, either by calling the
generic function with appropriate arguments, through a
\specopref{call-next-method}, or by using \functionref{call-method}.  A method
metaobject itself cannot be applied or called as a function.

\setter{method-function}
%
\begin{arguments}
    \item[method] A method.
    \item[function] A function.
\end{arguments}
%
\result%
This function sets the function which implements the method.

\function{method-generic-function}
%
\begin{arguments}
    \item[method] A method.
\end{arguments}
%
\result%
This function returns the generic function to which {\em method} is attached; if
{\em method} is not attached to a generic function, it returns \nil{}.
\end{optDefinition}

\sclause{Class Initialization}
%
\begin{optDefinition}
\method{initialize}
%
\begin{specargs}
    \item[class, \classref{class}] A class.
    \item[initlist, \classref{list}] A list of initialization options as
    follows:

    \begin{options}
        \item[name, symbol] Name of the class being initialized.
        \item[direct-superclasses, list] List of direct superclasses.
        \item[direct-slots, list] List of direct slot specifications.
        \item[direct-initargs, list] List of direct initargs.
    \end{options}
\end{specargs}
%
\result%
The initialized class.
%
\remarks%
The initialization of a class takes place as follows:
%
\begin{enumerate}
    \item Check compatibility of direct superclasses
    % \item Transform textual slot descriptions into slot description objects
    % (this is better done under the control of class beeing initialized).

    \item Perform the logical inheritance computations of:
    \begin{enumerate}
        \item class precedence list
        \item initargs
        \item slot descriptions
    \end{enumerate}

    \item Compute new slot accessors and ensure all (new and inherited)
    accessors to work correctly on instances of the new class.

    % \item Compute predicates and constructors if required.

    \item Make the results accessible by class readers.
\end{enumerate}
%
The basic call structure is laid out in figure~\ref{call-structure}
%
\begin{optPrivate}
\begin{verbatim}
COMPATIBLE-SUPERCLASSES-P cl direct-superclasses -> boolean
  COMPATIBLE-SUPERCLASS-P cl superclass -> boolean
COMPUTE-CLASS-PRECEDENCE-LIST
cl direct-superclasses -> list(class)
COMPUTE-INHERITED-INITARGS
cl direct-superclasses -> list(list(initarg))
COMPUTE-INITARGS
cl direct-initargs <inherited-initargs> -> list(initarg)
COMPUTE-INHERITED-SLOTS
cl direct-superclasses -> list(list(slotd))
COMPUTE-SLOTS
cl direct-slotds <inherited-slotds> -> list(slotd)
  COMPUTE-DEFINED-SLOT
  cl slotd-init-list -> slotd
    COMPUTE-DEFINED-SLOT-CLASS
    cl slotd-init-list -> slotd-class
  -or-
  COMPUTE-SPECIALIZED-SLOT
  cl inherited-slotds slotd-init-list -> slotd
    COMPUTE-SPECIALIZED-SLOT-CLASS
    cl inherited-slotds slotd-init-list -> slotd-class
COMPUTE-AND-ENSURE-SLOT-ACCESSORS
cl <effective-slotds> <inherited-slotds> -> list(slotd)
  COMPUTE-SLOT-READER
  cl slotd <effective-slotds> -> function
  COMPUTE-SLOT-WRITER
  cl slotd <effective-slotds> -> function
  ENSURE-SLOT-READER
  cl slotd <effective-slotds> reader -> function
    COMPUTE-PRIMITIVE-READER-USING-SLOT
    slotd cl <effective-slotds> -> function
      COMPUTE-PRIMITIVE-READER-USING-CLASS
      cl slotd <effective-slotds> -> function
  ENSURE-SLOT-WRITER cl slotd writer -> function
    COMPUTE-PRIMITIVE-WRITER-USING-SLOT
    slotd cl <effective-slotds> -> function
      COMPUTE-PRIMITIVE-WRITER-USING-CLASS
      cl slotd <effective-slotds> -> function
\end{verbatim}
\end{optPrivate}
%
\begin{table}%
\caption{Initialization Call Structure}%
\label{call-structure}%
\tt%
\begin{tabbing}%
    00\=00\=00\=00\=00\=00\=00\=00\= \kill
    {\tt compatible-superclasses-p}\\
    \>\>\>\>{\em cl direct-superclasses\/}
    $\rightarrow$ {\em boolean}\\
    \>{\tt compatible-superclass-p}\\
    \>\>\>\>\> {\em cl superclass\/}
    $\rightarrow$ {\em boolean}\\
    %\hline
    {\tt compute-class-precedence-list}\\
    \>\>\>\>{\em cl direct-superclasses\/}
    $\rightarrow$ ({\em cl\/}$^*$)\\
    %\hline
    {\tt compute-inherited-initargs}\\
    \>\>\>\>{\em cl direct-superclasses\/}
    $\rightarrow$ (({\em initarg\/}$^*$)$^*$)\\
    %\hline
    {\tt compute-initargs}\\
    \>\>\>\>{\em cl direct-initargs inherited-initargs\/}\\
    \>\>\>\>$\rightarrow$ ({\em initarg\/}$^*$)\\
    %\hline
    {\tt compute-inherited-slots}\\
    \>\>\>\>{\em cl direct-superclasses\/}
    $\rightarrow$ (({\em sd\/}$^*$)$^*$)\\
    %\hline
    {\tt compute-slots}\\
    \>\>\>\>{\em cl slot-specs inherited-sds\/}
    $\rightarrow$ ({\em sd\/}$^*$)\\
    \>{\em either}\\
    \>\>{\tt compute-defined-slot}\\
    \>\>\>\>\>\>{\em cl slot-spec\/}
    $\rightarrow$ {\em sd}\\
    \>\>\>{\tt compute-defined-slot-description-class}\\
    \>\>\>\>\>\>\>{\em cl slot-spec\/}
    $\rightarrow$ {\em sd-class}\\
    \>{\em or}\\
    \>\>{\tt compute-specialized-slot}\\
    \>\>\>\>\>\>{\em cl inherited-sds slot-spec\/}
    $\rightarrow$ {\em sd}\\
    \>\>\>{\tt compute-specialized-slot-class}\\
    \>\>\>\>\>\>\>{\em cl inherited-sds slot-spec\/}\\
    \>\>\>\>\>\>\>$\rightarrow$ {\em sd-class}\\
    %\hline
    {\tt compute-instance-size}\\
    \>\>\>\>{\em cl effective-sds\/}
    $\rightarrow$ {\em integer}\\
    %\hline
    {\tt compute-and-ensure-slot-accessors}\\
    \>\>\>\>{\em cl effective-sds inherited-sds\/}
    $\rightarrow$ ({\em sd\/}$^*$)\\
    \>{\tt compute-slot-reader}\\
    \>\>\>\>\>{\em cl sd effective-sds\/}
    $\rightarrow$ {\em function}\\
    \>{\tt compute-slot-writer}\\
    \>\>\>\>\>{\em cl sd effective-sds\/}
    $\rightarrow$ {\em function}\\
    \>{\tt ensure-slot-reader}\\
    \>\>\>\>\>{\em cl sd effective-sds reader\/}
    $\rightarrow$ {\em function}\\
    \>\>{\tt compute-primitive-reader-using-slot}\\
    \>\>\>\>\>\>{\em sd cl effective-sds\/}
    $\rightarrow$ {\em function}\\
    \>\>\>{\tt compute-primitive-reader-using-class}\\
    \>\>\>\>\>\>\>{\em cl sd effective-sds\/}
    $\rightarrow$ {\em function}\\
    \>{\tt ensure-slot-writer}\\
    \>\>\>\>\>{\em cl sd effective-sds writer\/}
    $\rightarrow$ {\em function}\\
    \>\>{\tt compute-primitive-writer-using-slot}\\
    \>\>\>\>\>\>{\em sd cl effective-sds\/}
    $\rightarrow$ {\em function}\\
    \>\>\>{\tt compute-primitive-writer-using-class}\\
    \>\>\>\>\>\>\>{\em cl sd effective-sds\/}
    $\rightarrow$ {\em function}%
\end{tabbing}%
\end{table}%
%
Note that \genericref{compute-initargs} is called by the default
\genericref{initialize} method with all direct initargs as the second argument:
those specified as slot option and those specified as class option.

\generic{compute-predicate}
%
\begin{genericargs}
    \item[class, \classref{class}] A class.
\end{genericargs}
%
\result%
Computes and returns a function of one argument, which is a predicate
function for {\em class}.

\method{compute-predicate}
%
\begin{specargs}
    \item[class, \classref{class}] A class.
\end{specargs}
%
\result%
Computes and returns a function of one argument, which returns true when applied
to direct or indirect instances of {\em class} and \nil{}\/ otherwise.

\generic{compute-constructor}
%
\begin{genericargs}
    \item[class, \classref{class}] A class.
    \item[parameters, \classref{list}] The argument list of the function being
    created.
\end{genericargs}
%
\result%
Computes and returns a constructor function for {\em class}.

\method{compute-constructor}
%
\begin{specargs}
    \item[class, \classref{class}] A class.
    \item[parameters, \classref{list}] The argument list of the function being
    created.
\end{specargs}
%
\result%
Computes and returns a constructor function, which returns a new
instance of {\em class}.

\generic{allocate}
%
\begin{genericargs}
    \item[class, \classref{class}] A class.
    \item[initlist, \classref{list}] A list of initialization arguments.
\end{genericargs}
%
\result%
An instance of the first argument.
%
\remarks%
Creates an instance of the first argument.  Users may define new
methods for new metaclasses.

\method{allocate}
%
\begin{specargs}
    \item[class, \classref{class}] A class.
    \item[initlist, \classref{list}] A list of initialization arguments.
\end{specargs}
%
\result%
An instance of the first argument.
%
\remarks%
The default method creates a new uninitialized instance of the first
argument.  The initlist is not used by this \genericref{allocate} method.
%
\end{optDefinition}

\sclause{Slot Description Initialization}
%
\begin{optDefinition}
%
\method{initialize}
%
\begin{specargs}
    \item[slot, \classref{slot}] A slot description.
    \item[initlist, \classref{list}] A list of initialization options as
    follows:
    \begin{options}
        \item[name, symbol] The name of the slot.
        \item[initfunction, function] A function.
        \item[initarg, symbol] A symbol.
        \item[reader, function] A slot reader function.
        \item[writer, function] A slot writer function.
    \end{options}
    % \item[initlist] A list of initialization arguments.
\end{specargs}
%
\result%
The initialized slot description.
%
\end{optDefinition}

\sclause{Generic Function Initialization}
%
\begin{optDefinition}

\method{initialize}
%
\begin{specargs}
    \item[gf, \classref{generic-function}] A generic function.

    \item[initlist, \classref{list}] A list of initialization options as
    follows:
    \begin{options}
        \item[name, symbol] The name of the generic function.
        \item[domain, list] List of argument classes.
        % \item[range, class] The class of the result.
        \item[method-class, class] Class of attached methods.
        \item[method, method-description] A method to be attached.  This
        option may be specified more than once.
        % hb: no!
    \end{options}
\end{specargs}
%
\result%
The initialized generic function.
%
\remarks%
This method initializes and returns the {\em generic-function}.  The specified
methods are attached to the generic function by \genericref{add-method}, and its
slots are initialized from the information passed in {\em initlist} and from the
results of calling \genericref{compute-method-lookup-function} and
\genericref{compute-discriminating-function} on the generic function.  Note that
these two functions may not be called during the call to
\genericref{initialize}, and that they may be called several times for the
generic function.

The basic call structure is:
\smallskip
\noindent
{\tt%
    add-method {\em gf method\/} -> {\em gf}\\
    compute-method-lookup-function {\em gf domain \/} -> {\em function}\\
    compute-discriminating-function {\em gf domain lookup-fn methods\/} -> {\em function}}
\smallskip
%
\end{optDefinition}

\sclause{Method Initialization}
%
\begin{optDefinition}
%
\method{initialize}
%
\begin{specargs}
    \item[method, \classref{method}] A method.

    \item[initlist, \classref{list}] A list of initialization options as
    follows:
    \begin{options}
        \item[domain, list] The list of argument classes.
        % \item[range, class] The class of the result.
        \item[function, fn] A function, created with
        \macroref{method-function-lambda}.
        \item[generic-function, gf] A generic function.
    \end{options}
\end{specargs}
%
\result%
This method returns the initialized method metaobject {\em method}.  If the {\em
    generic-function\/} option is supplied, \genericref{add-method} is called to
install the new method in the {\em generic-function}.
%
\end{optDefinition}

\sclause{Inheritance Protocol}
%
\begin{optDefinition}
%
\generic{compatible-superclasses-p}
%
\begin{genericargs}
    \item[class, \classref{class}] A class.
    \item[direct-superclasses, \classref{list}] A list of potential direct
    superclasses of {\em class}.
\end{genericargs}
%
\result%
Returns {\em t} if {\em class} is compatible with {\em
direct-superclasses}, otherwise \nil{}.

\method{compatible-superclasses-p}
%
\begin{specargs}
    \item[class, \classref{class}] A class.
    \item[direct-superclasses, \classref{list}] A list of potential direct
    superclasses.
\end{specargs}
%
\result%
Returns the result of calling \genericref{compatible-superclass-p} on {\em
    class} and the first element of the {\em direct-superclasses} (single
inheritance assumption).

\generic{compatible-superclass-p}
%
\begin{genericargs}
    \item[subclass, \classref{class}] A class.
    \item[superclass, \classref{class}] A potential direct superclass.
\end{genericargs}
%
\result%
Returns \true\/ if {\em subclass} is compatible with {\em superclass},
otherwise \nil{}.

\method{compatible-superclass-p}
%
\begin{specargs}
    \item[subclass, \classref{class}] A class.
    \item[superclass, \classref{class}] A potential direct superclass.
\end{specargs}
%
\result%
Returns \true\/ if the class of the first argument is a subclass of
the class of the second argument, otherwise \nil{}.

If the implementation wishes to restrict the instantiation tree (see
introduction to B.4), this method should return \nil{}\/ if {\em superclass}
is \theclass{metaclass}.

\method{compatible-superclass-p}
%
\begin{specargs}
    \item[subclass, \classref{class}] A class.
    \item[superclass, \theclass{abstract-class}] A potential direct superclass.
\end{specargs}
%
\result%
Always returns \true.

\method{compatible-superclass-p}
%
\begin{specargs}
    \item[subclass, \theclass{abstract-class}] A class.
    \item[superclass, \classref{class}] A potential direct superclass.
\end{specargs}
%
\result%
Always returns \nil{}.

\method{compatible-superclass-p}
%
\begin{specargs}
    \item[subclass, \theclass{abstract-class}] A class.
    \item[superclass, \theclass{abstract-class}] A potential direct superclass.
\end{specargs}
%
\result%
Always returns \true.

\generic{compute-class-precedence-list}
%
\begin{genericargs}
    \item[class, \classref{class}] Class being defined.
    \item[direct-superclasses, \classref{list}] List of direct superclasses.
\end{genericargs}
%
\result%
Computes and returns a list of classes which represents the linearized
inheritance hierarchy of {\em class} and the given list of direct superclasses,
beginning with {\em class} and ending with \classref{object}.

\method{compute-class-precedence-list}
%
\begin{specargs}
    \item[class, \classref{class}] Class being defined.
    \item[direct-superclasses, \classref{list}] List of direct superclasses.
\end{specargs}
%
\result%
A list of classes.
%
\remarks%
This method can be considered to return a
cons of {\em class} and the class precedence list of the first element
of {\em direct-superclasses} (single inheritance assumption). If no {\em
direct-superclasses} has been supplied, the result is the list of two elements:
{\em class} and \classref{object}.

\generic{compute-slots}
%
\begin{genericargs}
    \item[class, \classref{class}] Class being defined.
    \item[direct-slot-specifications, \classref{list}] A list of direct slot
    specification.
    \item[inherited-slots, \classref{list}] A list of lists of inherited slot
    descriptions.
\end{genericargs}
%
\result%
Computes and returns the list of effective slot descriptions of {\em class}.
%
\seealso%
\genericref{compute-inherited-slots}.

\method{compute-slots}
%
\begin{specargs}
    \item[class, \classref{class}] Class being defined.
    \item[slot-specs, \classref{list}] List of (direct) slot specifications.
    \item[inherited-slot-lists, \classref{list}] A list of lists (in
    fact one list in single inheritance) of inherited slot descriptions.
\end{specargs}
%
\result%
A list of effective slot descriptions.
%
\remarks%
% Update!
The default method computes two sublists:
\begin{enumerate}
    \item Calling \genericref{compute-specialized-slot} with the
    three arguments (i) {\em class}, (ii) each {\em inherited-slot}
    as a singleton list, (iii) the {\em slot-spec} corresponding (by having the
    same name) to the slot description, if it exists, otherwise \nil{}, giving a
    list of the specialized slot descriptions.

    \item Calling \genericref{compute-defined-slot} with the three
    arguments (i) {\em class}, (ii) each {\em slot-specification} which does not
    have a corresponding (by having the same name) {\em
        inherited-slot}.
\end{enumerate}
%
The method returns the concatenation of these two lists as its result.  The
order of elements in the list is significant. All specialized slot descriptions
have the same position as in the effective slot descriptions list of the direct
superclass (due to the single inheritance). The slot accessors (computed later)
may rely on this assumption minimizing the number of methods to one for all
subclasses and minimizing the access time to an indexed reference.
%
\seealso%
\genericref{compute-specialized-slot},
\genericref{compute-defined-slot},
\genericref{compute-and-ensure-slot-accessors}.

\generic{compute-initargs}
%
\begin{genericargs}
    \item[class, \classref{class}] Class being defined.
    \item[initargs, \classref{list}] List of direct initargs.
    \item[inherited-initarg-lists, \classref{list}] A list of lists of inherited
    initargs.
\end{genericargs}
%
\result%
List of symbols.
%
\remarks%
Computes and returns all legal initargs for {\em class}.
%
\seealso%
\genericref{compute-inherited-initargs}.

\method{compute-initargs}
%
\begin{specargs}
    \item[class, \classref{class}] Class being defined.
    \item[initargs, \classref{list}] List of direct initargs.
    \item[inherited-initarg-lists, \classref{list}] A list of lists of inherited
    initargs.
\end{specargs}
%
\result%
List of symbols.
%
\remarks%
This method appends the second argument with the first element of the third
argument (single inheritance assumption), removes duplicates and returns the
result. Note that \genericref{compute-initargs} is called by the default
\genericref{initialize} method with all direct initargs as the second argument:
those specified as slot option and those specified as class option.

\generic{compute-inherited-slots}
%
\begin{genericargs}
    \item[class, \classref{class}] Class being defined.
    \item[direct-superclasses, \classref{list}] List of direct superclasses.
\end{genericargs}
%
\result%
List of lists of inherited slot descriptions.
%
\remarks%
Computes and returns a list of lists of effective slot descriptions.
%
\seealso%
\genericref{compute-slots}.

\method{compute-inherited-slots}
%
\begin{specargs}
    \item[class, \classref{class}] Class being defined.
    \item[direct-superclasses, \classref{list}] List of direct superclasses.
\end{specargs}
%
\result%
List of lists of inherited slot descriptions.
%
\remarks%
The result of the default method is a list of one element: a list of effective
slot descriptions of the first element of the second argument (single
inheritance assumption). Its result is used by
\genericref{compute-slots} as an argument.

\generic{compute-inherited-initargs}
%
\begin{genericargs}
    \item[class, \classref{class}] Class being defined.
    \item[direct-superclasses, \classref{list}] List of direct superclasses.
\end{genericargs}
%
\result%
List of lists of symbols.
\remarks%
Computes and returns a list of lists of initargs. Its result is used by
\genericref{compute-initargs} as an argument.
%
\seealso%
\genericref{compute-initargs}.

\method{compute-inherited-initargs}
%
\begin{specargs}
    \item[class, \classref{class}] Class being defined.
    \item[direct-superclasses, \classref{list}] List of direct superclasses.
\end{specargs}
%
\result%
List of lists of symbols.
%
\remarks%
The result of the default method
contains one list of legal initargs of the first element of the second
argument (single inheritance assumption).

\generic{compute-defined-slot}
%
\begin{genericargs}
    \item[class, \classref{class}] Class being defined.
    \item[slot-spec, \classref{list}] Canonicalized slot specification.
\end{genericargs}
%
\result%
Slot description.
%
\remarks%
Computes and returns a new effective slot description.  It is called by
\genericref{compute-slots} on each slot specification which has no
corresponding inherited slot descriptions.  \seealso%
\genericref{compute-defined-slot-class}.

\method{compute-defined-slot}
%
\begin{specargs}
    \item[class, \classref{class}] Class being defined.
    \item[slot-spec, \classref{list}] Canonicalized slot specification.
\end{specargs}
%
\result%
Slot description.
%
\remarks%
Computes and returns a new effective slot description.  The class of the result
is determined by calling \genericref{compute-defined-slot-class}.
%
\seealso%
\genericref{compute-defined-slot-class}.

\generic{compute-defined-slot-class}
%
\begin{genericargs}
    \item[class, \classref{class}] Class being defined.
    \item[slot-spec, \classref{list}] Canonicalized slot specification.
\end{genericargs}
%
\result%
Slot description class.
%
\remarks%
Determines and returns the slot description class corresponding to {\em class}
and {\em slot-spec} .  \seealso \genericref{compute-defined-slot}.

\method{compute-defined-slot-class}
%
\begin{specargs}
    \item[class, \classref{class}] Class being defined.
    \item[slot-spec, \classref{list}] Canonicalized slot specification.
\end{specargs}
%
\result%
The class \classref{local-slot}.
%
\remarks%
This method just returns the class \classref{local-slot}.

\generic{compute-specialized-slot}
%
\begin{genericargs}
    \item[class, \classref{class}] Class being defined.
    \item[inherited-slots, \classref{list}] List of inherited slot descriptions
    (each of the same name as the slot being defined).
    \item[slot-spec, \classref{list}] Canonicalized slot specification or
    \nil{}.
\end{genericargs}
%
\result%
Slot description.
%
\remarks%
Computes and returns a new effective slot description. It is called by
\genericref{compute-slots} on the class, each list of inherited
slots with the same name and with the specialising slot specification
list or \nil{}\/ if no one is specified with the same name.
%
\seealso%
\genericref{compute-specialized-slot-class}.

\method{compute-specialized-slot}
%
\begin{specargs}
    \item[class, \classref{class}] Class being defined.
    \item[inherited-slots, \classref{list}] List of inherited slot descriptions.
    \item[slot-spec, \classref{list}] Canonicalized sdirect-lot specification or
    \nil{}.
\end{specargs}
%
\result%
Slot description.
%
\remarks%
Computes and returns a new effective slot description.  The class of the result
is determined by calling
\genericref{compute-specialized-slot-class}.
%
\seealso%
\genericref{compute-specialized-slot-class}.

\generic{compute-specialized-slot-class}
%
\begin{genericargs}
    \item[class, \classref{class}] Class being defined.
    \item[inherited-slots, \classref{list}] List of inherited slot descriptions.
    \item[slot-spec, \classref{list}] Canonicalized slot specification or
    \nil{}.
\end{genericargs}
%
\result%
Slot description class.
%
\remarks%
Determines and returns the slot description class corresponding to (i)
the class being defined, (ii) the inherited slot descriptions being
specialized (iii) the specializing information in {\em slot-spec}.
%
\seealso%
\genericref{compute-specialized-slot}.

\method{compute-specialized-slot-class}
%
\begin{specargs}
    \item[class, \classref{class}] Class being defined.
    \item[inherited-slots, \classref{list}] List of inherited slot
    descriptions.
    \item[slot-spec, \classref{list}] Canonicalized slot specification or
    \nil{}.
\end{specargs}
%
\result%
The class \classref{local-slot}.
%
\remarks%
This method just returns the class \classref{local-slot}.
%
\end{optDefinition}

\sclause{Slot Access Protocol}
%
\begin{optDefinition}
The slot access protocol is defined via accessors (readers and writers)
only. There is no primitive like CLOS's {\tt slot-value}. The accessors are
generic for standard classes, since they have to work on subclasses and should
do the applicability check anyway.  The key idea is that the discrimination on
slots and classes is performed once at class definition time rather
than again and again at slot access time.

Each slot has exactly one reader and one writer as anonymous
objects. If a reader/writer slot-option is specified in a class definition, the
anonymous reader/writer of that slot is bound to the specified
identifier. Thus, if a reader/writer option is specified more than once, the
same object is bound to all the identifiers. If the accessor slot-option is
specified the anonymous writer will be installed as the setter of the reader.
Specialized slots refer to the same objects as those in the
superclasses (single inheritance makes that possible).  Since the
readers/writers are generic, it is possible for a subclass (at the meta-level)
to add new methods for inherited slots in order to make the
readers/writers applicable on instances of the subclass. A new method might be
necessary if the subclasses have a different instance allocation or if the slot
positions cannot be kept the same as in the superclass (in multiple inheritance
extensions).  This can be done during the initialization computations.

\generic{compute-and-ensure-slot-accessors}
%
\begin{genericargs}
    \item[class, \classref{class}] Class being defined.
    \item[slots, \classref{list}] List of effective slot
    descriptions.
    \item[inherited-slots, \classref{list}] List of lists of
    inherited slot descriptions.
\end{genericargs}
%
\result%
List of effective slot descriptions.
%
\remarks%
Computes new accessors or ensures that inherited accessors work
correctly for each effective slot description.

\method{compute-and-ensure-slot-accessors}
%
\begin{specargs}
    \item[class, \classref{class}] Class being defined.
    \item[slots, \classref{list}] List of effective slot
    descriptions.
    \item[inherited-slots, \classref{list}] List of lists of
    inherited slot descriptions.
\end{specargs}
%
\result%
List of effective slot descriptions.
%
\remarks%
For each slot description in {\em slots\/} the default method
checks if it is a new slot description and not an
inherited one. If the slot description is new,
\begin{enumerate}
    % \item computes the slot position in an instance
    % calling \genericref{compute-slot-position}, and stores the result;

    \item calls \genericref{compute-slot-reader} to compute a new slot reader
    and stores the result in the slot description;

    \item calls \genericref{compute-slot-writer} to compute a new slot writer
    and stores the result in the slot description;
\end{enumerate}
%
Otherwise, it assumes that the inherited values remain valid.

Finally, for every slot description (new or inherited) it ensures the reader and
writer work correctly on instances of {\em class} by means of
\genericref{ensure-slot-reader} and \genericref{ensure-slot-writer}.

\generic{compute-slot-reader}
%
\begin{genericargs}
    \item[class, \classref{class}] Class.
    \item[slot, \classref{slot}] Slot description.
    \item[slot-list, \classref{list}] List of effective slot descriptions.
\end{genericargs}
%
\result%
Function.
%
\remarks%
Computes and returns a new slot reader applicable to instances of {\em class}
returning the slot value corresponding to {\em slot}. The third
argument can be used in order to compute the logical slot position.

\method{compute-slot-reader}
%
\begin{specargs}
    \item[class, \classref{class}] Class.
    \item[slot, \classref{slot}] Slot description.
    \item[slots, \classref{list}] List of effective slot
    descriptions.
\end{specargs}
%
\result%
Generic function.
%
\remarks%
The default method returns a new generic function of one argument
without any methods. Its domain is {\em class}.

\generic{compute-slot-writer}
%
\begin{genericargs}
    \item[class, \classref{class}] Class.
    \item[slot, \classref{slot}] Slot description.
    \item[slots, \classref{list}] List of effective slot
    descriptions.
\end{genericargs}
%
\result%
Function.
%
\remarks%
Computes and returns a new slot writer applicable to instances of {\em
class} and any value to be stored as the new slot value corresponding
to {\em slot}. The third argument can be used in order to
compute the logical slot position.

\method{compute-slot-writer}
%
\begin{specargs}
    \item[class, \classref{class}] Class.
    \item[slot, \classref{slot}] Slot description.
    \item[slots, \classref{list}] List of effective slot
    descriptions.
\end{specargs}
%
\result%
Generic function.
%
\remarks%
The default method returns a new generic function of two arguments
without any methods. Its domain is {\em class} $\times$ \classref{object}.

\generic{ensure-slot-reader}
%
\begin{genericargs}
    \item[class, \classref{class}] Class.
    \item[slot, \classref{slot}] Slot description.
    \item[slots, \classref{list}] List of effective slot
    descriptions.
    \item[reader, \classref{function}] The slot reader.
\end{genericargs}
%
\result%
Function.
%
\remarks%
Ensures {\em function} correctly fetches the value of the slot from
instances of {\em class}.

\method{ensure-slot-reader}
%
\begin{specargs}
    \item[class, \classref{class}] Class.
    \item[slot, \classref{slot}] Slot description.
    \item[slots, \classref{list}] List of effective slot descriptions.
    \item[reader, \classref{generic-function}] The slot reader.
\end{specargs}
%
\result%
Generic function.
%
\remarks%
The default method checks if there is a method in the {\em
generic-function}. If not, it creates and adds a new one, otherwise it
assumes that the existing method works correctly. The domain
of the new method is {\em class} and the function is
{\tt
\begin{tabbing}
    00\=00\=00\=00\= \kill
    (method-function-lambda ((object {\em class}))\\
    \>(primitive-reader object))
\end{tabbing}
}
\genericref{compute-primitive-reader-using-slot} is called by
\genericref{ensure-slot-reader} method to compute the primitive reader used
in the function of the new created reader method.

\generic{ensure-slot-writer}
%
\begin{genericargs}
    \item[class, \classref{class}] Class.
    \item[slot, \classref{slot}] Slot description.
    \item[slots, \classref{list}] List of effective slot
    descriptions.
    \item[writer, \classref{function}] The slot writer.
\end{genericargs}
%
\result%
Function.
%
\remarks%
Ensures {\em function} correctly updates the value of the slot in
instances of {\em class}.

\method{ensure-slot-writer}
%
\begin{specargs}
    \item[class, \classref{class}] Class.
    \item[slot, \classref{slot}] Slot description.
    \item[slot-list, \classref{list}] List of effective slot
    descriptions.
    \item[writer, \classref{generic-function}] The slot writer.
\end{specargs}
%
\result%
Generic function.
%
\remarks%
The default method checks if there is a method in the {\em
generic-function}. If not, creates and adds a new one, otherwise it
assumes that the existing method works correctly. The domain of
the new method is {\em class} $\times$ \classref{object} and the
function is:
%
{\tt
\begin{tabbing}
    00\=00\=00\=00\= \kill%
    (method-function-lambda ((obj {\em class})\\
    \>\>(new-value \classref{object}))\\
    \>(primitive-writer obj new-value))
\end{tabbing}
}
%
\genericref{compute-primitive-writer-using-slot} is called by
\genericref{ensure-slot-writer} method to compute the primitive writer used in
the function of the new created writer method.

\generic{compute-primitive-reader-using-slot}
%
\begin{genericargs}
    \item[slot, \classref{slot}] Slot description.
    \item[class, \classref{class}] Class.
    \item[slots, \classref{list}] List of effective slot
    descriptions.
\end{genericargs}
%
\result%
Function.
%
\remarks%
Computes and returns a function which returns a slot value when
applied to an instance of {\em class}.

\method{compute-primitive-reader-using-slot}
%
\begin{specargs}
    \item[slot, \classref{slot}] Slot description.
    \item[class, \classref{class}] Class.
    \item[slots, \classref{list}] List of effective slot
    descriptions.
\end{specargs}
%
\result%
Function.
%
\remarks%
Calls \genericref{compute-primitive-reader-using-class}.  This is the default
method.

\generic{compute-primitive-reader-using-class}
%
\begin{genericargs}
    \item[class, \classref{class}] Class.
    \item[slot, \classref{slot}] Slot description.
    \item[slots, \classref{list}] List of effective slot
    descriptions.
\end{genericargs}
%
\result%
Function.
%
\remarks%
Computes and returns a function which returns the slot value when
applied to an instance of {\em class}.

\method{compute-primitive-reader-using-class}
%
\begin{specargs}
    \item[class, \classref{class}] Class.
    \item[slot, \classref{slot}] Slot description.
    \item[slots, \classref{list}] List of effective slot
    descriptions.
\end{specargs}
%
\result%
Function.
%
\remarks%
The default method returns a function of one argument.

\generic{compute-primitive-writer-using-slot}
%
\begin{genericargs}
    \item[slot, \classref{slot}] Slot description.
    \item[class, \classref{class}] Class.
    \item[slots, \classref{list}] List of effective slot
    descriptions.
\end{genericargs}
%
\result%
Function.
%
\remarks%
Computes and returns a function which stores a new slot value when
applied on an instance of {\em class} and a new value.

\method{compute-primitive-writer-using-slot}
%
\begin{specargs}
    \item[slot, \classref{slot}] Slot description.
    \item[class, \classref{class}] Class.
    \item[slots, \classref{list}] List of effective slot
    descriptions.
\end{specargs}
%
\result%
Function.
%
\remarks%
Calls \genericref{compute-primitive-writer-using-class}.  This is the default
method.

\generic{compute-primitive-writer-using-class}
%
\begin{genericargs}
    \item[class, \classref{class}] Class.
    \item[slot, \classref{slot}] Slot description.
    \item[slots, \classref{list}] List of effective slot
    descriptions.
\end{genericargs}
%
\result%
Function.
%
\remarks%
Computes and returns a function which stores the new slot value when
applied on an instance of {\em class} and new value.

\method{compute-primitive-reader-using-class}
%
\begin{specargs}
    \item[class, \classref{class}] Class.
    \item[slot, \classref{slot}] Slot description.
    \item[slots, \classref{list}] List of effective slot
    descriptions.
\end{specargs}
%
\result%
Function.
%
\remarks%
The default method returns a function of two arguments.
%
\end{optDefinition}

\sclause{Method Lookup and Generic Dispatch}
%
\begin{optDefinition}
%
\generic{compute-method-lookup-function}
%
\begin{genericargs}
    \item[gf, \classref{generic-function}] A generic function.
    \item[domain, \classref{list}] A list of classes which cover the domain.
\end{genericargs}
%
\result%
A function.
%
\remarks%
Computes and returns a function which will be called at least once for
each domain to select and sort the applicable methods by the default
dispatch mechanism.  New methods may be defined for this function to
implement different method lookup strategies. Although only one method
lookup function generating method is provided by the system, each
generic function has its own specific lookup function which may vary
from generic function to generic function.

\method{compute-method-lookup-function}
%
\begin{specargs}
    \item[gf, \classref{generic-function}] A generic function.
    \item[domain, \classref{list}] A list of classes which cover the domain.
\end{specargs}
%
\result%
A function.
%
\remarks%
Computes and returns a function which will be called at least once for
each domain to select and sort the applicable methods by the default
dispatch mechanism.  It is not defined, whether each generic function
may have its own lookup function.

\generic{compute-discriminating-function}
%
\begin{genericargs}
    \item[gf, \classref{generic-function}] A generic function.
    \item[domain, \classref{list}] A list of classes which span the domain.
    \item[lookup-fn, \classref{function}] The method lookup function.
    \item[methods, \classref{list}] A list of methods attached to the {\em
        generic-function}.
\end{genericargs}
%
\result%
A function.
%
\remarks%
This generic function computes and returns a function which is called
whenever the generic function is called. The returned function
controls the generic dispatch.  Users may define methods on this
function for new generic function classes to implement non-default
dispatch strategies.

\method{compute-discriminating-function}
%
\begin{specargs}
    \item[gf, \classref{generic-function}] A generic function.
    \item[domain, \classref{list}] A list of classes which span the domain.
    \item[lookup-fn, \classref{function}] The method lookup function.
    \item[methods, \classref{list}] A list of methods attached to the {\em
        generic-function}.
\end{specargs}
%
\result%
A function.
%
\remarks%
This method computes and returns a function which is called whenever
the generic function is called.  This default method implements the
standard dispatch strategy:  The generic function's methods are sorted
using the function returned by \genericref{compute-method-lookup-function},
and the first is called as if by \functionref{call-method}, passing the others
as the list of next methods.  Note that \functionref{call-method} need not be
directly called.

\generic{add-method}
%
\begin{genericargs}
    \item[gf, \classref{generic-function}] A generic function.
    \item[method, \classref{method}] A method to be attached to the generic
    function.
\end{genericargs}
%
\result%
This generic function adds {\em method} to the generic function {\em
gf}.  This method will then be taken into account when {\em gf} is
called with appropriate arguments.  It returns the generic function
{\em gf}.  New methods may be defined on this generic function for new
generic function and method classes.
%
\remarks%
In contrast to CLOS, \genericref{add-method} does not remove a method with
the same domain as the method being added. Instead, a noncontinuable
error is signalled.

\method{add-method}
%
\begin{specargs}
    \item[gf, \classref{generic-function}] A generic function.
    \item[method, \classref{method}] A method to be attached.
\end{specargs}
%
\result%
The generic function.
%
\remarks%
This method checks that the domain classes of the method are subclasses of those
of the generic function, and that the method is an instance of the generic
function's method class.  If not, signals an error (condition:
\conditionref{incompatible-method-and-gf}
\indexcondition{incompatible-method-and-gf}).  It also checks if there is a
method with the same domain already attached to the generic function.  If so, a
noncontinuable error is signaled (condition: \conditionref{method-domain-clash}
\indexcondition{method-domain-clash}). If no error occurs, the method is added
to the generic function.  Depending on particular optimizations of the generic
dispatch, adding a method may cause some updating computations, {\em e.g.} by
calling compute-method-lookup-function and compute-discriminating-function.
%
\end{optDefinition}
