\clause{Concurrency}
\index{general}{thread}
\index{general}{level-0 modules!thread}
\index{general}{lock}
\index{general}{level-0 modules!lock}
%
\begin{optPrivate}
Added this section since it seems to need an introduction rather than
just falling over the definition of the operations.

Integrated Russell's modifications to thread state diagram. gbn.
\end{optPrivate}
%
\begin{optDefinition}
The basic elements of parallel processing in \eulisp\ are processes and
mutual exclusion, which are provided by the classes \classref{thread}\ and
\classref{lock}\ respectively.

A thread is allocated and initialized, by calling \functionref{make}.  The
initarg of a thread specifies the initial function, which is where execution
starts the first time the thread is dispatched by the scheduler.  In this
discussion four states of a thread are identified: {\it new}, {\it running},
{\it aborted} and {\it finished}.  These are for conceptual purposes only and a
EuLisp program cannot distinguish between new and running or between aborted and
finished.  (Although accessing the result of a thread would permit such a
distinction retrospectively, since an aborted thread will cause a condition to
be signalled on the accessing thread and a finished thread will not.)  In
practice, the running state is likely to have several internal states, but these
distinctions and the information about a thread's current state can serve no
useful purpose to a running program, since the information may be incorrect as
soon as it is known.  The initial state of a thread is new.  The union of the
two final states is known as {\em determined}.  Although a program can find out
whether a thread is determined or not by means of {\tt wait} with a timeout of
\true\/ (denoting a poll), the information is only useful if the thread has been
determined.

A thread is made available for dispatch by starting it, using the function
\functionref{thread-start}, which changes its state from new to running.  After
running a thread becomes either finished or aborted.  When a thread is finished,
the result of the initial function may be accessed using
\functionref{thread-value}.  If a thread is aborted, which can only occur as a
result of a signal handled by the default handler (installed when the thread is
created), then \functionref{thread-value} will signal the condition that aborted
the thread on the thread accessing the value.  Note that
\functionref{thread-value} suspends the calling thread if the thread whose
result is sought is not determined.

While a thread is running, its progress can be suspended by accessing a lock, by
a stream operation or by calling \functionref{thread-value} on an undetermined
thread.  In each of these cases, \functionref{thread-reschedule} is called to
allow another thread to execute.  This function may also be called voluntarily.
Progress can resume when the lock becomes unlocked, the input/output operation
completes or the undetermined thread becomes determined.

The actions of a thread can be influenced externally by \functionref{signal}.
This function registers a condition to be signalled no later than when the
specified thread is rescheduled for execution---when
\functionref{thread-reschedule} returns.  The condition must be an instance of
\conditionref{thread-condition}.  Conditions are delivered to the thread in
order of receipt.  This ordering requirement is only important in the case of a
thread sending more than one signal to the same thread, but in other
circumstances the delivery order cannot be verified.  A \functionref{signal} on
a determined thread has no discernable effect on either the signalled or
signalling thread unless the condition is not an instance of
\conditionref{thread-condition}, in which case an error is signalled on the
signalling thread.  See also Section~\ref{condition}.

A lock is an abstract data type protecting a binary value which denotes whether
the lock is locked or unlocked.  The operations on a lock are \functionref{lock}
and \functionref{unlock}.  Executing a \functionref{lock} operation will
eventually give the calling thread exclusive control of a lock.  The
\functionref{unlock} operation unlocks the lock so that either a thread
subsequently calling \functionref{lock} or one of the threads which has already
called \functionref{lock} on the lock can gain exclusive access.
%
\begin{note}
    It is intended that implementations of locks based on spin-locks, semaphores
    or channels should all be capable of satisfying the above description.
    However, to be a conforming implementation, the use of a spin-lock must
    observe the fairness requirement, which demands that between attempts to
    acquire the lock, control must be ceded to the scheduler.
\end{note}
%
The programming model is that of concurrently executing threads,
regardless of whether the configuration is a multi-processor or not,
with some constraints and some weak fairness guarantees.
%
\begin{enumerate}
    \item A processor is free to use run-to-completion, timeslicing and/or
    concurrent execution.

    \item A conforming program must assume the possibility of concurrent
    execution of threads and will have the same semantics in all cases---see
    discussion of fairness which follows.

    \item The default condition handler for a new thread, when invoked, will
    change the state of the thread to {\it aborted}, save the signalled
    condition and reschedule the thread.

    \item A continuation must only be called from within its dynamic extent.
    This does not include threads created within the dynamic extent.  An error
    is signalled (condition class: \conditionref{wrong-thread-continuation}
    \indexcondition{wrong-thread-continuation}), if a continuation is called on
    a thread other than the one on which it was created.

    \item The lexical environment (inner and top) associated with the initial
    function may be shared, as is the top-dynamic environment, but each thread
    has a distinct inner-dynamic environment.  In consequence, any modifications
    of bindings in the lexical environment or in the top-dynamic environment
    should be mediated by locks to avoid non-deterministic behaviour.

    \item The creation and starting of a thread represent changes to the state
    of the processor and as such are not affected by the processor's handling of
    errors signalled subsequently on the creating/starting thread
    (c.f. streams).  That is to say, a non-local exit to a point dynamically
    outside the creation of the subsidiary thread has no default effect on the
    subsidiary thread.

    \item The behaviour of i/o on the same stream by multiple threads is
    undefined unless it is mediated by explicit locks.
\end{enumerate}

The parallel semantics are preserved on a sequential run-to-completion
implementation by requiring communication between threads to use only
thread primitives and shared data protected by locks---both the
thread primitives and locks will cause rescheduling, so other
threads can be assumed to have a chance of execution.

There is no guarantee about which thread is selected next.  However, a
fairness guarantee is needed to provide the illusion that every other
thread is running.  A strong guarantee would ensure that every other
thread gets scheduled before a thread which reschedules itself is
scheduled again.  Such a scheme is usually called ``round-robin''.
This could be stronger than the guarantee provided by a parallel
implementation or the scheduler of the host operating system and
cannot be mandated in this definition.

A weak but sufficient guarantee is that if any thread reschedules
infinitely often then every other thread will be scheduled infinitely
often.  Hence if a thread is waiting for shared data to be changed by
another thread and is using a lock, the other thread is
guaranteed to have the opportunity to change the data.  If it is not
using a lock, the fairness guarantee ensures that in the same
scenario the following loop will exit eventually:
%
{\small\syntax
(while (= data 0) (thread-reschedule))
\endsyntax}
%
\end{optDefinition}

\sclause{Threads}
\index{general}{thread}
\label{thread}
\gdef\module{thread}
%
\begin{optPrivate}
    Fixed semaphore discussion as per Harry Bretthauer's observation and changed
    from binary to counting as agreed at October '90 meeting.
\end{optPrivate}
%
\begin{optRationale}
    Many threads can exist and process concurrently.  Switching processing
    between threads can happen either when a thread voluntarily relinquishes
    control, or perhaps at regular intervals, under the control of a preemptive
    scheduler.
\end{optRationale}
%
\begin{optDefinition}
The defined name of this module is {\tt thread}.
This section defines the operations on threads.

\derivedclass{thread}{object}
\index{general}{level-0 classes!\theclass{thread}}
%
The class of all instances of \classref{thread}.
%
\begin{initoptions}
    \item[init-function, fn] an instance of \classref{function} which will be
    called when the resulting thread is started by \functionref{thread-start}.
\end{initoptions}

\function{threadp}
%
\begin{arguments}
    \item[object] An object to examine.
\end{arguments}
%
\result%
The supplied argument if it is an instance of \classref{thread}, otherwise
\nil.

\function{thread-reschedule}
%
This function takes no arguments.
%
\result%
The result is \nil.
%
\remarks%
This function is called for side-effect only and may cause the thread which
calls it to be suspended, while other threads are run.  In addition, if the
thread's condition queue is not empty, the first condition is removed from the
queue and signalled on the thread.  The resume continuation of the signal will
be one which will eventually call the continuation of the call to
\functionref{thread-reschedule}.
%
\seealso%
\functionref{thread-value}, \functionref{signal} and Section~\ref{condition} for
details of conditions and signalling.

\function{current-thread}
%
This function takes no arguments.
%
\result%
The thread on which \functionref{current-thread} was executed.

\function{thread-start}
\begin{arguments}
    \item[thread] the thread to be started, which must be new.  If {\em thread}
    is not new, an error is signalled (condition class:
    \classref{thread-already-started}\indexcondition{thread-already-started}).

    \item[obj$_1$ \ldots obj$_n$] values to be passed as the arguments to the
    initial function of {\em thread}.
\end{arguments}
%
\result%
The thread which was supplied as the first argument.
%
\remarks%
The state of thread is changed to running.  The values {\em obj$_1$} to {\em
    obj$_n$} will be passed as arguments to the initial function of {\em
    thread}.

\function{thread-value}
\begin{arguments}
    \item[thread] the thread whose finished value is to be accessed.
\end{arguments}
%
\result%
The result of the initial function applied to the arguments passed from
\functionref{thread-start}.  However, if a condition is signalled on {\em
    thread} which is handled by the default handler the condition will now be
signalled on the thread calling \functionref{thread-value}---that is the
condition will be propagated to the accessing thread.
%
\remarks%
If {\em thread} is not determined, each thread calling
\functionref{thread-value} is suspended until {\em thread} is determined, when
each will either get the thread's value or signal the condition.
%
\seealso%
\functionref{thread-reschedule}, \functionref{signal}.

\generic{wait}
%
\begin{genericargs}
%
    \item[obj] An object.
%
    \item[timeout, \classref{object}] One of \nil, \true\/ or a non-negative
    integer.
%
\end{genericargs}
%
\result%
Returns \nil\/ if {\em timeout} was reached, otherwise a
non-\nil\/ value.
%
\remarks%
\genericref{wait} provides a generic interface to operations which may block.
Execution of the current thread will continue beyond the \genericref{wait} form
only when one of the following happened:
\begin{enumerate}
    \item\label{case-a} A condition associated with {\em obj} returns true;
    \item\label{case-b} {\em timeout} time units elapse;
    \item\label{case-c} A condition is raised by another thread on this thread.
\end{enumerate}
\genericref{wait} returns \nil\/ if timeout occurs, else it returns a
non-nil value.

A {\em timeout} argument of \nil\/ or zero denotes a polling operation.  A {\em
    timeout} argument of \true\/ denotes indefinite blocking (cases \ref{case-a}
or \ref{case-c} above).  A {\em timeout} argument of a non-negative integer
denotes the minimum number of time units before timeout.  The number of time
units in a second is given by the implementation-defined constant
\constantref{ticks-per-second}.
%
\examples
This code fragment copies characters from stream {\tt s} to the
current output stream until no data is received on the stream for a
period of at least 1 second.
%
{\small\syntax
(labels
  ((loop ()
     (when (wait s (round ticks-per-second))
           (print (read-char s))
           (loop))))
   (loop))
\endsyntax}
%
\seealso%
threads (section~\ref{thread}), streams (section~\ref{stream}).

\method{wait}
%
\begin{specargs}
%
    \item[thread, \classref{thread}] The thread on which to wait.
%
    \item[timeout, <object>] The timeout period which is specified by one of
    \nil, \true, and non-negative integer.
%
\end{specargs}
%
\result%
Result is either {\em thread} or \nil.  If {\em timeout} is \nil, the result is
{\em thread} if it is {\em determined}.  If {\em timeout} is \true, {\em thread}
suspends until {\em thread} is {\em determined} and the result is guaranteed to
be {\em thread}.  If {\em timeout} is a non-negative integer, the call blocks
until either {\em thread} is determined, in which case the result is {\em
    thread}, or until the {\em timeout} period is reached, in which case the
result is \nil, whichever is the sooner.  The units for the non-negative integer
timeout are the number of clock ticks to wait.  The implementation-defined
constant \constantref{ticks-per-second} is used to make timeout periods
processor independent.
%
\seealso%
\genericref{wait} and \constantref{ticks-per-second} (Section~\ref{condition}).
%
\constant{ticks-per-second}{double-float}
%
The number of time units in a second expressed as a double precision floating
point number.  This value is
implementation-defined\index{general}{implementation-defined!time units per
    second}.

\condition{thread-condition}{condition}
%
\begin{initoptions}
    \item[current-thread, thread] The thread which is signalling the condition.
\end{initoptions}
%
\remarks%
This is the general condition class for all conditions arising from
thread operations.

\condition{wrong-thread-continuation}{thread-condition}
%
\begin{initoptions}
%
    \item[continuation, continuation] A continuation.
%
    \item[thread, thread] The thread on which {\em continuation} was created.
%
\end{initoptions}
%
\remarks%
Signalled if the given continuation is called on a thread other than
the one on which it was created.

\condition{thread-already-started}{thread-condition}
%
\begin{initoptions}
    \item[thread, thread] A thread.
\end{initoptions}
%
\remarks%
Signalled by \functionref{thread-start} if the given thread has been started
already.

\sclause{Locks}
\index{general}{lock}
\label{lock}
\gdef\module{lock}
The defined name of this module is {\tt lock}.

\derivedclass{lock}{object}
\index{general}{level-0 classes!\theclass{lock}}
%
The class of all instances of \classref{lock}.  This class has no
init-options.  The result of calling \functionref{make} on \classref{lock}
is a new, open lock.

\function{lockp}
%
\begin{arguments}
    \item[object] An object to examine.
\end{arguments}
%
\result%
The supplied argument if it is an instance of \classref{lock},
otherwise \nil.

\function{lock}
%
\begin{arguments}
    \item[lock] the lock to be acquired.
\end{arguments}
%
\result%
The lock supplied as argument.
%
\remarks%
Executing a \classref{lock} operation will eventually give the calling thread
exclusive control of {\em lock}.  A consequence of calling \classref{lock} is
that a condition from another thread may be signalled on this thread.  Such a
condition will be signalled before {\em lock} has been acquired, so a thread
which does not handle the condition will not lead to starvation; the condition
will be signalled continuably so that the process of acquiring the lock may
continue after the condition has been handled.
%
\seealso%
\functionref{unlock} and Section~\ref{condition} for details of conditions and
signalling.

\function{unlock}
%
\begin{arguments}
    \item[lock] the lock to be released.
\end{arguments}
%
\result%
The lock supplied as argument.
%
\remarks%
The \functionref{unlock} operation unlocks {\em lock} so that either a thread
subsequently calling \classref{lock} or one of the threads which has already
called \classref{lock} on the lock can gain exclusive access.
%
\seealso%
\classref{lock}.
%
\end{optDefinition}
